\chapter{Conclusão}

\lettrine{A}{} partir da análise dos dados, foi possível tirar algumas conclusões. Pôde-se ver que o município melhorou como um todo. Também se nota que ele está no caminho certo, com as matrículas aumentando e a fila diminuindo. No entanto, ainda faltam mais matrículas para que ele atinja a meta proporcional de 75\%.

A evolução do atendimento foi diferente pelos distritos. Quase todos apresentaram aumento nas matrículas, uns mais do que outros, como o distrito de Lajeado, que subiu 14 posições no \textit{ranking} de matrículas absolutas e 62 no \textit{ranking} de matrículas proporcionais à população. Em contrapartida, o distrito da República foi o único que apresentou queda nas matrículas.

Pensando nas filas, é possível confirmar que elas caíram na maioria dos distritos. No entanto, alguns distritos apresentaram aumento, como a Vila Andrade, Pedreira e Marsilac.

O distrito do Marsilac, o mais meridional da cidade, é o único que atende mais de 100\% de sua população estimada. Tal informação leva a uma reflexão quanto a acurácia das estimativas populacionais e também em relação a distritos que atendem crianças de outro distrito, este último um fato que não pode ser ignorado ao longo do consumo desta análise.

De acordo com os dados mais recentes, os distritos que apresentam o maior número de vagas se localizam na periferia das zonas Sul, Leste e Norte. São as mesmas regiões que concentram boa parte da demanda. No entanto, foram justamente essas regiões que tiveram os maiores aumentos no número de matrículas e maiores reduções na fila.

Portanto, conclui-se que embora o município de São Paulo esteja melhorando no atendimento de creches, essa melhora não é refletida de maneira igual pela cidade. Isso não deve ser considerado um fato ruim, e se deve a diversas razões. Uma delas é que a população se distribui de maneira diferente pela cidade. Uma outra é que nem todas as crianças se matriculam em creches municipais, e sim, na rede privada. Acontece também de crianças com idades entre 0 e 3 anos não estarem matriculadas em nenhum lugar, ficando em casa, com parentes, por exemplo.

As análises feitas aqui foram bem simples, e, no final, não precisaram de tanto conhecimento técnico. Para as próximas análises nesse tema, pode ser estudado quem são as crianças que frequentam as creches municipais, traçando um perfil sócioeconômico dos estudantes. Também é possível analisar quais são os critérios que mais influenciam na escolha das creches municipais. Algumas variáveis podem ser estudadas, tais como oferta de vagas no distrito e renda das famílias. É possível também pesquisar agrupamentos entre os distritos, para caracterizá-los e traçar ações em comuns para os grupos eventualmente identificados. 

