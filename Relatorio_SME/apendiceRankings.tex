\chapter{Tabelas dos \textit{rankings}}
\label{apend:rankings}

\lettrine{N}{este} apêndice, temos as tabelas com os \textit{rankings} dos distritos, como referido no \autoref{cap:rankings}, ordenados por, respectivamente: número de matrículas, tamanho absoluto da demanda, razão de matrículas pela população estimada e razão da demanda pela população estimada.

Todas as tabelas possuem cinco colunas, e as quatro primeiras são iguais, sendo:

\begin{itemize}
	\item \textbf{Posição}: A colocação em que o distrito se encontra no determinado \textit{ranking} em dezembro de 2017.
	\item \textbf{Variação}: Apresenta duas informações e compara o \textit{ranking} de dezembro de 2017 com o \textit{ranking} de 2006 (em junho para os \textit{rankings} de matrícula ou em dezembro, para de demanda). Se a variação foi positiva, é exibida uma seta verde apontada para cima (\aumento); se foi negativa, é mostrada uma seta vermelha apontada para baixo (\queda); se não houve variação, é observado um traço azul (\mesmo). Quando há variação, a tabela apresenta também o tamanho da variação.
	\item \textbf{Distrito}: Nome do distrito analisado.
	\item \textbf{Zona}: Considerando a divisão da cidade em 5 zonas (Norte, Sul, Leste, Oeste e Centro), mostra a zona do distrito analisado.
\end{itemize}

A última coluna da tabela se refere ao dado em questão, podendo ser, na ordem em que as tabelas são apresentadas:

\begin{itemize}
	\item \textbf{Matrículas}: o número de matrículas total num distrito.
	\item \textbf{Demanda}: o número de crianças na fila nos distritos.
	\item \textbf{Razão de Matrículas}: a porcentagem de crianças matriculadas nas creches municipais, em relação à população estimada.
	\item \textbf{Razão da demanda}: a porcentagem de crianças na fila das creches municipais, em relação à população estimada.
\end{itemize}