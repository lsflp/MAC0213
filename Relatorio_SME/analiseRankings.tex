\chapter{\textit{Rankings}}
\label{cap:rankings}

\lettrine{P}{ara} fazer uma classificação dos distritos, assim como se faz em campeonatos de futebol, por exemplo, foram gerados quatro \textit{rankings} em relação ao atendimento municipal de creches. Tais \textit{rankings} podem ser vistos, na íntegra, no \autoref{apend:rankings}.

\section{Número de matrículas absolutas}

Esse \textit{ranking} compara a situação municipal de junho de 2006 com a de dezembro de 2017. Embora esse número tenha aumentado em 95 dos 96 distritos, os distritos com os maiores números de vagas mudaram, como pôde ser visto no \autoref{cap:distribuicao}.

Dos 96 distritos, 20 possuem mais de 5000 matrículas cada. Por outro lado, 26 distritos possuem menos que 1000 matrículas cada.

A maior ascensão no \textit{ranking} foi do distrito Anhanguera (38 posições). A maior queda pertence à Mooca (32 posições).

No topo da lista, as 10 primeiras posições possuem 4 distritos da zona Sul e 4 da zona Leste. No fim da lista (10 últimos), temos 5 distritos da zona Oeste e 3 do Centro.

\section{Tamanho absoluto da demanda}

Esse \textit{ranking} compara a situação municipal de dezembro de 2006 com a de dezembro de 2017, por causa da sazonalidade da fila. Analogamente à seção anterior, embora a fila tenha sido reduzida na maior parte da cidade, os distritos com mais fila mudaram.

Pode-se ver aqui que 12 distritos possuem mais de 1000 crianças na fila cada. Por outro lado, 13 distritos possuem menos de 100 crianças na demanda por vagas.

A maior ascensão no \textit{ranking} foi do distrito Morumbi (47 posições). A maior queda pertence aos distritos Penha e Ponte Rasa (45 posições).

Das 10 primeiras posições, 7 são de distritos da zona Sul. Nas 10 últimas posições, 4 são da zona Oeste e 3 pertencem ao Centro.

\section{Razão de matrículas pela população estimada}

Esse \textit{ranking} compara a situação municipal de junho de 2006 com a de dezembro de 2017, analisando quanto da população estimada é atendida pelas creches municipais. Um resultado de 100\% indica que toda a população de 0 a 3 anos seria atendida.

Dos 96 distritos, 31 atendem mais de 50\% de sua população estimada. Pelo outro lado, 14 atendem menos do que 20\%. 

A maior ascensão nesse \textit{ranking} foi do distrito Anhanguera (69 posições). A maior queda pertence à Mooca (75 posições).

Nas 10 primeiras posições, podem ser encontrados 5 distritos da zona Leste. No lado oposto (10 últimos), 4 são da zona Oeste e 3 pertencem ao Centro.

\section{Razão da demanda pela população}

Esse \textit{ranking} compara a situação municipal de dezembro de 2006 com a de dezembro de 2017, como já explicado anteriormente. É analisando quanto da população estimada está na fila das creches municipais. Um resultado de 100\% indica que toda a população de 0 a 3 anos está na fila.

Dos 96 distritos, 8 possuem mais de 15\% de sua população estimada na fila. Pelo outro lado, 51 possuem menos do que 5\% na espera. 

A maior ascensão nesse \textit{ranking} foi do distrito Marsilac (95 posições), ou seja, ele saiu do último e foi para o primeiro lugar. A maior queda pertence ao Butantã (81 posições).

Nas 10 primeiras posições, podem ser encontrados 5 distritos da zona Sul. Dos 10 últimos, 6 são da zona Leste.