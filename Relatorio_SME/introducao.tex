\chapter{Introdução}

\lettrine{E}{ste} projeto é fruto de uma parceria entre o Tecs - Grupo de computação social da USP e a Secretaria Municipal de Educação (SME) de São Paulo. Ele é uma Análise Histórica dos Dados sobre o Atendimento das Creches de São Paulo, de junho de 2006 a dezembro de 2017.

Para sua realização, foram usadas algumas fontes de dados. Uma delas foram os dados educacionais, provenientes do \textit{site}\footnote{\url{http://dados.prefeitura.sp.gov.br/}} da Prefeitura. Deles, apenas o recorte relacionado às creches (Matrículas e Demanda) foi utilizado.

Outra fonte foi a de dados populacionais, vindos do \textit{site}\footnote{\url{http://www.seade.gov.br/}} da Fundação SEADE. Desses dados, apenas a faixa etária de 0 a 3 anos foi utilizada. Junto com os dados educacionais selecionados, esses dados foram salvos num arquivo separado, utilizado para todas as análises.

Além desses dados, foram usados os dados das subdivisões de São Paulo em cinco zonas, de acordo com o Plano Diretor Estratégico de 2002 (PDE), também proveniente do \textit{site} da Prefeitura.

O desenvolvimento da análise foi inteiramente feito em Python 3\footnote{\url{https://www.python.org/}}. A ferramenta utilizada para os códigos foi a Jupyter\footnote{\url{http://jupyter.org/}}, que coloca o desenvolvimento numa janela de navegador Web. A principal biblioteca de análise utilizada foi a Pandas\footnote{\url{https://pandas.pydata.org/}}, que permite a conversão de dados .csv para estruturas de dados com usabilidade simples. Para os gráficos, foi usada a Matplotlib\footnote{\url{https://matplotlib.org/}}.

A análise gerou 10 arquivos de código-fonte, que podem ser vistos no repositório\footnote{\url{https://www.github.com/lsflp/MAC0213}} criado especialmente para esta disciplina.