\chapter{Dados da Análise Distrital}
\label{cap:apendDist}

\lettrine{A}{s} análises apresentadas a seguir seguem um padrão, com os dados sendo apresentados de maneira muito parecida com a Análise Municipal. São divididas em três partes, sendo uma figura e duas tabelas.

A figura mostra duas informações: a primeira é um gráfico que mostra a evolução do atendimento e da população no distrito; a segunda é a localização do distrito no município.

A primeira tabela exibe as informações do primeiro período analisado e do último, junho de 2006 e dezembro de 2017, respectivamente. Assim como na análise anterior, ela mostra o número de matrículas, o número de crianças na fila, a população de 0 a 3 anos estimada e as matrículas e fila relativas à população do distrito.

Além disso, essa tabela apresenta informações comparativas do distrito com o município. Cinco novas linhas foram adicionadas. As três primeiras exibem qual a proporção que as matrículas, a demanda e a população do distrito representam em relação aos dados municipais. As últimas duas mostram a localização do distrito em dois \textit{rankings}: um referente ao número de matrículas e o outro, ao tamanho da demanda. Tais listagens são exibidas na íntegra no \autoref{cap:rankings}.

Como essa informação pode levar a erros de interpretação, sobre a demanda como escrito no \autoref{cap:sp}, foi colocado também uma tabela que faz uma comparação entre a demanda nos meses de dezembro de 2006 e dezembro de 2017, assim como na Análise Municipal.