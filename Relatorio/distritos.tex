\section{Água Rasa}
\begin{figure}[H]
	\centering
	\includegraphics[width=0.66\linewidth]{../Analises/graficos/tempo/tempo_AGUA_RASA}
	\includegraphics[width=0.33\linewidth]{../Analises/mapas/mapas/distrito_ARA}
	\caption{Evolução das matrículas, da demanda, e da população de 0 a 3 anos no distrito Água Rasa e a sua localização no município, respectivamente.}
\end{figure}
\begin{table}[H]
	\begin{tabular}{|l|l|l|l|}
		\hline
		\textbf{}                                      & \textbf{junho de 2006}       & \textbf{dezembro de 2017}    & \textbf{Variação} \\ \hline
		\textbf{Número de matrículas}                  & 573 & 1045 & 82.37\% \\ \hline
		\textbf{Crianças na fila}                      & 505 & 102 & -79.8\% \\ \hline
		\textbf{População de 0 a 3 anos estimada}      & 3468 & 3772 & 8.77\% \\ \hline
		\textbf{Matrículas relativas à população}      & 16.52\% & 27.7\% & 67.68\% \\ \hline
		\textbf{Fila relativa à população}             & 14.56\% & 2.7\% & -81.43\% \\ \hline
		\textbf{Proporção das matrículas no município} & 0.93\% & 0.35\% & -62.0\% \\ \hline
		\textbf{Proporção da demanda no município}     & 0.6\% & 0.23\% & -61.33\% \\ \hline
		\textbf{Proporção da população de 0 a 3 anos}  & 0.55\% & 0.59\% & 5.97\% \\ \hline
		\textbf{Posição no \textit{ranking} das matrículas}     & 41 & 69 & -28 \\ \hline
		\textbf{Posição no \textit{ranking} da demanda}         & 56 & 81 & -25 \\ \hline
	\end{tabular}
	\caption{Comparação entre o primeiro e o último período da amostra}
\end{table}
\begin{table}[H]
	\begin{tabular}{|l|l|l|l|}
		\hline
		\textbf{}                                 & \textbf{dezembro de 2006} & \textbf{dezembro de 2017} & \textbf{Variação} \\ \hline
		\textbf{Crianças na fila}                      & 683 & 102 & -85.07\% \\ \hline
		\textbf{População de 0 a 3 anos estimada}      & 3468 & 3772 & 8.77\% \\ \hline
		\textbf{Fila relativa à população}             & 18.28\% & 2.7\% & -85.21\% \\ \hline
	\end{tabular}
	\caption{Comparação da demanda no mês de dezembro, em 2006 e 2017.}
\end{table}
\section{Alto de Pinheiros}
\begin{figure}[H]
	\centering
	\includegraphics[width=0.66\linewidth]{../Analises/graficos/tempo/tempo_ALTO_DE_PINHEIROS}
	\includegraphics[width=0.33\linewidth]{../Analises/mapas/mapas/distrito_API}
	\caption{Evolução das matrículas, da demanda, e da população de 0 a 3 anos no distrito Alto de Pinheiros e a sua localização no município, respectivamente.}
\end{figure}
\begin{table}[H]
	\begin{tabular}{|l|l|l|l|}
		\hline
		\textbf{}                                      & \textbf{junho de 2006}       & \textbf{dezembro de 2017}    & \textbf{Variação} \\ \hline
		\textbf{Número de matrículas}                  & 236 & 404 & 71.19\% \\ \hline
		\textbf{Crianças na fila}                      & 64 & 43 & -32.81\% \\ \hline
		\textbf{População de 0 a 3 anos estimada}      & 1454 & 1340 & -7.84\% \\ \hline
		\textbf{Matrículas relativas à população}      & 16.23\% & 30.15\% & 85.75\% \\ \hline
		\textbf{Fila relativa à população}             & 4.4\% & 3.21\% & -27.1\% \\ \hline
		\textbf{Proporção das matrículas no município} & 0.38\% & 0.14\% & -64.33\% \\ \hline
		\textbf{Proporção da demanda no município}     & 0.08\% & 0.1\% & 28.62\% \\ \hline
		\textbf{Proporção da população de 0 a 3 anos}  & 0.23\% & 0.21\% & -10.21\% \\ \hline
		\textbf{Posição no \textit{ranking} das matrículas}     & 72 & 87 & -15 \\ \hline
		\textbf{Posição no \textit{ranking} da demanda}         & 94 & 91 & 3 \\ \hline
	\end{tabular}
	\caption{Comparação entre o primeiro e o último período da amostra}
\end{table}
\begin{table}[H]
	\begin{tabular}{|l|l|l|l|}
		\hline
		\textbf{}                                 & \textbf{dezembro de 2006} & \textbf{dezembro de 2017} & \textbf{Variação} \\ \hline
		\textbf{Crianças na fila}                      & 170 & 43 & -74.71\% \\ \hline
		\textbf{População de 0 a 3 anos estimada}      & 1454 & 1340 & -7.84\% \\ \hline
		\textbf{Fila relativa à população}             & 17.06\% & 3.21\% & -81.19\% \\ \hline
	\end{tabular}
	\caption{Comparação da demanda no mês de dezembro, em 2006 e 2017.}
\end{table}
\section{Anhanguera}
\begin{figure}[H]
	\centering
	\includegraphics[width=0.66\linewidth]{../Analises/graficos/tempo/tempo_ANHANGUERA}
	\includegraphics[width=0.33\linewidth]{../Analises/mapas/mapas/distrito_ANH}
	\caption{Evolução das matrículas, da demanda, e da população de 0 a 3 anos no distrito Anhanguera e a sua localização no município, respectivamente.}
\end{figure}
\begin{table}[H]
	\begin{tabular}{|l|l|l|l|}
		\hline
		\textbf{}                                      & \textbf{junho de 2006}       & \textbf{dezembro de 2017}    & \textbf{Variação} \\ \hline
		\textbf{Número de matrículas}                  & 207 & 2668 & 1188.89\% \\ \hline
		\textbf{Crianças na fila}                      & 138 & 382 & 176.81\% \\ \hline
		\textbf{População de 0 a 3 anos estimada}      & 4023 & 4660 & 15.83\% \\ \hline
		\textbf{Matrículas relativas à população}      & 5.15\% & 57.25\% & 1012.7\% \\ \hline
		\textbf{Fila relativa à população}             & 3.43\% & 8.2\% & 138.97\% \\ \hline
		\textbf{Proporção das matrículas no município} & 0.34\% & 0.9\% & 168.55\% \\ \hline
		\textbf{Proporção da demanda no município}     & 0.16\% & 0.87\% & 429.92\% \\ \hline
		\textbf{Proporção da população de 0 a 3 anos}  & 0.64\% & 0.72\% & 12.86\% \\ \hline
		\textbf{Posição no \textit{ranking} das matrículas}     & 79 & 41 & 38 \\ \hline
		\textbf{Posição no \textit{ranking} da demanda}         & 89 & 29 & 60 \\ \hline
	\end{tabular}
	\caption{Comparação entre o primeiro e o último período da amostra}
\end{table}
\begin{table}[H]
	\begin{tabular}{|l|l|l|l|}
		\hline
		\textbf{}                                 & \textbf{dezembro de 2006} & \textbf{dezembro de 2017} & \textbf{Variação} \\ \hline
		\textbf{Crianças na fila}                      & 424 & 382 & -9.91\% \\ \hline
		\textbf{População de 0 a 3 anos estimada}      & 4023 & 4660 & 15.83\% \\ \hline
		\textbf{Fila relativa à população}             & 6.14\% & 8.2\% & 33.51\% \\ \hline
	\end{tabular}
	\caption{Comparação da demanda no mês de dezembro, em 2006 e 2017.}
\end{table}
\section{Aricanduva}
\begin{figure}[H]
	\centering
	\includegraphics[width=0.66\linewidth]{../Analises/graficos/tempo/tempo_ARICANDUVA}
	\includegraphics[width=0.33\linewidth]{../Analises/mapas/mapas/distrito_ARI}
	\caption{Evolução das matrículas, da demanda, e da população de 0 a 3 anos no distrito Aricanduva e a sua localização no município, respectivamente.}
\end{figure}
\begin{table}[H]
	\begin{tabular}{|l|l|l|l|}
		\hline
		\textbf{}                                      & \textbf{junho de 2006}       & \textbf{dezembro de 2017}    & \textbf{Variação} \\ \hline
		\textbf{Número de matrículas}                  & 435 & 1583 & 263.91\% \\ \hline
		\textbf{Crianças na fila}                      & 523 & 364 & -30.4\% \\ \hline
		\textbf{População de 0 a 3 anos estimada}      & 4737 & 4156 & -12.27\% \\ \hline
		\textbf{Matrículas relativas à população}      & 9.18\% & 38.09\% & 314.78\% \\ \hline
		\textbf{Fila relativa à população}             & 11.04\% & 8.76\% & -20.67\% \\ \hline
		\textbf{Proporção das matrículas no município} & 0.7\% & 0.53\% & -24.18\% \\ \hline
		\textbf{Proporção da demanda no município}     & 0.62\% & 0.83\% & 33.24\% \\ \hline
		\textbf{Proporção da população de 0 a 3 anos}  & 0.75\% & 0.65\% & -14.52\% \\ \hline
		\textbf{Posição no \textit{ranking} das matrículas}     & 54 & 55 & -1 \\ \hline
		\textbf{Posição no \textit{ranking} da demanda}         & 54 & 31 & 23 \\ \hline
	\end{tabular}
	\caption{Comparação entre o primeiro e o último período da amostra}
\end{table}
\begin{table}[H]
	\begin{tabular}{|l|l|l|l|}
		\hline
		\textbf{}                                 & \textbf{dezembro de 2006} & \textbf{dezembro de 2017} & \textbf{Variação} \\ \hline
		\textbf{Crianças na fila}                      & 717 & 364 & -49.23\% \\ \hline
		\textbf{População de 0 a 3 anos estimada}      & 4737 & 4156 & -12.27\% \\ \hline
		\textbf{Fila relativa à população}             & 9.71\% & 8.76\% & -9.8\% \\ \hline
	\end{tabular}
	\caption{Comparação da demanda no mês de dezembro, em 2006 e 2017.}
\end{table}
\section{Artur Alvim}
\begin{figure}[H]
	\centering
	\includegraphics[width=0.66\linewidth]{../Analises/graficos/tempo/tempo_ARTUR_ALVIM}
	\includegraphics[width=0.33\linewidth]{../Analises/mapas/mapas/distrito_AAL}
	\caption{Evolução das matrículas, da demanda, e da população de 0 a 3 anos no distrito Artur Alvim e a sua localização no município, respectivamente.}
\end{figure}
\begin{table}[H]
	\begin{tabular}{|l|l|l|l|}
		\hline
		\textbf{}                                      & \textbf{junho de 2006}       & \textbf{dezembro de 2017}    & \textbf{Variação} \\ \hline
		\textbf{Número de matrículas}                  & 901 & 2620 & 190.79\% \\ \hline
		\textbf{Crianças na fila}                      & 670 & 153 & -77.16\% \\ \hline
		\textbf{População de 0 a 3 anos estimada}      & 5689 & 4910 & -13.69\% \\ \hline
		\textbf{Matrículas relativas à população}      & 15.84\% & 53.36\% & 236.92\% \\ \hline
		\textbf{Fila relativa à população}             & 11.78\% & 3.12\% & -73.54\% \\ \hline
		\textbf{Proporção das matrículas no município} & 1.46\% & 0.88\% & -39.41\% \\ \hline
		\textbf{Proporção da demanda no município}     & 0.79\% & 0.35\% & -56.28\% \\ \hline
		\textbf{Proporção da população de 0 a 3 anos}  & 0.91\% & 0.76\% & -15.91\% \\ \hline
		\textbf{Posição no \textit{ranking} das matrículas}     & 25 & 43 & -18 \\ \hline
		\textbf{Posição no \textit{ranking} da demanda}         & 41 & 63 & -22 \\ \hline
	\end{tabular}
	\caption{Comparação entre o primeiro e o último período da amostra}
\end{table}
\begin{table}[H]
	\begin{tabular}{|l|l|l|l|}
		\hline
		\textbf{}                                 & \textbf{dezembro de 2006} & \textbf{dezembro de 2017} & \textbf{Variação} \\ \hline
		\textbf{Crianças na fila}                      & 1368 & 153 & -88.82\% \\ \hline
		\textbf{População de 0 a 3 anos estimada}      & 5689 & 4910 & -13.69\% \\ \hline
		\textbf{Fila relativa à população}             & 15.01\% & 3.12\% & -79.24\% \\ \hline
	\end{tabular}
	\caption{Comparação da demanda no mês de dezembro, em 2006 e 2017.}
\end{table}
\section{Barra Funda}
\begin{figure}[H]
	\centering
	\includegraphics[width=0.66\linewidth]{../Analises/graficos/tempo/tempo_BARRA_FUNDA}
	\includegraphics[width=0.33\linewidth]{../Analises/mapas/mapas/distrito_BFU}
	\caption{Evolução das matrículas, da demanda, e da população de 0 a 3 anos no distrito Barra Funda e a sua localização no município, respectivamente.}
\end{figure}
\begin{table}[H]
	\begin{tabular}{|l|l|l|l|}
		\hline
		\textbf{}                                      & \textbf{junho de 2006}       & \textbf{dezembro de 2017}    & \textbf{Variação} \\ \hline
		\textbf{Número de matrículas}                  & 220 & 337 & 53.18\% \\ \hline
		\textbf{Crianças na fila}                      & 196 & 26 & -86.73\% \\ \hline
		\textbf{População de 0 a 3 anos estimada}      & 574 & 813 & 41.64\% \\ \hline
		\textbf{Matrículas relativas à população}      & 38.33\% & 41.45\% & 8.15\% \\ \hline
		\textbf{Fila relativa à população}             & 34.15\% & 3.2\% & -90.63\% \\ \hline
		\textbf{Proporção das matrículas no município} & 0.36\% & 0.11\% & -68.08\% \\ \hline
		\textbf{Proporção da demanda no município}     & 0.23\% & 0.06\% & -74.61\% \\ \hline
		\textbf{Proporção da população de 0 a 3 anos}  & 0.09\% & 0.13\% & 38.0\% \\ \hline
		\textbf{Posição no \textit{ranking} das matrículas}     & 75 & 92 & -17 \\ \hline
		\textbf{Posição no \textit{ranking} da demanda}         & 82 & 93 & -11 \\ \hline
	\end{tabular}
	\caption{Comparação entre o primeiro e o último período da amostra}
\end{table}
\begin{table}[H]
	\begin{tabular}{|l|l|l|l|}
		\hline
		\textbf{}                                 & \textbf{dezembro de 2006} & \textbf{dezembro de 2017} & \textbf{Variação} \\ \hline
		\textbf{Crianças na fila}                      & 266 & 26 & -90.23\% \\ \hline
		\textbf{População de 0 a 3 anos estimada}      & 574 & 813 & 41.64\% \\ \hline
		\textbf{Fila relativa à população}             & 45.12\% & 3.2\% & -92.91\% \\ \hline
	\end{tabular}
	\caption{Comparação da demanda no mês de dezembro, em 2006 e 2017.}
\end{table}
\section{Bela Vista}
\begin{figure}[H]
	\centering
	\includegraphics[width=0.66\linewidth]{../Analises/graficos/tempo/tempo_BELA_VISTA}
	\includegraphics[width=0.33\linewidth]{../Analises/mapas/mapas/distrito_BVI}
	\caption{Evolução das matrículas, da demanda, e da população de 0 a 3 anos no distrito Bela Vista e a sua localização no município, respectivamente.}
\end{figure}
\begin{table}[H]
	\begin{tabular}{|l|l|l|l|}
		\hline
		\textbf{}                                      & \textbf{junho de 2006}       & \textbf{dezembro de 2017}    & \textbf{Variação} \\ \hline
		\textbf{Número de matrículas}                  & 493 & 1509 & 206.09\% \\ \hline
		\textbf{Crianças na fila}                      & 738 & 387 & -47.56\% \\ \hline
		\textbf{População de 0 a 3 anos estimada}      & 2342 & 3350 & 43.04\% \\ \hline
		\textbf{Matrículas relativas à população}      & 21.05\% & 45.04\% & 113.99\% \\ \hline
		\textbf{Fila relativa à população}             & 31.51\% & 11.55\% & -63.34\% \\ \hline
		\textbf{Proporção das matrículas no município} & 0.8\% & 0.51\% & -36.22\% \\ \hline
		\textbf{Proporção da demanda no município}     & 0.87\% & 0.88\% & 0.39\% \\ \hline
		\textbf{Proporção da população de 0 a 3 anos}  & 0.37\% & 0.52\% & 39.37\% \\ \hline
		\textbf{Posição no \textit{ranking} das matrículas}     & 46 & 60 & -14 \\ \hline
		\textbf{Posição no \textit{ranking} da demanda}         & 38 & 28 & 10 \\ \hline
	\end{tabular}
	\caption{Comparação entre o primeiro e o último período da amostra}
\end{table}
\begin{table}[H]
	\begin{tabular}{|l|l|l|l|}
		\hline
		\textbf{}                                 & \textbf{dezembro de 2006} & \textbf{dezembro de 2017} & \textbf{Variação} \\ \hline
		\textbf{Crianças na fila}                      & 808 & 387 & -52.1\% \\ \hline
		\textbf{População de 0 a 3 anos estimada}      & 2342 & 3350 & 43.04\% \\ \hline
		\textbf{Fila relativa à população}             & 25.83\% & 11.55\% & -55.28\% \\ \hline
	\end{tabular}
	\caption{Comparação da demanda no mês de dezembro, em 2006 e 2017.}
\end{table}
\section{Belém}
\begin{figure}[H]
	\centering
	\includegraphics[width=0.66\linewidth]{../Analises/graficos/tempo/tempo_BELEM}
	\includegraphics[width=0.33\linewidth]{../Analises/mapas/mapas/distrito_BEL}
	\caption{Evolução das matrículas, da demanda, e da população de 0 a 3 anos no distrito Belém e a sua localização no município, respectivamente.}
\end{figure}
\begin{table}[H]
	\begin{tabular}{|l|l|l|l|}
		\hline
		\textbf{}                                      & \textbf{junho de 2006}       & \textbf{dezembro de 2017}    & \textbf{Variação} \\ \hline
		\textbf{Número de matrículas}                  & 154 & 1145 & 643.51\% \\ \hline
		\textbf{Crianças na fila}                      & 285 & 198 & -30.53\% \\ \hline
		\textbf{População de 0 a 3 anos estimada}      & 2058 & 3318 & 61.22\% \\ \hline
		\textbf{Matrículas relativas à população}      & 7.48\% & 34.51\% & 361.16\% \\ \hline
		\textbf{Fila relativa à população}             & 13.85\% & 5.97\% & -56.91\% \\ \hline
		\textbf{Proporção das matrículas no município} & 0.25\% & 0.39\% & 54.92\% \\ \hline
		\textbf{Proporção da demanda no município}     & 0.34\% & 0.45\% & 33.0\% \\ \hline
		\textbf{Proporção da população de 0 a 3 anos}  & 0.33\% & 0.52\% & 57.08\% \\ \hline
		\textbf{Posição no \textit{ranking} das matrículas}     & 84 & 67 & 17 \\ \hline
		\textbf{Posição no \textit{ranking} da demanda}         & 75 & 51 & 24 \\ \hline
	\end{tabular}
	\caption{Comparação entre o primeiro e o último período da amostra}
\end{table}
\begin{table}[H]
	\begin{tabular}{|l|l|l|l|}
		\hline
		\textbf{}                                 & \textbf{dezembro de 2006} & \textbf{dezembro de 2017} & \textbf{Variação} \\ \hline
		\textbf{Crianças na fila}                      & 532 & 198 & -62.78\% \\ \hline
		\textbf{População de 0 a 3 anos estimada}      & 2058 & 3318 & 61.22\% \\ \hline
		\textbf{Fila relativa à população}             & 9.77\% & 5.97\% & -38.92\% \\ \hline
	\end{tabular}
	\caption{Comparação da demanda no mês de dezembro, em 2006 e 2017.}
\end{table}
\section{Bom Retiro}
\begin{figure}[H]
	\centering
	\includegraphics[width=0.66\linewidth]{../Analises/graficos/tempo/tempo_BOM_RETIRO}
	\includegraphics[width=0.33\linewidth]{../Analises/mapas/mapas/distrito_BRE}
	\caption{Evolução das matrículas, da demanda, e da população de 0 a 3 anos no distrito Bom Retiro e a sua localização no município, respectivamente.}
\end{figure}
\begin{table}[H]
	\begin{tabular}{|l|l|l|l|}
		\hline
		\textbf{}                                      & \textbf{junho de 2006}       & \textbf{dezembro de 2017}    & \textbf{Variação} \\ \hline
		\textbf{Número de matrículas}                  & 452 & 910 & 101.33\% \\ \hline
		\textbf{Crianças na fila}                      & 640 & 319 & -50.16\% \\ \hline
		\textbf{População de 0 a 3 anos estimada}      & 1651 & 2392 & 44.88\% \\ \hline
		\textbf{Matrículas relativas à população}      & 27.38\% & 38.04\% & 38.96\% \\ \hline
		\textbf{Fila relativa à população}             & 38.76\% & 13.34\% & -65.6\% \\ \hline
		\textbf{Proporção das matrículas no município} & 0.73\% & 0.31\% & -58.05\% \\ \hline
		\textbf{Proporção da demanda no município}     & 0.76\% & 0.72\% & -4.58\% \\ \hline
		\textbf{Proporção da população de 0 a 3 anos}  & 0.26\% & 0.37\% & 41.16\% \\ \hline
		\textbf{Posição no \textit{ranking} das matrículas}     & 51 & 73 & -22 \\ \hline
		\textbf{Posição no \textit{ranking} da demanda}         & 43 & 36 & 7 \\ \hline
	\end{tabular}
	\caption{Comparação entre o primeiro e o último período da amostra}
\end{table}
\begin{table}[H]
	\begin{tabular}{|l|l|l|l|}
		\hline
		\textbf{}                                 & \textbf{dezembro de 2006} & \textbf{dezembro de 2017} & \textbf{Variação} \\ \hline
		\textbf{Crianças na fila}                      & 711 & 319 & -55.13\% \\ \hline
		\textbf{População de 0 a 3 anos estimada}      & 1651 & 2392 & 44.88\% \\ \hline
		\textbf{Fila relativa à população}             & 28.59\% & 13.34\% & -53.35\% \\ \hline
	\end{tabular}
	\caption{Comparação da demanda no mês de dezembro, em 2006 e 2017.}
\end{table}
\section{Brás}
\begin{figure}[H]
	\centering
	\includegraphics[width=0.66\linewidth]{../Analises/graficos/tempo/tempo_BRAS}
	\includegraphics[width=0.33\linewidth]{../Analises/mapas/mapas/distrito_BRS}
	\caption{Evolução das matrículas, da demanda, e da população de 0 a 3 anos no distrito Brás e a sua localização no município, respectivamente.}
\end{figure}
\begin{table}[H]
	\begin{tabular}{|l|l|l|l|}
		\hline
		\textbf{}                                      & \textbf{junho de 2006}       & \textbf{dezembro de 2017}    & \textbf{Variação} \\ \hline
		\textbf{Número de matrículas}                  & 199 & 511 & 156.78\% \\ \hline
		\textbf{Crianças na fila}                      & 253 & 304 & 20.16\% \\ \hline
		\textbf{População de 0 a 3 anos estimada}      & 1473 & 2304 & 56.42\% \\ \hline
		\textbf{Matrículas relativas à população}      & 13.51\% & 22.18\% & 64.17\% \\ \hline
		\textbf{Fila relativa à população}             & 17.18\% & 13.19\% & -23.18\% \\ \hline
		\textbf{Proporção das matrículas no município} & 0.32\% & 0.17\% & -46.5\% \\ \hline
		\textbf{Proporção da demanda no município}     & 0.3\% & 0.69\% & 130.03\% \\ \hline
		\textbf{Proporção da população de 0 a 3 anos}  & 0.23\% & 0.36\% & 52.4\% \\ \hline
		\textbf{Posição no \textit{ranking} das matrículas}     & 80 & 83 & -3 \\ \hline
		\textbf{Posição no \textit{ranking} da demanda}         & 76 & 39 & 37 \\ \hline
	\end{tabular}
	\caption{Comparação entre o primeiro e o último período da amostra}
\end{table}
\begin{table}[H]
	\begin{tabular}{|l|l|l|l|}
		\hline
		\textbf{}                                 & \textbf{dezembro de 2006} & \textbf{dezembro de 2017} & \textbf{Variação} \\ \hline
		\textbf{Crianças na fila}                      & 617 & 304 & -50.73\% \\ \hline
		\textbf{População de 0 a 3 anos estimada}      & 1473 & 2304 & 56.42\% \\ \hline
		\textbf{Fila relativa à população}             & 20.84\% & 13.19\% & -36.69\% \\ \hline
	\end{tabular}
	\caption{Comparação da demanda no mês de dezembro, em 2006 e 2017.}
\end{table}
\section{Brasilândia}
\begin{figure}[H]
	\centering
	\includegraphics[width=0.66\linewidth]{../Analises/graficos/tempo/tempo_BRASILANDIA}
	\includegraphics[width=0.33\linewidth]{../Analises/mapas/mapas/distrito_BRL}
	\caption{Evolução das matrículas, da demanda, e da população de 0 a 3 anos no distrito Brasilândia e a sua localização no município, respectivamente.}
\end{figure}
\begin{table}[H]
	\begin{tabular}{|l|l|l|l|}
		\hline
		\textbf{}                                      & \textbf{junho de 2006}       & \textbf{dezembro de 2017}    & \textbf{Variação} \\ \hline
		\textbf{Número de matrículas}                  & 2121 & 11488 & 441.63\% \\ \hline
		\textbf{Crianças na fila}                      & 2392 & 914 & -61.79\% \\ \hline
		\textbf{População de 0 a 3 anos estimada}      & 18269 & 19242 & 5.33\% \\ \hline
		\textbf{Matrículas relativas à população}      & 11.61\% & 59.7\% & 414.24\% \\ \hline
		\textbf{Fila relativa à população}             & 13.09\% & 4.75\% & -63.72\% \\ \hline
		\textbf{Proporção das matrículas no município} & 3.44\% & 3.88\% & 12.85\% \\ \hline
		\textbf{Proporção da demanda no município}     & 2.83\% & 2.07\% & -26.85\% \\ \hline
		\textbf{Proporção da população de 0 a 3 anos}  & 2.91\% & 2.99\% & 2.62\% \\ \hline
		\textbf{Posição no \textit{ranking} das matrículas}     & 4 & 2 & 2 \\ \hline
		\textbf{Posição no \textit{ranking} da demanda}         & 8 & 13 & -5 \\ \hline
	\end{tabular}
	\caption{Comparação entre o primeiro e o último período da amostra}
\end{table}
\begin{table}[H]
	\begin{tabular}{|l|l|l|l|}
		\hline
		\textbf{}                                 & \textbf{dezembro de 2006} & \textbf{dezembro de 2017} & \textbf{Variação} \\ \hline
		\textbf{Crianças na fila}                      & 3361 & 914 & -72.81\% \\ \hline
		\textbf{População de 0 a 3 anos estimada}      & 18269 & 19242 & 5.33\% \\ \hline
		\textbf{Fila relativa à população}             & 9.73\% & 4.75\% & -51.18\% \\ \hline
	\end{tabular}
	\caption{Comparação da demanda no mês de dezembro, em 2006 e 2017.}
\end{table}
\section{Butantã}
\begin{figure}[H]
	\centering
	\includegraphics[width=0.66\linewidth]{../Analises/graficos/tempo/tempo_BUTANTA}
	\includegraphics[width=0.33\linewidth]{../Analises/mapas/mapas/distrito_BUT}
	\caption{Evolução das matrículas, da demanda, e da população de 0 a 3 anos no distrito Butantã e a sua localização no município, respectivamente.}
\end{figure}
\begin{table}[H]
	\begin{tabular}{|l|l|l|l|}
		\hline
		\textbf{}                                      & \textbf{junho de 2006}       & \textbf{dezembro de 2017}    & \textbf{Variação} \\ \hline
		\textbf{Número de matrículas}                  & 231 & 654 & 183.12\% \\ \hline
		\textbf{Crianças na fila}                      & 479 & 25 & -94.78\% \\ \hline
		\textbf{População de 0 a 3 anos estimada}      & 2014 & 2220 & 10.23\% \\ \hline
		\textbf{Matrículas relativas à população}      & 11.47\% & 29.46\% & 156.85\% \\ \hline
		\textbf{Fila relativa à população}             & 23.78\% & 1.13\% & -95.27\% \\ \hline
		\textbf{Proporção das matrículas no município} & 0.37\% & 0.22\% & -41.01\% \\ \hline
		\textbf{Proporção da demanda no município}     & 0.57\% & 0.06\% & -90.01\% \\ \hline
		\textbf{Proporção da população de 0 a 3 anos}  & 0.32\% & 0.34\% & 7.4\% \\ \hline
		\textbf{Posição no \textit{ranking} das matrículas}     & 73 & 77 & -4 \\ \hline
		\textbf{Posição no \textit{ranking} da demanda}         & 57 & 94 & -37 \\ \hline
	\end{tabular}
	\caption{Comparação entre o primeiro e o último período da amostra}
\end{table}
\begin{table}[H]
	\begin{tabular}{|l|l|l|l|}
		\hline
		\textbf{}                                 & \textbf{dezembro de 2006} & \textbf{dezembro de 2017} & \textbf{Variação} \\ \hline
		\textbf{Crianças na fila}                      & 660 & 25 & -96.21\% \\ \hline
		\textbf{População de 0 a 3 anos estimada}      & 2014 & 2220 & 10.23\% \\ \hline
		\textbf{Fila relativa à população}             & 11.97\% & 1.13\% & -90.59\% \\ \hline
	\end{tabular}
	\caption{Comparação da demanda no mês de dezembro, em 2006 e 2017.}
\end{table}
\section{Cachoeirinha}
\begin{figure}[H]
	\centering
	\includegraphics[width=0.66\linewidth]{../Analises/graficos/tempo/tempo_CACHOEIRINHA}
	\includegraphics[width=0.33\linewidth]{../Analises/mapas/mapas/distrito_CAC}
	\caption{Evolução das matrículas, da demanda, e da população de 0 a 3 anos no distrito Cachoeirinha e a sua localização no município, respectivamente.}
\end{figure}
\begin{table}[H]
	\begin{tabular}{|l|l|l|l|}
		\hline
		\textbf{}                                      & \textbf{junho de 2006}       & \textbf{dezembro de 2017}    & \textbf{Variação} \\ \hline
		\textbf{Número de matrículas}                  & 909 & 4798 & 427.83\% \\ \hline
		\textbf{Crianças na fila}                      & 1167 & 345 & -70.44\% \\ \hline
		\textbf{População de 0 a 3 anos estimada}      & 9522 & 9223 & -3.14\% \\ \hline
		\textbf{Matrículas relativas à população}      & 9.55\% & 52.02\% & 444.94\% \\ \hline
		\textbf{Fila relativa à população}             & 12.26\% & 3.74\% & -69.48\% \\ \hline
		\textbf{Proporção das matrículas no município} & 1.47\% & 1.62\% & 9.98\% \\ \hline
		\textbf{Proporção da demanda no município}     & 1.38\% & 0.78\% & -43.41\% \\ \hline
		\textbf{Proporção da população de 0 a 3 anos}  & 1.52\% & 1.43\% & -5.63\% \\ \hline
		\textbf{Posição no \textit{ranking} das matrículas}     & 23 & 21 & 2 \\ \hline
		\textbf{Posição no \textit{ranking} da demanda}         & 23 & 34 & -11 \\ \hline
	\end{tabular}
	\caption{Comparação entre o primeiro e o último período da amostra}
\end{table}
\begin{table}[H]
	\begin{tabular}{|l|l|l|l|}
		\hline
		\textbf{}                                 & \textbf{dezembro de 2006} & \textbf{dezembro de 2017} & \textbf{Variação} \\ \hline
		\textbf{Crianças na fila}                      & 1228 & 345 & -71.91\% \\ \hline
		\textbf{População de 0 a 3 anos estimada}      & 9522 & 9223 & -3.14\% \\ \hline
		\textbf{Fila relativa à população}             & 7.75\% & 3.74\% & -51.73\% \\ \hline
	\end{tabular}
	\caption{Comparação da demanda no mês de dezembro, em 2006 e 2017.}
\end{table}
\section{Cambuci}
\begin{figure}[H]
	\centering
	\includegraphics[width=0.66\linewidth]{../Analises/graficos/tempo/tempo_CAMBUCI}
	\includegraphics[width=0.33\linewidth]{../Analises/mapas/mapas/distrito_CMB}
	\caption{Evolução das matrículas, da demanda, e da população de 0 a 3 anos no distrito Cambuci e a sua localização no município, respectivamente.}
\end{figure}
\begin{table}[H]
	\begin{tabular}{|l|l|l|l|}
		\hline
		\textbf{}                                      & \textbf{junho de 2006}       & \textbf{dezembro de 2017}    & \textbf{Variação} \\ \hline
		\textbf{Número de matrículas}                  & 131 & 475 & 262.6\% \\ \hline
		\textbf{Crianças na fila}                      & 234 & 62 & -73.5\% \\ \hline
		\textbf{População de 0 a 3 anos estimada}      & 1600 & 2020 & 26.25\% \\ \hline
		\textbf{Matrículas relativas à população}      & 8.19\% & 23.51\% & 187.2\% \\ \hline
		\textbf{Fila relativa à população}             & 14.62\% & 3.07\% & -79.01\% \\ \hline
		\textbf{Proporção das matrículas no município} & 0.21\% & 0.16\% & -24.45\% \\ \hline
		\textbf{Proporção da demanda no município}     & 0.28\% & 0.14\% & -49.28\% \\ \hline
		\textbf{Proporção da população de 0 a 3 anos}  & 0.25\% & 0.31\% & 23.01\% \\ \hline
		\textbf{Posição no \textit{ranking} das matrículas}     & 89 & 84 & 5 \\ \hline
		\textbf{Posição no \textit{ranking} da demanda}         & 78 & 88 & -10 \\ \hline
	\end{tabular}
	\caption{Comparação entre o primeiro e o último período da amostra}
\end{table}
\begin{table}[H]
	\begin{tabular}{|l|l|l|l|}
		\hline
		\textbf{}                                 & \textbf{dezembro de 2006} & \textbf{dezembro de 2017} & \textbf{Variação} \\ \hline
		\textbf{Crianças na fila}                      & 287 & 62 & -78.4\% \\ \hline
		\textbf{População de 0 a 3 anos estimada}      & 1600 & 2020 & 26.25\% \\ \hline
		\textbf{Fila relativa à população}             & 8.25\% & 3.07\% & -62.8\% \\ \hline
	\end{tabular}
	\caption{Comparação da demanda no mês de dezembro, em 2006 e 2017.}
\end{table}
\section{Campo Belo}
\begin{figure}[H]
	\centering
	\includegraphics[width=0.66\linewidth]{../Analises/graficos/tempo/tempo_CAMPO_BELO}
	\includegraphics[width=0.33\linewidth]{../Analises/mapas/mapas/distrito_CBE}
	\caption{Evolução das matrículas, da demanda, e da população de 0 a 3 anos no distrito Campo Belo e a sua localização no município, respectivamente.}
\end{figure}
\begin{table}[H]
	\begin{tabular}{|l|l|l|l|}
		\hline
		\textbf{}                                      & \textbf{junho de 2006}       & \textbf{dezembro de 2017}    & \textbf{Variação} \\ \hline
		\textbf{Número de matrículas}                  & 336 & 1084 & 222.62\% \\ \hline
		\textbf{Crianças na fila}                      & 529 & 166 & -68.62\% \\ \hline
		\textbf{População de 0 a 3 anos estimada}      & 2704 & 2732 & 1.04\% \\ \hline
		\textbf{Matrículas relativas à população}      & 12.43\% & 39.68\% & 219.31\% \\ \hline
		\textbf{Fila relativa à população}             & 19.56\% & 6.08\% & -68.94\% \\ \hline
		\textbf{Proporção das matrículas no município} & 0.54\% & 0.37\% & -32.78\% \\ \hline
		\textbf{Proporção da demanda no município}     & 0.63\% & 0.38\% & -39.93\% \\ \hline
		\textbf{Proporção da população de 0 a 3 anos}  & 0.43\% & 0.42\% & -1.56\% \\ \hline
		\textbf{Posição no \textit{ranking} das matrículas}     & 60 & 68 & -8 \\ \hline
		\textbf{Posição no \textit{ranking} da demanda}         & 52 & 59 & -7 \\ \hline
	\end{tabular}
	\caption{Comparação entre o primeiro e o último período da amostra}
\end{table}
\begin{table}[H]
	\begin{tabular}{|l|l|l|l|}
		\hline
		\textbf{}                                 & \textbf{dezembro de 2006} & \textbf{dezembro de 2017} & \textbf{Variação} \\ \hline
		\textbf{Crianças na fila}                      & 630 & 166 & -73.65\% \\ \hline
		\textbf{População de 0 a 3 anos estimada}      & 2704 & 2732 & 1.04\% \\ \hline
		\textbf{Fila relativa à população}             & 15.38\% & 6.08\% & -60.49\% \\ \hline
	\end{tabular}
	\caption{Comparação da demanda no mês de dezembro, em 2006 e 2017.}
\end{table}
\section{Campo Grande}
\begin{figure}[H]
	\centering
	\includegraphics[width=0.66\linewidth]{../Analises/graficos/tempo/tempo_CAMPO_GRANDE}
	\includegraphics[width=0.33\linewidth]{../Analises/mapas/mapas/distrito_CGR}
	\caption{Evolução das matrículas, da demanda, e da população de 0 a 3 anos no distrito Campo Grande e a sua localização no município, respectivamente.}
\end{figure}
\begin{table}[H]
	\begin{tabular}{|l|l|l|l|}
		\hline
		\textbf{}                                      & \textbf{junho de 2006}       & \textbf{dezembro de 2017}    & \textbf{Variação} \\ \hline
		\textbf{Número de matrículas}                  & 218 & 1441 & 561.01\% \\ \hline
		\textbf{Crianças na fila}                      & 431 & 321 & -25.52\% \\ \hline
		\textbf{População de 0 a 3 anos estimada}      & 4756 & 4681 & -1.58\% \\ \hline
		\textbf{Matrículas relativas à população}      & 4.58\% & 30.78\% & 571.6\% \\ \hline
		\textbf{Fila relativa à população}             & 9.06\% & 6.86\% & -24.33\% \\ \hline
		\textbf{Proporção das matrículas no município} & 0.35\% & 0.49\% & 37.73\% \\ \hline
		\textbf{Proporção da demanda no município}     & 0.51\% & 0.73\% & 42.58\% \\ \hline
		\textbf{Proporção da população de 0 a 3 anos}  & 0.76\% & 0.73\% & -4.11\% \\ \hline
		\textbf{Posição no \textit{ranking} das matrículas}     & 76 & 62 & 14 \\ \hline
		\textbf{Posição no \textit{ranking} da demanda}         & 63 & 35 & 28 \\ \hline
	\end{tabular}
	\caption{Comparação entre o primeiro e o último período da amostra}
\end{table}
\begin{table}[H]
	\begin{tabular}{|l|l|l|l|}
		\hline
		\textbf{}                                 & \textbf{dezembro de 2006} & \textbf{dezembro de 2017} & \textbf{Variação} \\ \hline
		\textbf{Crianças na fila}                      & 543 & 321 & -40.88\% \\ \hline
		\textbf{População de 0 a 3 anos estimada}      & 4756 & 4681 & -1.58\% \\ \hline
		\textbf{Fila relativa à população}             & 6.22\% & 6.86\% & 10.25\% \\ \hline
	\end{tabular}
	\caption{Comparação da demanda no mês de dezembro, em 2006 e 2017.}
\end{table}
\section{Campo Limpo}
\begin{figure}[H]
	\centering
	\includegraphics[width=0.66\linewidth]{../Analises/graficos/tempo/tempo_CAMPO_LIMPO}
	\includegraphics[width=0.33\linewidth]{../Analises/mapas/mapas/distrito_CLM}
	\caption{Evolução das matrículas, da demanda, e da população de 0 a 3 anos no distrito Campo Limpo e a sua localização no município, respectivamente.}
\end{figure}
\begin{table}[H]
	\begin{tabular}{|l|l|l|l|}
		\hline
		\textbf{}                                      & \textbf{junho de 2006}       & \textbf{dezembro de 2017}    & \textbf{Variação} \\ \hline
		\textbf{Número de matrículas}                  & 1286 & 6963 & 441.45\% \\ \hline
		\textbf{Crianças na fila}                      & 3051 & 1750 & -42.64\% \\ \hline
		\textbf{População de 0 a 3 anos estimada}      & 13652 & 13223 & -3.14\% \\ \hline
		\textbf{Matrículas relativas à população}      & 9.42\% & 52.66\% & 459.01\% \\ \hline
		\textbf{Fila relativa à população}             & 22.35\% & 13.23\% & -40.78\% \\ \hline
		\textbf{Proporção das matrículas no município} & 2.08\% & 2.35\% & 12.82\% \\ \hline
		\textbf{Proporção da demanda no município}     & 3.61\% & 3.97\% & 9.8\% \\ \hline
		\textbf{Proporção da população de 0 a 3 anos}  & 2.18\% & 2.05\% & -5.63\% \\ \hline
		\textbf{Posição no \textit{ranking} das matrículas}     & 11 & 12 & -1 \\ \hline
		\textbf{Posição no \textit{ranking} da demanda}         & 5 & 8 & -3 \\ \hline
	\end{tabular}
	\caption{Comparação entre o primeiro e o último período da amostra}
\end{table}
\begin{table}[H]
	\begin{tabular}{|l|l|l|l|}
		\hline
		\textbf{}                                 & \textbf{dezembro de 2006} & \textbf{dezembro de 2017} & \textbf{Variação} \\ \hline
		\textbf{Crianças na fila}                      & 5254 & 1750 & -66.69\% \\ \hline
		\textbf{População de 0 a 3 anos estimada}      & 13652 & 13223 & -3.14\% \\ \hline
		\textbf{Fila relativa à população}             & 9.21\% & 13.23\% & 43.7\% \\ \hline
	\end{tabular}
	\caption{Comparação da demanda no mês de dezembro, em 2006 e 2017.}
\end{table}
\section{Cangaiba}
\begin{figure}[H]
	\centering
	\includegraphics[width=0.66\linewidth]{../Analises/graficos/tempo/tempo_CANGAIBA}
	\includegraphics[width=0.33\linewidth]{../Analises/mapas/mapas/distrito_CNG}
	\caption{Evolução das matrículas, da demanda, e da população de 0 a 3 anos no distrito Cangaiba e a sua localização no município, respectivamente.}
\end{figure}
\begin{table}[H]
	\begin{tabular}{|l|l|l|l|}
		\hline
		\textbf{}                                      & \textbf{junho de 2006}       & \textbf{dezembro de 2017}    & \textbf{Variação} \\ \hline
		\textbf{Número de matrículas}                  & 750 & 4145 & 452.67\% \\ \hline
		\textbf{Crianças na fila}                      & 1225 & 230 & -81.22\% \\ \hline
		\textbf{População de 0 a 3 anos estimada}      & 7773 & 7807 & 0.44\% \\ \hline
		\textbf{Matrículas relativas à população}      & 9.65\% & 53.09\% & 450.26\% \\ \hline
		\textbf{Fila relativa à população}             & 15.76\% & 2.95\% & -81.31\% \\ \hline
		\textbf{Proporção das matrículas no município} & 1.21\% & 1.4\% & 15.15\% \\ \hline
		\textbf{Proporção da demanda no município}     & 1.45\% & 0.52\% & -64.06\% \\ \hline
		\textbf{Proporção da população de 0 a 3 anos}  & 1.24\% & 1.21\% & -2.14\% \\ \hline
		\textbf{Posição no \textit{ranking} das matrículas}     & 32 & 25 & 7 \\ \hline
		\textbf{Posição no \textit{ranking} da demanda}         & 20 & 44 & -24 \\ \hline
	\end{tabular}
	\caption{Comparação entre o primeiro e o último período da amostra}
\end{table}
\begin{table}[H]
	\begin{tabular}{|l|l|l|l|}
		\hline
		\textbf{}                                 & \textbf{dezembro de 2006} & \textbf{dezembro de 2017} & \textbf{Variação} \\ \hline
		\textbf{Crianças na fila}                      & 2103 & 230 & -89.06\% \\ \hline
		\textbf{População de 0 a 3 anos estimada}      & 7773 & 7807 & 0.44\% \\ \hline
		\textbf{Fila relativa à população}             & 9.88\% & 2.95\% & -70.18\% \\ \hline
	\end{tabular}
	\caption{Comparação da demanda no mês de dezembro, em 2006 e 2017.}
\end{table}
\section{Capão Redondo}
\begin{figure}[H]
	\centering
	\includegraphics[width=0.66\linewidth]{../Analises/graficos/tempo/tempo_CAPAO_REDONDO}
	\includegraphics[width=0.33\linewidth]{../Analises/mapas/mapas/distrito_CRE}
	\caption{Evolução das matrículas, da demanda, e da população de 0 a 3 anos no distrito Capão Redondo e a sua localização no município, respectivamente.}
\end{figure}
\begin{table}[H]
	\begin{tabular}{|l|l|l|l|}
		\hline
		\textbf{}                                      & \textbf{junho de 2006}       & \textbf{dezembro de 2017}    & \textbf{Variação} \\ \hline
		\textbf{Número de matrículas}                  & 902 & 8456 & 837.47\% \\ \hline
		\textbf{Crianças na fila}                      & 2345 & 2776 & 18.38\% \\ \hline
		\textbf{População de 0 a 3 anos estimada}      & 17395 & 18271 & 5.04\% \\ \hline
		\textbf{Matrículas relativas à população}      & 5.19\% & 46.28\% & 792.53\% \\ \hline
		\textbf{Fila relativa à população}             & 13.48\% & 15.19\% & 12.7\% \\ \hline
		\textbf{Proporção das matrículas no município} & 1.46\% & 2.85\% & 95.33\% \\ \hline
		\textbf{Proporção da demanda no município}     & 2.78\% & 6.3\% & 126.62\% \\ \hline
		\textbf{Proporção da população de 0 a 3 anos}  & 2.77\% & 2.84\% & 2.34\% \\ \hline
		\textbf{Posição no \textit{ranking} das matrículas}     & 24 & 8 & 16 \\ \hline
		\textbf{Posição no \textit{ranking} da demanda}         & 9 & 1 & 8 \\ \hline
	\end{tabular}
	\caption{Comparação entre o primeiro e o último período da amostra}
\end{table}
\begin{table}[H]
	\begin{tabular}{|l|l|l|l|}
		\hline
		\textbf{}                                 & \textbf{dezembro de 2006} & \textbf{dezembro de 2017} & \textbf{Variação} \\ \hline
		\textbf{Crianças na fila}                      & 4117 & 2776 & -32.57\% \\ \hline
		\textbf{População de 0 a 3 anos estimada}      & 17395 & 18271 & 5.04\% \\ \hline
		\textbf{Fila relativa à população}             & 5.54\% & 15.19\% & 174.25\% \\ \hline
	\end{tabular}
	\caption{Comparação da demanda no mês de dezembro, em 2006 e 2017.}
\end{table}
\section{Carrão}
\begin{figure}[H]
	\centering
	\includegraphics[width=0.66\linewidth]{../Analises/graficos/tempo/tempo_CARRAO}
	\includegraphics[width=0.33\linewidth]{../Analises/mapas/mapas/distrito_CAR}
	\caption{Evolução das matrículas, da demanda, e da população de 0 a 3 anos no distrito Carrão e a sua localização no município, respectivamente.}
\end{figure}
\begin{table}[H]
	\begin{tabular}{|l|l|l|l|}
		\hline
		\textbf{}                                      & \textbf{junho de 2006}       & \textbf{dezembro de 2017}    & \textbf{Variação} \\ \hline
		\textbf{Número de matrículas}                  & 506 & 1757 & 247.23\% \\ \hline
		\textbf{Crianças na fila}                      & 470 & 118 & -74.89\% \\ \hline
		\textbf{População de 0 a 3 anos estimada}      & 3487 & 3525 & 1.09\% \\ \hline
		\textbf{Matrículas relativas à população}      & 14.51\% & 49.84\% & 243.49\% \\ \hline
		\textbf{Fila relativa à população}             & 13.48\% & 3.35\% & -75.16\% \\ \hline
		\textbf{Proporção das matrículas no município} & 0.82\% & 0.59\% & -27.65\% \\ \hline
		\textbf{Proporção da demanda no município}     & 0.56\% & 0.27\% & -51.94\% \\ \hline
		\textbf{Proporção da população de 0 a 3 anos}  & 0.56\% & 0.55\% & -1.51\% \\ \hline
		\textbf{Posição no \textit{ranking} das matrículas}     & 44 & 51 & -7 \\ \hline
		\textbf{Posição no \textit{ranking} da demanda}         & 59 & 75 & -16 \\ \hline
	\end{tabular}
	\caption{Comparação entre o primeiro e o último período da amostra}
\end{table}
\begin{table}[H]
	\begin{tabular}{|l|l|l|l|}
		\hline
		\textbf{}                                 & \textbf{dezembro de 2006} & \textbf{dezembro de 2017} & \textbf{Variação} \\ \hline
		\textbf{Crianças na fila}                      & 817 & 118 & -85.56\% \\ \hline
		\textbf{População de 0 a 3 anos estimada}      & 3487 & 3525 & 1.09\% \\ \hline
		\textbf{Fila relativa à população}             & 15.54\% & 3.35\% & -78.46\% \\ \hline
	\end{tabular}
	\caption{Comparação da demanda no mês de dezembro, em 2006 e 2017.}
\end{table}
\section{Casa Verde}
\begin{figure}[H]
	\centering
	\includegraphics[width=0.66\linewidth]{../Analises/graficos/tempo/tempo_CASA_VERDE}
	\includegraphics[width=0.33\linewidth]{../Analises/mapas/mapas/distrito_CVE}
	\caption{Evolução das matrículas, da demanda, e da população de 0 a 3 anos no distrito Casa Verde e a sua localização no município, respectivamente.}
\end{figure}
\begin{table}[H]
	\begin{tabular}{|l|l|l|l|}
		\hline
		\textbf{}                                      & \textbf{junho de 2006}       & \textbf{dezembro de 2017}    & \textbf{Variação} \\ \hline
		\textbf{Número de matrículas}                  & 403 & 1523 & 277.92\% \\ \hline
		\textbf{Crianças na fila}                      & 667 & 191 & -71.36\% \\ \hline
		\textbf{População de 0 a 3 anos estimada}      & 4101 & 4551 & 10.97\% \\ \hline
		\textbf{Matrículas relativas à população}      & 9.83\% & 33.47\% & 240.55\% \\ \hline
		\textbf{Fila relativa à população}             & 16.26\% & 4.2\% & -74.2\% \\ \hline
		\textbf{Proporção das matrículas no município} & 0.65\% & 0.51\% & -21.26\% \\ \hline
		\textbf{Proporção da demanda no município}     & 0.79\% & 0.43\% & -45.18\% \\ \hline
		\textbf{Proporção da população de 0 a 3 anos}  & 0.65\% & 0.71\% & 8.12\% \\ \hline
		\textbf{Posição no \textit{ranking} das matrículas}     & 56 & 59 & -3 \\ \hline
		\textbf{Posição no \textit{ranking} da demanda}         & 42 & 52 & -10 \\ \hline
	\end{tabular}
	\caption{Comparação entre o primeiro e o último período da amostra}
\end{table}
\begin{table}[H]
	\begin{tabular}{|l|l|l|l|}
		\hline
		\textbf{}                                 & \textbf{dezembro de 2006} & \textbf{dezembro de 2017} & \textbf{Variação} \\ \hline
		\textbf{Crianças na fila}                      & 736 & 191 & -74.05\% \\ \hline
		\textbf{População de 0 a 3 anos estimada}      & 4101 & 4551 & 10.97\% \\ \hline
		\textbf{Fila relativa à população}             & 9.63\% & 4.2\% & -56.42\% \\ \hline
	\end{tabular}
	\caption{Comparação da demanda no mês de dezembro, em 2006 e 2017.}
\end{table}
\section{Cidade Ademar}
\begin{figure}[H]
	\centering
	\includegraphics[width=0.66\linewidth]{../Analises/graficos/tempo/tempo_CIDADE_ADEMAR}
	\includegraphics[width=0.33\linewidth]{../Analises/mapas/mapas/distrito_CAD}
	\caption{Evolução das matrículas, da demanda, e da população de 0 a 3 anos no distrito Cidade Ademar e a sua localização no município, respectivamente.}
\end{figure}
\begin{table}[H]
	\begin{tabular}{|l|l|l|l|}
		\hline
		\textbf{}                                      & \textbf{junho de 2006}       & \textbf{dezembro de 2017}    & \textbf{Variação} \\ \hline
		\textbf{Número de matrículas}                  & 981 & 6211 & 533.13\% \\ \hline
		\textbf{Crianças na fila}                      & 3214 & 2009 & -37.49\% \\ \hline
		\textbf{População de 0 a 3 anos estimada}      & 16368 & 17354 & 6.02\% \\ \hline
		\textbf{Matrículas relativas à população}      & 5.99\% & 35.79\% & 497.16\% \\ \hline
		\textbf{Fila relativa à população}             & 19.64\% & 11.58\% & -41.04\% \\ \hline
		\textbf{Proporção das matrículas no município} & 1.59\% & 2.1\% & 31.92\% \\ \hline
		\textbf{Proporção da demanda no município}     & 3.81\% & 4.56\% & 19.66\% \\ \hline
		\textbf{Proporção da população de 0 a 3 anos}  & 2.61\% & 2.69\% & 3.3\% \\ \hline
		\textbf{Posição no \textit{ranking} das matrículas}     & 18 & 15 & 3 \\ \hline
		\textbf{Posição no \textit{ranking} da demanda}         & 4 & 5 & -1 \\ \hline
	\end{tabular}
	\caption{Comparação entre o primeiro e o último período da amostra}
\end{table}
\begin{table}[H]
	\begin{tabular}{|l|l|l|l|}
		\hline
		\textbf{}                                 & \textbf{dezembro de 2006} & \textbf{dezembro de 2017} & \textbf{Variação} \\ \hline
		\textbf{Crianças na fila}                      & 4183 & 2009 & -51.97\% \\ \hline
		\textbf{População de 0 a 3 anos estimada}      & 16368 & 17354 & 6.02\% \\ \hline
		\textbf{Fila relativa à população}             & 6.68\% & 11.58\% & 73.3\% \\ \hline
	\end{tabular}
	\caption{Comparação da demanda no mês de dezembro, em 2006 e 2017.}
\end{table}
\section{Cidade Dutra}
\begin{figure}[H]
	\centering
	\includegraphics[width=0.66\linewidth]{../Analises/graficos/tempo/tempo_CIDADE_DUTRA}
	\includegraphics[width=0.33\linewidth]{../Analises/mapas/mapas/distrito_CDU}
	\caption{Evolução das matrículas, da demanda, e da população de 0 a 3 anos no distrito Cidade Dutra e a sua localização no município, respectivamente.}
\end{figure}
\begin{table}[H]
	\begin{tabular}{|l|l|l|l|}
		\hline
		\textbf{}                                      & \textbf{junho de 2006}       & \textbf{dezembro de 2017}    & \textbf{Variação} \\ \hline
		\textbf{Número de matrículas}                  & 2421 & 6763 & 179.35\% \\ \hline
		\textbf{Crianças na fila}                      & 1934 & 865 & -55.27\% \\ \hline
		\textbf{População de 0 a 3 anos estimada}      & 11789 & 11462 & -2.77\% \\ \hline
		\textbf{Matrículas relativas à população}      & 20.54\% & 59.0\% & 187.32\% \\ \hline
		\textbf{Fila relativa à população}             & 16.41\% & 7.55\% & -54.0\% \\ \hline
		\textbf{Proporção das matrículas no município} & 3.92\% & 2.28\% & -41.79\% \\ \hline
		\textbf{Proporção da demanda no município}     & 2.29\% & 1.96\% & -14.38\% \\ \hline
		\textbf{Proporção da população de 0 a 3 anos}  & 1.88\% & 1.78\% & -5.27\% \\ \hline
		\textbf{Posição no \textit{ranking} das matrículas}     & 2 & 14 & -12 \\ \hline
		\textbf{Posição no \textit{ranking} da demanda}         & 13 & 14 & -1 \\ \hline
	\end{tabular}
	\caption{Comparação entre o primeiro e o último período da amostra}
\end{table}
\begin{table}[H]
	\begin{tabular}{|l|l|l|l|}
		\hline
		\textbf{}                                 & \textbf{dezembro de 2006} & \textbf{dezembro de 2017} & \textbf{Variação} \\ \hline
		\textbf{Crianças na fila}                      & 2045 & 865 & -57.7\% \\ \hline
		\textbf{População de 0 a 3 anos estimada}      & 11789 & 11462 & -2.77\% \\ \hline
		\textbf{Fila relativa à população}             & 25.2\% & 7.55\% & -70.05\% \\ \hline
	\end{tabular}
	\caption{Comparação da demanda no mês de dezembro, em 2006 e 2017.}
\end{table}
\section{Cidade Lider}
\begin{figure}[H]
	\centering
	\includegraphics[width=0.66\linewidth]{../Analises/graficos/tempo/tempo_CIDADE_LIDER}
	\includegraphics[width=0.33\linewidth]{../Analises/mapas/mapas/distrito_CLD}
	\caption{Evolução das matrículas, da demanda, e da população de 0 a 3 anos no distrito Cidade Lider e a sua localização no município, respectivamente.}
\end{figure}
\begin{table}[H]
	\begin{tabular}{|l|l|l|l|}
		\hline
		\textbf{}                                      & \textbf{junho de 2006}       & \textbf{dezembro de 2017}    & \textbf{Variação} \\ \hline
		\textbf{Número de matrículas}                  & 822 & 3795 & 361.68\% \\ \hline
		\textbf{Crianças na fila}                      & 1356 & 505 & -62.76\% \\ \hline
		\textbf{População de 0 a 3 anos estimada}      & 7379 & 7408 & 0.39\% \\ \hline
		\textbf{Matrículas relativas à população}      & 11.14\% & 51.23\% & 359.87\% \\ \hline
		\textbf{Fila relativa à população}             & 18.38\% & 6.82\% & -62.9\% \\ \hline
		\textbf{Proporção das matrículas no município} & 1.33\% & 1.28\% & -3.8\% \\ \hline
		\textbf{Proporção da demanda no município}     & 1.61\% & 1.15\% & -28.71\% \\ \hline
		\textbf{Proporção da população de 0 a 3 anos}  & 1.18\% & 1.15\% & -2.19\% \\ \hline
		\textbf{Posição no \textit{ranking} das matrículas}     & 28 & 28 & 0 \\ \hline
		\textbf{Posição no \textit{ranking} da demanda}         & 18 & 24 & -6 \\ \hline
	\end{tabular}
	\caption{Comparação entre o primeiro e o último período da amostra}
\end{table}
\begin{table}[H]
	\begin{tabular}{|l|l|l|l|}
		\hline
		\textbf{}                                 & \textbf{dezembro de 2006} & \textbf{dezembro de 2017} & \textbf{Variação} \\ \hline
		\textbf{Crianças na fila}                      & 1821 & 505 & -72.27\% \\ \hline
		\textbf{População de 0 a 3 anos estimada}      & 7379 & 7408 & 0.39\% \\ \hline
		\textbf{Fila relativa à população}             & 12.35\% & 6.82\% & -44.8\% \\ \hline
	\end{tabular}
	\caption{Comparação da demanda no mês de dezembro, em 2006 e 2017.}
\end{table}
\section{Cidade Tiradentes}
\begin{figure}[H]
	\centering
	\includegraphics[width=0.66\linewidth]{../Analises/graficos/tempo/tempo_CIDADE_TIRADENTES}
	\includegraphics[width=0.33\linewidth]{../Analises/mapas/mapas/distrito_CTI}
	\caption{Evolução das matrículas, da demanda, e da população de 0 a 3 anos no distrito Cidade Tiradentes e a sua localização no município, respectivamente.}
\end{figure}
\begin{table}[H]
	\begin{tabular}{|l|l|l|l|}
		\hline
		\textbf{}                                      & \textbf{junho de 2006}       & \textbf{dezembro de 2017}    & \textbf{Variação} \\ \hline
		\textbf{Número de matrículas}                  & 1676 & 11423 & 581.56\% \\ \hline
		\textbf{Crianças na fila}                      & 1559 & 178 & -88.58\% \\ \hline
		\textbf{População de 0 a 3 anos estimada}      & 15277 & 14531 & -4.88\% \\ \hline
		\textbf{Matrículas relativas à população}      & 10.97\% & 78.61\% & 616.55\% \\ \hline
		\textbf{Fila relativa à população}             & 10.2\% & 1.22\% & -88.0\% \\ \hline
		\textbf{Proporção das matrículas no município} & 2.72\% & 3.86\% & 42.01\% \\ \hline
		\textbf{Proporção da demanda no município}     & 1.85\% & 0.4\% & -78.14\% \\ \hline
		\textbf{Proporção da população de 0 a 3 anos}  & 2.43\% & 2.26\% & -7.33\% \\ \hline
		\textbf{Posição no \textit{ranking} das matrículas}     & 8 & 3 & 5 \\ \hline
		\textbf{Posição no \textit{ranking} da demanda}         & 17 & 54 & -37 \\ \hline
	\end{tabular}
	\caption{Comparação entre o primeiro e o último período da amostra}
\end{table}
\begin{table}[H]
	\begin{tabular}{|l|l|l|l|}
		\hline
		\textbf{}                                 & \textbf{dezembro de 2006} & \textbf{dezembro de 2017} & \textbf{Variação} \\ \hline
		\textbf{Crianças na fila}                      & 3309 & 178 & -94.62\% \\ \hline
		\textbf{População de 0 a 3 anos estimada}      & 15277 & 14531 & -4.88\% \\ \hline
		\textbf{Fila relativa à população}             & 12.04\% & 1.22\% & -89.83\% \\ \hline
	\end{tabular}
	\caption{Comparação da demanda no mês de dezembro, em 2006 e 2017.}
\end{table}
\section{Consolação}
\begin{figure}[H]
	\centering
	\includegraphics[width=0.66\linewidth]{../Analises/graficos/tempo/tempo_CONSOLACAO}
	\includegraphics[width=0.33\linewidth]{../Analises/mapas/mapas/distrito_CON}
	\caption{Evolução das matrículas, da demanda, e da população de 0 a 3 anos no distrito Consolação e a sua localização no município, respectivamente.}
\end{figure}
\begin{table}[H]
	\begin{tabular}{|l|l|l|l|}
		\hline
		\textbf{}                                      & \textbf{junho de 2006}       & \textbf{dezembro de 2017}    & \textbf{Variação} \\ \hline
		\textbf{Número de matrículas}                  & 135 & 210 & 55.56\% \\ \hline
		\textbf{Crianças na fila}                      & 122 & 75 & -38.52\% \\ \hline
		\textbf{População de 0 a 3 anos estimada}      & 1422 & 1875 & 31.86\% \\ \hline
		\textbf{Matrículas relativas à população}      & 9.49\% & 11.2\% & 17.97\% \\ \hline
		\textbf{Fila relativa à população}             & 8.58\% & 4.0\% & -53.38\% \\ \hline
		\textbf{Proporção das matrículas no município} & 0.22\% & 0.07\% & -67.59\% \\ \hline
		\textbf{Proporção da demanda no município}     & 0.14\% & 0.17\% & 17.69\% \\ \hline
		\textbf{Proporção da população de 0 a 3 anos}  & 0.23\% & 0.29\% & 28.47\% \\ \hline
		\textbf{Posição no \textit{ranking} das matrículas}     & 88 & 93 & -5 \\ \hline
		\textbf{Posição no \textit{ranking} da demanda}         & 92 & 87 & 5 \\ \hline
	\end{tabular}
	\caption{Comparação entre o primeiro e o último período da amostra}
\end{table}
\begin{table}[H]
	\begin{tabular}{|l|l|l|l|}
		\hline
		\textbf{}                                 & \textbf{dezembro de 2006} & \textbf{dezembro de 2017} & \textbf{Variação} \\ \hline
		\textbf{Crianças na fila}                      & 382 & 75 & -80.37\% \\ \hline
		\textbf{População de 0 a 3 anos estimada}      & 1422 & 1875 & 31.86\% \\ \hline
		\textbf{Fila relativa à população}             & 19.13\% & 4.0\% & -79.09\% \\ \hline
	\end{tabular}
	\caption{Comparação da demanda no mês de dezembro, em 2006 e 2017.}
\end{table}
\section{Cursino}
\begin{figure}[H]
	\centering
	\includegraphics[width=0.66\linewidth]{../Analises/graficos/tempo/tempo_CURSINO}
	\includegraphics[width=0.33\linewidth]{../Analises/mapas/mapas/distrito_CUR}
	\caption{Evolução das matrículas, da demanda, e da população de 0 a 3 anos no distrito Cursino e a sua localização no município, respectivamente.}
\end{figure}
\begin{table}[H]
	\begin{tabular}{|l|l|l|l|}
		\hline
		\textbf{}                                      & \textbf{junho de 2006}       & \textbf{dezembro de 2017}    & \textbf{Variação} \\ \hline
		\textbf{Número de matrículas}                  & 252 & 1359 & 439.29\% \\ \hline
		\textbf{Crianças na fila}                      & 571 & 126 & -77.93\% \\ \hline
		\textbf{População de 0 a 3 anos estimada}      & 4940 & 5304 & 7.37\% \\ \hline
		\textbf{Matrículas relativas à população}      & 5.1\% & 25.62\% & 402.28\% \\ \hline
		\textbf{Fila relativa à população}             & 11.56\% & 2.38\% & -79.45\% \\ \hline
		\textbf{Proporção das matrículas no município} & 0.41\% & 0.46\% & 12.37\% \\ \hline
		\textbf{Proporção da demanda no município}     & 0.68\% & 0.29\% & -57.76\% \\ \hline
		\textbf{Proporção da população de 0 a 3 anos}  & 0.79\% & 0.82\% & 4.61\% \\ \hline
		\textbf{Posição no \textit{ranking} das matrículas}     & 70 & 63 & 7 \\ \hline
		\textbf{Posição no \textit{ranking} da demanda}         & 50 & 71 & -21 \\ \hline
	\end{tabular}
	\caption{Comparação entre o primeiro e o último período da amostra}
\end{table}
\begin{table}[H]
	\begin{tabular}{|l|l|l|l|}
		\hline
		\textbf{}                                 & \textbf{dezembro de 2006} & \textbf{dezembro de 2017} & \textbf{Variação} \\ \hline
		\textbf{Crianças na fila}                      & 700 & 126 & -82.0\% \\ \hline
		\textbf{População de 0 a 3 anos estimada}      & 4940 & 5304 & 7.37\% \\ \hline
		\textbf{Fila relativa à população}             & 5.16\% & 2.38\% & -53.96\% \\ \hline
	\end{tabular}
	\caption{Comparação da demanda no mês de dezembro, em 2006 e 2017.}
\end{table}
\section{Ermelino Matarazzo}
\begin{figure}[H]
	\centering
	\includegraphics[width=0.66\linewidth]{../Analises/graficos/tempo/tempo_ERMELINO_MATARAZZO}
	\includegraphics[width=0.33\linewidth]{../Analises/mapas/mapas/distrito_ERM}
	\caption{Evolução das matrículas, da demanda, e da população de 0 a 3 anos no distrito Ermelino Matarazzo e a sua localização no município, respectivamente.}
\end{figure}
\begin{table}[H]
	\begin{tabular}{|l|l|l|l|}
		\hline
		\textbf{}                                      & \textbf{junho de 2006}       & \textbf{dezembro de 2017}    & \textbf{Variação} \\ \hline
		\textbf{Número de matrículas}                  & 422 & 3631 & 760.43\% \\ \hline
		\textbf{Crianças na fila}                      & 901 & 166 & -81.58\% \\ \hline
		\textbf{População de 0 a 3 anos estimada}      & 6993 & 6717 & -3.95\% \\ \hline
		\textbf{Matrículas relativas à população}      & 6.03\% & 54.06\% & 795.78\% \\ \hline
		\textbf{Fila relativa à população}             & 12.88\% & 2.47\% & -80.82\% \\ \hline
		\textbf{Proporção das matrículas no município} & 0.68\% & 1.23\% & 79.28\% \\ \hline
		\textbf{Proporção da demanda no município}     & 1.07\% & 0.38\% & -64.73\% \\ \hline
		\textbf{Proporção da população de 0 a 3 anos}  & 1.11\% & 1.04\% & -6.41\% \\ \hline
		\textbf{Posição no \textit{ranking} das matrículas}     & 55 & 30 & 25 \\ \hline
		\textbf{Posição no \textit{ranking} da demanda}         & 31 & 60 & -29 \\ \hline
	\end{tabular}
	\caption{Comparação entre o primeiro e o último período da amostra}
\end{table}
\begin{table}[H]
	\begin{tabular}{|l|l|l|l|}
		\hline
		\textbf{}                                 & \textbf{dezembro de 2006} & \textbf{dezembro de 2017} & \textbf{Variação} \\ \hline
		\textbf{Crianças na fila}                      & 1578 & 166 & -89.48\% \\ \hline
		\textbf{População de 0 a 3 anos estimada}      & 6993 & 6717 & -3.95\% \\ \hline
		\textbf{Fila relativa à população}             & 6.88\% & 2.47\% & -64.08\% \\ \hline
	\end{tabular}
	\caption{Comparação da demanda no mês de dezembro, em 2006 e 2017.}
\end{table}
\section{Freguesia do Ó}
\begin{figure}[H]
	\centering
	\includegraphics[width=0.66\linewidth]{../Analises/graficos/tempo/tempo_FREGUESIA_DO_O}
	\includegraphics[width=0.33\linewidth]{../Analises/mapas/mapas/distrito_FRE}
	\caption{Evolução das matrículas, da demanda, e da população de 0 a 3 anos no distrito Freguesia do Ó e a sua localização no município, respectivamente.}
\end{figure}
\begin{table}[H]
	\begin{tabular}{|l|l|l|l|}
		\hline
		\textbf{}                                      & \textbf{junho de 2006}       & \textbf{dezembro de 2017}    & \textbf{Variação} \\ \hline
		\textbf{Número de matrículas}                  & 673 & 3206 & 376.37\% \\ \hline
		\textbf{Crianças na fila}                      & 599 & 245 & -59.1\% \\ \hline
		\textbf{População de 0 a 3 anos estimada}      & 7168 & 7248 & 1.12\% \\ \hline
		\textbf{Matrículas relativas à população}      & 9.39\% & 44.23\% & 371.12\% \\ \hline
		\textbf{Fila relativa à população}             & 8.36\% & 3.38\% & -59.55\% \\ \hline
		\textbf{Proporção das matrículas no município} & 1.09\% & 1.08\% & -0.74\% \\ \hline
		\textbf{Proporção da demanda no município}     & 0.71\% & 0.56\% & -21.7\% \\ \hline
		\textbf{Proporção da população de 0 a 3 anos}  & 1.14\% & 1.13\% & -1.48\% \\ \hline
		\textbf{Posição no \textit{ranking} das matrículas}     & 36 & 35 & 1 \\ \hline
		\textbf{Posição no \textit{ranking} da demanda}         & 47 & 43 & 4 \\ \hline
	\end{tabular}
	\caption{Comparação entre o primeiro e o último período da amostra}
\end{table}
\begin{table}[H]
	\begin{tabular}{|l|l|l|l|}
		\hline
		\textbf{}                                 & \textbf{dezembro de 2006} & \textbf{dezembro de 2017} & \textbf{Variação} \\ \hline
		\textbf{Crianças na fila}                      & 843 & 245 & -70.94\% \\ \hline
		\textbf{População de 0 a 3 anos estimada}      & 7168 & 7248 & 1.12\% \\ \hline
		\textbf{Fila relativa à população}             & 10.16\% & 3.38\% & -66.73\% \\ \hline
	\end{tabular}
	\caption{Comparação da demanda no mês de dezembro, em 2006 e 2017.}
\end{table}
\section{Grajaú}
\begin{figure}[H]
	\centering
	\includegraphics[width=0.66\linewidth]{../Analises/graficos/tempo/tempo_GRAJAU}
	\includegraphics[width=0.33\linewidth]{../Analises/mapas/mapas/distrito_GRA}
	\caption{Evolução das matrículas, da demanda, e da população de 0 a 3 anos no distrito Grajaú e a sua localização no município, respectivamente.}
\end{figure}
\begin{table}[H]
	\begin{tabular}{|l|l|l|l|}
		\hline
		\textbf{}                                      & \textbf{junho de 2006}       & \textbf{dezembro de 2017}    & \textbf{Variação} \\ \hline
		\textbf{Número de matrículas}                  & 2555 & 12486 & 388.69\% \\ \hline
		\textbf{Crianças na fila}                      & 3914 & 2269 & -42.03\% \\ \hline
		\textbf{População de 0 a 3 anos estimada}      & 25304 & 25332 & 0.11\% \\ \hline
		\textbf{Matrículas relativas à população}      & 10.1\% & 49.29\% & 388.15\% \\ \hline
		\textbf{Fila relativa à população}             & 15.47\% & 8.96\% & -42.09\% \\ \hline
		\textbf{Proporção das matrículas no município} & 4.14\% & 4.21\% & 1.82\% \\ \hline
		\textbf{Proporção da demanda no município}     & 4.64\% & 5.15\% & 10.98\% \\ \hline
		\textbf{Proporção da população de 0 a 3 anos}  & 4.03\% & 3.93\% & -2.46\% \\ \hline
		\textbf{Posição no \textit{ranking} das matrículas}     & 1 & 1 & 0 \\ \hline
		\textbf{Posição no \textit{ranking} da demanda}         & 1 & 4 & -3 \\ \hline
	\end{tabular}
	\caption{Comparação entre o primeiro e o último período da amostra}
\end{table}
\begin{table}[H]
	\begin{tabular}{|l|l|l|l|}
		\hline
		\textbf{}                                 & \textbf{dezembro de 2006} & \textbf{dezembro de 2017} & \textbf{Variação} \\ \hline
		\textbf{Crianças na fila}                      & 4385 & 2269 & -48.26\% \\ \hline
		\textbf{População de 0 a 3 anos estimada}      & 25304 & 25332 & 0.11\% \\ \hline
		\textbf{Fila relativa à população}             & 9.93\% & 8.96\% & -9.8\% \\ \hline
	\end{tabular}
	\caption{Comparação da demanda no mês de dezembro, em 2006 e 2017.}
\end{table}
\section{Guaianases}
\begin{figure}[H]
	\centering
	\includegraphics[width=0.66\linewidth]{../Analises/graficos/tempo/tempo_GUAIANASES}
	\includegraphics[width=0.33\linewidth]{../Analises/mapas/mapas/distrito_GUA}
	\caption{Evolução das matrículas, da demanda, e da população de 0 a 3 anos no distrito Guaianases e a sua localização no município, respectivamente.}
\end{figure}
\begin{table}[H]
	\begin{tabular}{|l|l|l|l|}
		\hline
		\textbf{}                                      & \textbf{junho de 2006}       & \textbf{dezembro de 2017}    & \textbf{Variação} \\ \hline
		\textbf{Número de matrículas}                  & 750 & 5438 & 625.07\% \\ \hline
		\textbf{Crianças na fila}                      & 509 & 59 & -88.41\% \\ \hline
		\textbf{População de 0 a 3 anos estimada}      & 7129 & 7037 & -1.29\% \\ \hline
		\textbf{Matrículas relativas à população}      & 10.52\% & 77.28\% & 634.55\% \\ \hline
		\textbf{Fila relativa à população}             & 7.14\% & 0.84\% & -88.26\% \\ \hline
		\textbf{Proporção das matrículas no município} & 1.21\% & 1.84\% & 51.08\% \\ \hline
		\textbf{Proporção da demanda no município}     & 0.6\% & 0.13\% & -77.81\% \\ \hline
		\textbf{Proporção da população de 0 a 3 anos}  & 1.14\% & 1.09\% & -3.83\% \\ \hline
		\textbf{Posição no \textit{ranking} das matrículas}     & 31 & 18 & 13 \\ \hline
		\textbf{Posição no \textit{ranking} da demanda}         & 55 & 89 & -34 \\ \hline
	\end{tabular}
	\caption{Comparação entre o primeiro e o último período da amostra}
\end{table}
\begin{table}[H]
	\begin{tabular}{|l|l|l|l|}
		\hline
		\textbf{}                                 & \textbf{dezembro de 2006} & \textbf{dezembro de 2017} & \textbf{Variação} \\ \hline
		\textbf{Crianças na fila}                      & 1365 & 59 & -95.68\% \\ \hline
		\textbf{População de 0 a 3 anos estimada}      & 7129 & 7037 & -1.29\% \\ \hline
		\textbf{Fila relativa à população}             & 13.58\% & 0.84\% & -93.83\% \\ \hline
	\end{tabular}
	\caption{Comparação da demanda no mês de dezembro, em 2006 e 2017.}
\end{table}
\section{Iguatemi}
\begin{figure}[H]
	\centering
	\includegraphics[width=0.66\linewidth]{../Analises/graficos/tempo/tempo_IGUATEMI}
	\includegraphics[width=0.33\linewidth]{../Analises/mapas/mapas/distrito_IGU}
	\caption{Evolução das matrículas, da demanda, e da população de 0 a 3 anos no distrito Iguatemi e a sua localização no município, respectivamente.}
\end{figure}
\begin{table}[H]
	\begin{tabular}{|l|l|l|l|}
		\hline
		\textbf{}                                      & \textbf{junho de 2006}       & \textbf{dezembro de 2017}    & \textbf{Variação} \\ \hline
		\textbf{Número de matrículas}                  & 704 & 5240 & 644.32\% \\ \hline
		\textbf{Crianças na fila}                      & 1296 & 1035 & -20.14\% \\ \hline
		\textbf{População de 0 a 3 anos estimada}      & 8361 & 9634 & 15.23\% \\ \hline
		\textbf{Matrículas relativas à população}      & 8.42\% & 54.39\% & 545.97\% \\ \hline
		\textbf{Fila relativa à população}             & 15.5\% & 10.74\% & -30.69\% \\ \hline
		\textbf{Proporção das matrículas no município} & 1.14\% & 1.77\% & 55.09\% \\ \hline
		\textbf{Proporção da demanda no município}     & 1.54\% & 2.35\% & 52.88\% \\ \hline
		\textbf{Proporção da população de 0 a 3 anos}  & 1.33\% & 1.5\% & 12.27\% \\ \hline
		\textbf{Posição no \textit{ranking} das matrículas}     & 35 & 20 & 15 \\ \hline
		\textbf{Posição no \textit{ranking} da demanda}         & 19 & 12 & 7 \\ \hline
	\end{tabular}
	\caption{Comparação entre o primeiro e o último período da amostra}
\end{table}
\begin{table}[H]
	\begin{tabular}{|l|l|l|l|}
		\hline
		\textbf{}                                 & \textbf{dezembro de 2006} & \textbf{dezembro de 2017} & \textbf{Variação} \\ \hline
		\textbf{Crianças na fila}                      & 2274 & 1035 & -54.49\% \\ \hline
		\textbf{População de 0 a 3 anos estimada}      & 8361 & 9634 & 15.23\% \\ \hline
		\textbf{Fila relativa à população}             & 9.75\% & 10.74\% & 10.19\% \\ \hline
	\end{tabular}
	\caption{Comparação da demanda no mês de dezembro, em 2006 e 2017.}
\end{table}
\section{Ipiranga}
\begin{figure}[H]
	\centering
	\includegraphics[width=0.66\linewidth]{../Analises/graficos/tempo/tempo_IPIRANGA}
	\includegraphics[width=0.33\linewidth]{../Analises/mapas/mapas/distrito_IPI}
	\caption{Evolução das matrículas, da demanda, e da população de 0 a 3 anos no distrito Ipiranga e a sua localização no município, respectivamente.}
\end{figure}
\begin{table}[H]
	\begin{tabular}{|l|l|l|l|}
		\hline
		\textbf{}                                      & \textbf{junho de 2006}       & \textbf{dezembro de 2017}    & \textbf{Variação} \\ \hline
		\textbf{Número de matrículas}                  & 728 & 2465 & 238.6\% \\ \hline
		\textbf{Crianças na fila}                      & 605 & 100 & -83.47\% \\ \hline
		\textbf{População de 0 a 3 anos estimada}      & 4800 & 5232 & 9.0\% \\ \hline
		\textbf{Matrículas relativas à população}      & 15.17\% & 47.11\% & 210.64\% \\ \hline
		\textbf{Fila relativa à população}             & 12.6\% & 1.91\% & -84.84\% \\ \hline
		\textbf{Proporção das matrículas no município} & 1.18\% & 0.83\% & -29.45\% \\ \hline
		\textbf{Proporção da demanda no município}     & 0.72\% & 0.23\% & -68.36\% \\ \hline
		\textbf{Proporção da população de 0 a 3 anos}  & 0.76\% & 0.81\% & 6.2\% \\ \hline
		\textbf{Posição no \textit{ranking} das matrículas}     & 33 & 45 & -12 \\ \hline
		\textbf{Posição no \textit{ranking} da demanda}         & 45 & 83 & -38 \\ \hline
	\end{tabular}
	\caption{Comparação entre o primeiro e o último período da amostra}
\end{table}
\begin{table}[H]
	\begin{tabular}{|l|l|l|l|}
		\hline
		\textbf{}                                 & \textbf{dezembro de 2006} & \textbf{dezembro de 2017} & \textbf{Variação} \\ \hline
		\textbf{Crianças na fila}                      & 1025 & 100 & -90.24\% \\ \hline
		\textbf{População de 0 a 3 anos estimada}      & 4800 & 5232 & 9.0\% \\ \hline
		\textbf{Fila relativa à população}             & 16.06\% & 1.91\% & -88.1\% \\ \hline
	\end{tabular}
	\caption{Comparação da demanda no mês de dezembro, em 2006 e 2017.}
\end{table}
\section{Itaim Bibi}
\begin{figure}[H]
	\centering
	\includegraphics[width=0.66\linewidth]{../Analises/graficos/tempo/tempo_ITAIM_BIBI}
	\includegraphics[width=0.33\linewidth]{../Analises/mapas/mapas/distrito_IBI}
	\caption{Evolução das matrículas, da demanda, e da população de 0 a 3 anos no distrito Itaim Bibi e a sua localização no município, respectivamente.}
\end{figure}
\begin{table}[H]
	\begin{tabular}{|l|l|l|l|}
		\hline
		\textbf{}                                      & \textbf{junho de 2006}       & \textbf{dezembro de 2017}    & \textbf{Variação} \\ \hline
		\textbf{Número de matrículas}                  & 95 & 392 & 312.63\% \\ \hline
		\textbf{Crianças na fila}                      & 124 & 180 & 45.16\% \\ \hline
		\textbf{População de 0 a 3 anos estimada}      & 3217 & 3998 & 24.28\% \\ \hline
		\textbf{Matrículas relativas à população}      & 2.95\% & 9.8\% & 232.02\% \\ \hline
		\textbf{Fila relativa à população}             & 3.85\% & 4.5\% & 16.8\% \\ \hline
		\textbf{Proporção das matrículas no município} & 0.15\% & 0.13\% & -14.02\% \\ \hline
		\textbf{Proporção da demanda no município}     & 0.15\% & 0.41\% & 177.89\% \\ \hline
		\textbf{Proporção da população de 0 a 3 anos}  & 0.51\% & 0.62\% & 21.09\% \\ \hline
		\textbf{Posição no \textit{ranking} das matrículas}     & 91 & 89 & 2 \\ \hline
		\textbf{Posição no \textit{ranking} da demanda}         & 91 & 53 & 38 \\ \hline
	\end{tabular}
	\caption{Comparação entre o primeiro e o último período da amostra}
\end{table}
\begin{table}[H]
	\begin{tabular}{|l|l|l|l|}
		\hline
		\textbf{}                                 & \textbf{dezembro de 2006} & \textbf{dezembro de 2017} & \textbf{Variação} \\ \hline
		\textbf{Crianças na fila}                      & 234 & 180 & -23.08\% \\ \hline
		\textbf{População de 0 a 3 anos estimada}      & 3217 & 3998 & 24.28\% \\ \hline
		\textbf{Fila relativa à população}             & 4.63\% & 4.5\% & -2.76\% \\ \hline
	\end{tabular}
	\caption{Comparação da demanda no mês de dezembro, em 2006 e 2017.}
\end{table}
\section{Itaim Paulista}
\begin{figure}[H]
	\centering
	\includegraphics[width=0.66\linewidth]{../Analises/graficos/tempo/tempo_ITAIM_PAULISTA}
	\includegraphics[width=0.33\linewidth]{../Analises/mapas/mapas/distrito_IPA}
	\caption{Evolução das matrículas, da demanda, e da população de 0 a 3 anos no distrito Itaim Paulista e a sua localização no município, respectivamente.}
\end{figure}
\begin{table}[H]
	\begin{tabular}{|l|l|l|l|}
		\hline
		\textbf{}                                      & \textbf{junho de 2006}       & \textbf{dezembro de 2017}    & \textbf{Variação} \\ \hline
		\textbf{Número de matrículas}                  & 1284 & 9897 & 670.79\% \\ \hline
		\textbf{Crianças na fila}                      & 1963 & 512 & -73.92\% \\ \hline
		\textbf{População de 0 a 3 anos estimada}      & 15318 & 14390 & -6.06\% \\ \hline
		\textbf{Matrículas relativas à população}      & 8.38\% & 68.78\% & 720.5\% \\ \hline
		\textbf{Fila relativa à população}             & 12.81\% & 3.56\% & -72.24\% \\ \hline
		\textbf{Proporção das matrículas no município} & 2.08\% & 3.34\% & 60.6\% \\ \hline
		\textbf{Proporção da demanda no município}     & 2.33\% & 1.16\% & -50.07\% \\ \hline
		\textbf{Proporção da população de 0 a 3 anos}  & 2.44\% & 2.23\% & -8.47\% \\ \hline
		\textbf{Posição no \textit{ranking} das matrículas}     & 12 & 4 & 8 \\ \hline
		\textbf{Posição no \textit{ranking} da demanda}         & 12 & 23 & -11 \\ \hline
	\end{tabular}
	\caption{Comparação entre o primeiro e o último período da amostra}
\end{table}
\begin{table}[H]
	\begin{tabular}{|l|l|l|l|}
		\hline
		\textbf{}                                 & \textbf{dezembro de 2006} & \textbf{dezembro de 2017} & \textbf{Variação} \\ \hline
		\textbf{Crianças na fila}                      & 3374 & 512 & -84.83\% \\ \hline
		\textbf{População de 0 a 3 anos estimada}      & 15318 & 14390 & -6.06\% \\ \hline
		\textbf{Fila relativa à população}             & 8.73\% & 3.56\% & -59.24\% \\ \hline
	\end{tabular}
	\caption{Comparação da demanda no mês de dezembro, em 2006 e 2017.}
\end{table}
\section{Itaquera}
\begin{figure}[H]
	\centering
	\includegraphics[width=0.66\linewidth]{../Analises/graficos/tempo/tempo_ITAQUERA}
	\includegraphics[width=0.33\linewidth]{../Analises/mapas/mapas/distrito_ITQ}
	\caption{Evolução das matrículas, da demanda, e da população de 0 a 3 anos no distrito Itaquera e a sua localização no município, respectivamente.}
\end{figure}
\begin{table}[H]
	\begin{tabular}{|l|l|l|l|}
		\hline
		\textbf{}                                      & \textbf{junho de 2006}       & \textbf{dezembro de 2017}    & \textbf{Variação} \\ \hline
		\textbf{Número de matrículas}                  & 843 & 7211 & 755.4\% \\ \hline
		\textbf{Crianças na fila}                      & 1120 & 260 & -76.79\% \\ \hline
		\textbf{População de 0 a 3 anos estimada}      & 13114 & 12238 & -6.68\% \\ \hline
		\textbf{Matrículas relativas à população}      & 6.43\% & 58.92\% & 816.63\% \\ \hline
		\textbf{Fila relativa à população}             & 8.54\% & 2.12\% & -75.12\% \\ \hline
		\textbf{Proporção das matrículas no município} & 1.37\% & 2.43\% & 78.23\% \\ \hline
		\textbf{Proporção da demanda no município}     & 1.33\% & 0.59\% & -55.56\% \\ \hline
		\textbf{Proporção da população de 0 a 3 anos}  & 2.09\% & 1.9\% & -9.08\% \\ \hline
		\textbf{Posição no \textit{ranking} das matrículas}     & 27 & 10 & 17 \\ \hline
		\textbf{Posição no \textit{ranking} da demanda}         & 26 & 41 & -15 \\ \hline
	\end{tabular}
	\caption{Comparação entre o primeiro e o último período da amostra}
\end{table}
\begin{table}[H]
	\begin{tabular}{|l|l|l|l|}
		\hline
		\textbf{}                                 & \textbf{dezembro de 2006} & \textbf{dezembro de 2017} & \textbf{Variação} \\ \hline
		\textbf{Crianças na fila}                      & 2069 & 260 & -87.43\% \\ \hline
		\textbf{População de 0 a 3 anos estimada}      & 13114 & 12238 & -6.68\% \\ \hline
		\textbf{Fila relativa à população}             & 6.52\% & 2.12\% & -67.42\% \\ \hline
	\end{tabular}
	\caption{Comparação da demanda no mês de dezembro, em 2006 e 2017.}
\end{table}
\section{Jabaquara}
\begin{figure}[H]
	\centering
	\includegraphics[width=0.66\linewidth]{../Analises/graficos/tempo/tempo_JABAQUARA}
	\includegraphics[width=0.33\linewidth]{../Analises/mapas/mapas/distrito_JAB}
	\caption{Evolução das matrículas, da demanda, e da população de 0 a 3 anos no distrito Jabaquara e a sua localização no município, respectivamente.}
\end{figure}
\begin{table}[H]
	\begin{tabular}{|l|l|l|l|}
		\hline
		\textbf{}                                      & \textbf{junho de 2006}       & \textbf{dezembro de 2017}    & \textbf{Variação} \\ \hline
		\textbf{Número de matrículas}                  & 1004 & 4746 & 372.71\% \\ \hline
		\textbf{Crianças na fila}                      & 2279 & 708 & -68.93\% \\ \hline
		\textbf{População de 0 a 3 anos estimada}      & 11995 & 11844 & -1.26\% \\ \hline
		\textbf{Matrículas relativas à população}      & 8.37\% & 40.07\% & 378.74\% \\ \hline
		\textbf{Fila relativa à população}             & 19.0\% & 5.98\% & -68.54\% \\ \hline
		\textbf{Proporção das matrículas no município} & 1.63\% & 1.6\% & -1.51\% \\ \hline
		\textbf{Proporção da demanda no município}     & 2.7\% & 1.61\% & -40.53\% \\ \hline
		\textbf{Proporção da população de 0 a 3 anos}  & 1.91\% & 1.84\% & -3.8\% \\ \hline
		\textbf{Posição no \textit{ranking} das matrículas}     & 16 & 22 & -6 \\ \hline
		\textbf{Posição no \textit{ranking} da demanda}         & 10 & 17 & -7 \\ \hline
	\end{tabular}
	\caption{Comparação entre o primeiro e o último período da amostra}
\end{table}
\begin{table}[H]
	\begin{tabular}{|l|l|l|l|}
		\hline
		\textbf{}                                 & \textbf{dezembro de 2006} & \textbf{dezembro de 2017} & \textbf{Variação} \\ \hline
		\textbf{Crianças na fila}                      & 2961 & 708 & -76.09\% \\ \hline
		\textbf{População de 0 a 3 anos estimada}      & 11995 & 11844 & -1.26\% \\ \hline
		\textbf{Fila relativa à população}             & 8.67\% & 5.98\% & -31.05\% \\ \hline
	\end{tabular}
	\caption{Comparação da demanda no mês de dezembro, em 2006 e 2017.}
\end{table}
\section{Jaçanã}
\begin{figure}[H]
	\centering
	\includegraphics[width=0.66\linewidth]{../Analises/graficos/tempo/tempo_JACANA}
	\includegraphics[width=0.33\linewidth]{../Analises/mapas/mapas/distrito_JAC}
	\caption{Evolução das matrículas, da demanda, e da população de 0 a 3 anos no distrito Jaçanã e a sua localização no município, respectivamente.}
\end{figure}
\begin{table}[H]
	\begin{tabular}{|l|l|l|l|}
		\hline
		\textbf{}                                      & \textbf{junho de 2006}       & \textbf{dezembro de 2017}    & \textbf{Variação} \\ \hline
		\textbf{Número de matrículas}                  & 600 & 2894 & 382.33\% \\ \hline
		\textbf{Crianças na fila}                      & 830 & 755 & -9.04\% \\ \hline
		\textbf{População de 0 a 3 anos estimada}      & 5505 & 4858 & -11.75\% \\ \hline
		\textbf{Matrículas relativas à população}      & 10.9\% & 59.57\% & 446.57\% \\ \hline
		\textbf{Fila relativa à população}             & 15.08\% & 15.54\% & 3.08\% \\ \hline
		\textbf{Proporção das matrículas no município} & 0.97\% & 0.98\% & 0.5\% \\ \hline
		\textbf{Proporção da demanda no município}     & 0.98\% & 1.71\% & 74.14\% \\ \hline
		\textbf{Proporção da população de 0 a 3 anos}  & 0.88\% & 0.75\% & -14.02\% \\ \hline
		\textbf{Posição no \textit{ranking} das matrículas}     & 39 & 40 & -1 \\ \hline
		\textbf{Posição no \textit{ranking} da demanda}         & 33 & 16 & 17 \\ \hline
	\end{tabular}
	\caption{Comparação entre o primeiro e o último período da amostra}
\end{table}
\begin{table}[H]
	\begin{tabular}{|l|l|l|l|}
		\hline
		\textbf{}                                 & \textbf{dezembro de 2006} & \textbf{dezembro de 2017} & \textbf{Variação} \\ \hline
		\textbf{Crianças na fila}                      & 1416 & 755 & -46.68\% \\ \hline
		\textbf{População de 0 a 3 anos estimada}      & 5505 & 4858 & -11.75\% \\ \hline
		\textbf{Fila relativa à população}             & 10.88\% & 15.54\% & 42.84\% \\ \hline
	\end{tabular}
	\caption{Comparação da demanda no mês de dezembro, em 2006 e 2017.}
\end{table}
\section{Jaguara}
\begin{figure}[H]
	\centering
	\includegraphics[width=0.66\linewidth]{../Analises/graficos/tempo/tempo_JAGUARA}
	\includegraphics[width=0.33\linewidth]{../Analises/mapas/mapas/distrito_JAG}
	\caption{Evolução das matrículas, da demanda, e da população de 0 a 3 anos no distrito Jaguara e a sua localização no município, respectivamente.}
\end{figure}
\begin{table}[H]
	\begin{tabular}{|l|l|l|l|}
		\hline
		\textbf{}                                      & \textbf{junho de 2006}       & \textbf{dezembro de 2017}    & \textbf{Variação} \\ \hline
		\textbf{Número de matrículas}                  & 324 & 1019 & 214.51\% \\ \hline
		\textbf{Crianças na fila}                      & 330 & 171 & -48.18\% \\ \hline
		\textbf{População de 0 a 3 anos estimada}      & 1133 & 1140 & 0.62\% \\ \hline
		\textbf{Matrículas relativas à população}      & 28.6\% & 89.39\% & 212.57\% \\ \hline
		\textbf{Fila relativa à população}             & 29.13\% & 15.0\% & -48.5\% \\ \hline
		\textbf{Proporção das matrículas no município} & 0.52\% & 0.34\% & -34.47\% \\ \hline
		\textbf{Proporção da demanda no município}     & 0.39\% & 0.39\% & -0.8\% \\ \hline
		\textbf{Proporção da população de 0 a 3 anos}  & 0.18\% & 0.18\% & -1.97\% \\ \hline
		\textbf{Posição no \textit{ranking} das matrículas}     & 63 & 70 & -7 \\ \hline
		\textbf{Posição no \textit{ranking} da demanda}         & 71 & 57 & 14 \\ \hline
	\end{tabular}
	\caption{Comparação entre o primeiro e o último período da amostra}
\end{table}
\begin{table}[H]
	\begin{tabular}{|l|l|l|l|}
		\hline
		\textbf{}                                 & \textbf{dezembro de 2006} & \textbf{dezembro de 2017} & \textbf{Variação} \\ \hline
		\textbf{Crianças na fila}                      & 563 & 171 & -69.63\% \\ \hline
		\textbf{População de 0 a 3 anos estimada}      & 1133 & 1140 & 0.62\% \\ \hline
		\textbf{Fila relativa à população}             & 28.68\% & 15.0\% & -47.7\% \\ \hline
	\end{tabular}
	\caption{Comparação da demanda no mês de dezembro, em 2006 e 2017.}
\end{table}
\section{Jaguaré}
\begin{figure}[H]
	\centering
	\includegraphics[width=0.66\linewidth]{../Analises/graficos/tempo/tempo_JAGUARE}
	\includegraphics[width=0.33\linewidth]{../Analises/mapas/mapas/distrito_JRE}
	\caption{Evolução das matrículas, da demanda, e da população de 0 a 3 anos no distrito Jaguaré e a sua localização no município, respectivamente.}
\end{figure}
\begin{table}[H]
	\begin{tabular}{|l|l|l|l|}
		\hline
		\textbf{}                                      & \textbf{junho de 2006}       & \textbf{dezembro de 2017}    & \textbf{Variação} \\ \hline
		\textbf{Número de matrículas}                  & 230 & 1675 & 628.26\% \\ \hline
		\textbf{Crianças na fila}                      & 448 & 313 & -30.13\% \\ \hline
		\textbf{População de 0 a 3 anos estimada}      & 2729 & 2886 & 5.75\% \\ \hline
		\textbf{Matrículas relativas à população}      & 8.43\% & 58.04\% & 588.64\% \\ \hline
		\textbf{Fila relativa à população}             & 16.42\% & 10.85\% & -33.93\% \\ \hline
		\textbf{Proporção das matrículas no município} & 0.37\% & 0.57\% & 51.74\% \\ \hline
		\textbf{Proporção da demanda no município}     & 0.53\% & 0.71\% & 33.75\% \\ \hline
		\textbf{Proporção da população de 0 a 3 anos}  & 0.43\% & 0.45\% & 3.04\% \\ \hline
		\textbf{Posição no \textit{ranking} das matrículas}     & 74 & 53 & 21 \\ \hline
		\textbf{Posição no \textit{ranking} da demanda}         & 60 & 37 & 23 \\ \hline
	\end{tabular}
	\caption{Comparação entre o primeiro e o último período da amostra}
\end{table}
\begin{table}[H]
	\begin{tabular}{|l|l|l|l|}
		\hline
		\textbf{}                                 & \textbf{dezembro de 2006} & \textbf{dezembro de 2017} & \textbf{Variação} \\ \hline
		\textbf{Crianças na fila}                      & 548 & 313 & -42.88\% \\ \hline
		\textbf{População de 0 a 3 anos estimada}      & 2729 & 2886 & 5.75\% \\ \hline
		\textbf{Fila relativa à população}             & 8.65\% & 10.85\% & 25.38\% \\ \hline
	\end{tabular}
	\caption{Comparação da demanda no mês de dezembro, em 2006 e 2017.}
\end{table}
\section{Jaraguá}
\begin{figure}[H]
	\centering
	\includegraphics[width=0.66\linewidth]{../Analises/graficos/tempo/tempo_JARAGUA}
	\includegraphics[width=0.33\linewidth]{../Analises/mapas/mapas/distrito_JAR}
	\caption{Evolução das matrículas, da demanda, e da população de 0 a 3 anos no distrito Jaraguá e a sua localização no município, respectivamente.}
\end{figure}
\begin{table}[H]
	\begin{tabular}{|l|l|l|l|}
		\hline
		\textbf{}                                      & \textbf{junho de 2006}       & \textbf{dezembro de 2017}    & \textbf{Variação} \\ \hline
		\textbf{Número de matrículas}                  & 2008 & 7870 & 291.93\% \\ \hline
		\textbf{Crianças na fila}                      & 1609 & 626 & -61.09\% \\ \hline
		\textbf{População de 0 a 3 anos estimada}      & 11207 & 12705 & 13.37\% \\ \hline
		\textbf{Matrículas relativas à população}      & 17.92\% & 61.94\% & 245.72\% \\ \hline
		\textbf{Fila relativa à população}             & 14.36\% & 4.93\% & -65.68\% \\ \hline
		\textbf{Proporção das matrículas no município} & 3.25\% & 2.66\% & -18.34\% \\ \hline
		\textbf{Proporção da demanda no município}     & 1.91\% & 1.42\% & -25.52\% \\ \hline
		\textbf{Proporção da população de 0 a 3 anos}  & 1.79\% & 1.97\% & 10.45\% \\ \hline
		\textbf{Posição no \textit{ranking} das matrículas}     & 5 & 9 & -4 \\ \hline
		\textbf{Posição no \textit{ranking} da demanda}         & 15 & 18 & -3 \\ \hline
	\end{tabular}
	\caption{Comparação entre o primeiro e o último período da amostra}
\end{table}
\begin{table}[H]
	\begin{tabular}{|l|l|l|l|}
		\hline
		\textbf{}                                 & \textbf{dezembro de 2006} & \textbf{dezembro de 2017} & \textbf{Variação} \\ \hline
		\textbf{Crianças na fila}                      & 3062 & 626 & -79.56\% \\ \hline
		\textbf{População de 0 a 3 anos estimada}      & 11207 & 12705 & 13.37\% \\ \hline
		\textbf{Fila relativa à população}             & 18.53\% & 4.93\% & -73.41\% \\ \hline
	\end{tabular}
	\caption{Comparação da demanda no mês de dezembro, em 2006 e 2017.}
\end{table}
\section{Jardim Ângela}
\begin{figure}[H]
	\centering
	\includegraphics[width=0.66\linewidth]{../Analises/graficos/tempo/tempo_JARDIM_ANGELA}
	\includegraphics[width=0.33\linewidth]{../Analises/mapas/mapas/distrito_JDA}
	\caption{Evolução das matrículas, da demanda, e da população de 0 a 3 anos no distrito Jardim Ângela e a sua localização no município, respectivamente.}
\end{figure}
\begin{table}[H]
	\begin{tabular}{|l|l|l|l|}
		\hline
		\textbf{}                                      & \textbf{junho de 2006}       & \textbf{dezembro de 2017}    & \textbf{Variação} \\ \hline
		\textbf{Número de matrículas}                  & 1683 & 9408 & 459.0\% \\ \hline
		\textbf{Crianças na fila}                      & 3539 & 2683 & -24.19\% \\ \hline
		\textbf{População de 0 a 3 anos estimada}      & 20961 & 21567 & 2.89\% \\ \hline
		\textbf{Matrículas relativas à população}      & 8.03\% & 43.62\% & 443.29\% \\ \hline
		\textbf{Fila relativa à população}             & 16.88\% & 12.44\% & -26.32\% \\ \hline
		\textbf{Proporção das matrículas no município} & 2.73\% & 3.18\% & 16.47\% \\ \hline
		\textbf{Proporção da demanda no município}     & 4.19\% & 6.09\% & 45.13\% \\ \hline
		\textbf{Proporção da população de 0 a 3 anos}  & 3.34\% & 3.35\% & 0.25\% \\ \hline
		\textbf{Posição no \textit{ranking} das matrículas}     & 7 & 5 & 2 \\ \hline
		\textbf{Posição no \textit{ranking} da demanda}         & 3 & 2 & 1 \\ \hline
	\end{tabular}
	\caption{Comparação entre o primeiro e o último período da amostra}
\end{table}
\begin{table}[H]
	\begin{tabular}{|l|l|l|l|}
		\hline
		\textbf{}                                 & \textbf{dezembro de 2006} & \textbf{dezembro de 2017} & \textbf{Variação} \\ \hline
		\textbf{Crianças na fila}                      & 5404 & 2683 & -50.35\% \\ \hline
		\textbf{População de 0 a 3 anos estimada}      & 20961 & 21567 & 2.89\% \\ \hline
		\textbf{Fila relativa à população}             & 7.61\% & 12.44\% & 63.47\% \\ \hline
	\end{tabular}
	\caption{Comparação da demanda no mês de dezembro, em 2006 e 2017.}
\end{table}
\section{Jardim Helena}
\begin{figure}[H]
	\centering
	\includegraphics[width=0.66\linewidth]{../Analises/graficos/tempo/tempo_JARDIM_HELENA}
	\includegraphics[width=0.33\linewidth]{../Analises/mapas/mapas/distrito_JDH}
	\caption{Evolução das matrículas, da demanda, e da população de 0 a 3 anos no distrito Jardim Helena e a sua localização no município, respectivamente.}
\end{figure}
\begin{table}[H]
	\begin{tabular}{|l|l|l|l|}
		\hline
		\textbf{}                                      & \textbf{junho de 2006}       & \textbf{dezembro de 2017}    & \textbf{Variação} \\ \hline
		\textbf{Número de matrículas}                  & 792 & 5452 & 588.38\% \\ \hline
		\textbf{Crianças na fila}                      & 851 & 481 & -43.48\% \\ \hline
		\textbf{População de 0 a 3 anos estimada}      & 9715 & 9261 & -4.67\% \\ \hline
		\textbf{Matrículas relativas à população}      & 8.15\% & 58.87\% & 622.13\% \\ \hline
		\textbf{Fila relativa à população}             & 8.76\% & 5.19\% & -40.71\% \\ \hline
		\textbf{Proporção das matrículas no município} & 1.28\% & 1.84\% & 43.43\% \\ \hline
		\textbf{Proporção da demanda no município}     & 1.01\% & 1.09\% & 8.2\% \\ \hline
		\textbf{Proporção da população de 0 a 3 anos}  & 1.55\% & 1.44\% & -7.12\% \\ \hline
		\textbf{Posição no \textit{ranking} das matrículas}     & 29 & 17 & 12 \\ \hline
		\textbf{Posição no \textit{ranking} da demanda}         & 32 & 25 & 7 \\ \hline
	\end{tabular}
	\caption{Comparação entre o primeiro e o último período da amostra}
\end{table}
\begin{table}[H]
	\begin{tabular}{|l|l|l|l|}
		\hline
		\textbf{}                                 & \textbf{dezembro de 2006} & \textbf{dezembro de 2017} & \textbf{Variação} \\ \hline
		\textbf{Crianças na fila}                      & 2159 & 481 & -77.72\% \\ \hline
		\textbf{População de 0 a 3 anos estimada}      & 9715 & 9261 & -4.67\% \\ \hline
		\textbf{Fila relativa à população}             & 9.77\% & 5.19\% & -46.84\% \\ \hline
	\end{tabular}
	\caption{Comparação da demanda no mês de dezembro, em 2006 e 2017.}
\end{table}
\section{Jardim Paulista}
\begin{figure}[H]
	\centering
	\includegraphics[width=0.66\linewidth]{../Analises/graficos/tempo/tempo_JARDIM_PAULISTA}
	\includegraphics[width=0.33\linewidth]{../Analises/mapas/mapas/distrito_JDP}
	\caption{Evolução das matrículas, da demanda, e da população de 0 a 3 anos no distrito Jardim Paulista e a sua localização no município, respectivamente.}
\end{figure}
\begin{table}[H]
	\begin{tabular}{|l|l|l|l|}
		\hline
		\textbf{}                                      & \textbf{junho de 2006}       & \textbf{dezembro de 2017}    & \textbf{Variação} \\ \hline
		\textbf{Número de matrículas}                  & 24 & 106 & 341.67\% \\ \hline
		\textbf{Crianças na fila}                      & 3 & 18 & 500.0\% \\ \hline
		\textbf{População de 0 a 3 anos estimada}      & 2566 & 3424 & 33.44\% \\ \hline
		\textbf{Matrículas relativas à população}      & 0.94\% & 3.1\% & 230.99\% \\ \hline
		\textbf{Fila relativa à população}             & 0.12\% & 0.53\% & 349.65\% \\ \hline
		\textbf{Proporção das matrículas no município} & 0.04\% & 0.04\% & -7.97\% \\ \hline
		\textbf{Proporção da demanda no município}     & 0.0\% & 0.04\% & 1048.62\% \\ \hline
		\textbf{Proporção da população de 0 a 3 anos}  & 0.41\% & 0.53\% & 30.01\% \\ \hline
		\textbf{Posição no \textit{ranking} das matrículas}     & 96 & 94 & 2 \\ \hline
		\textbf{Posição no \textit{ranking} da demanda}         & 95 & 95 & 0 \\ \hline
	\end{tabular}
	\caption{Comparação entre o primeiro e o último período da amostra}
\end{table}
\begin{table}[H]
	\begin{tabular}{|l|l|l|l|}
		\hline
		\textbf{}                                 & \textbf{dezembro de 2006} & \textbf{dezembro de 2017} & \textbf{Variação} \\ \hline
		\textbf{Crianças na fila}                      & 42 & 18 & -57.14\% \\ \hline
		\textbf{População de 0 a 3 anos estimada}      & 2566 & 3424 & 33.44\% \\ \hline
		\textbf{Fila relativa à população}             & 0.94\% & 0.53\% & -44.07\% \\ \hline
	\end{tabular}
	\caption{Comparação da demanda no mês de dezembro, em 2006 e 2017.}
\end{table}
\section{Jardim São Luís}
\begin{figure}[H]
	\centering
	\includegraphics[width=0.66\linewidth]{../Analises/graficos/tempo/tempo_JARDIM_SAO_LUIS}
	\includegraphics[width=0.33\linewidth]{../Analises/mapas/mapas/distrito_JDS}
	\caption{Evolução das matrículas, da demanda, e da população de 0 a 3 anos no distrito Jardim São Luís e a sua localização no município, respectivamente.}
\end{figure}
\begin{table}[H]
	\begin{tabular}{|l|l|l|l|}
		\hline
		\textbf{}                                      & \textbf{junho de 2006}       & \textbf{dezembro de 2017}    & \textbf{Variação} \\ \hline
		\textbf{Número de matrículas}                  & 1847 & 9257 & 401.19\% \\ \hline
		\textbf{Crianças na fila}                      & 3674 & 2370 & -35.49\% \\ \hline
		\textbf{População de 0 a 3 anos estimada}      & 16691 & 17445 & 4.52\% \\ \hline
		\textbf{Matrículas relativas à população}      & 11.07\% & 53.06\% & 379.53\% \\ \hline
		\textbf{Fila relativa à população}             & 22.01\% & 13.59\% & -38.28\% \\ \hline
		\textbf{Proporção das matrículas no município} & 2.99\% & 3.12\% & 4.43\% \\ \hline
		\textbf{Proporção da demanda no município}     & 4.35\% & 5.38\% & 23.49\% \\ \hline
		\textbf{Proporção da população de 0 a 3 anos}  & 2.66\% & 2.71\% & 1.83\% \\ \hline
		\textbf{Posição no \textit{ranking} das matrículas}     & 6 & 7 & -1 \\ \hline
		\textbf{Posição no \textit{ranking} da demanda}         & 2 & 3 & -1 \\ \hline
	\end{tabular}
	\caption{Comparação entre o primeiro e o último período da amostra}
\end{table}
\begin{table}[H]
	\begin{tabular}{|l|l|l|l|}
		\hline
		\textbf{}                                 & \textbf{dezembro de 2006} & \textbf{dezembro de 2017} & \textbf{Variação} \\ \hline
		\textbf{Crianças na fila}                      & 5671 & 2370 & -58.21\% \\ \hline
		\textbf{População de 0 a 3 anos estimada}      & 16691 & 17445 & 4.52\% \\ \hline
		\textbf{Fila relativa à população}             & 11.81\% & 13.59\% & 15.03\% \\ \hline
	\end{tabular}
	\caption{Comparação da demanda no mês de dezembro, em 2006 e 2017.}
\end{table}
\section{José Bonifácio}
\begin{figure}[H]
	\centering
	\includegraphics[width=0.66\linewidth]{../Analises/graficos/tempo/tempo_JOSE_BONIFACIO}
	\includegraphics[width=0.33\linewidth]{../Analises/mapas/mapas/distrito_JBO}
	\caption{Evolução das matrículas, da demanda, e da população de 0 a 3 anos no distrito José Bonifácio e a sua localização no município, respectivamente.}
\end{figure}
\begin{table}[H]
	\begin{tabular}{|l|l|l|l|}
		\hline
		\textbf{}                                      & \textbf{junho de 2006}       & \textbf{dezembro de 2017}    & \textbf{Variação} \\ \hline
		\textbf{Número de matrículas}                  & 473 & 3569 & 654.55\% \\ \hline
		\textbf{Crianças na fila}                      & 789 & 83 & -89.48\% \\ \hline
		\textbf{População de 0 a 3 anos estimada}      & 7160 & 6716 & -6.2\% \\ \hline
		\textbf{Matrículas relativas à população}      & 6.61\% & 53.14\% & 704.43\% \\ \hline
		\textbf{Fila relativa à população}             & 11.02\% & 1.24\% & -88.78\% \\ \hline
		\textbf{Proporção das matrículas no município} & 0.77\% & 1.2\% & 57.22\% \\ \hline
		\textbf{Proporção da demanda no município}     & 0.93\% & 0.19\% & -79.86\% \\ \hline
		\textbf{Proporção da população de 0 a 3 anos}  & 1.14\% & 1.04\% & -8.61\% \\ \hline
		\textbf{Posição no \textit{ranking} das matrículas}     & 49 & 31 & 18 \\ \hline
		\textbf{Posição no \textit{ranking} da demanda}         & 35 & 85 & -50 \\ \hline
	\end{tabular}
	\caption{Comparação entre o primeiro e o último período da amostra}
\end{table}
\begin{table}[H]
	\begin{tabular}{|l|l|l|l|}
		\hline
		\textbf{}                                 & \textbf{dezembro de 2006} & \textbf{dezembro de 2017} & \textbf{Variação} \\ \hline
		\textbf{Crianças na fila}                      & 1103 & 83 & -92.48\% \\ \hline
		\textbf{População de 0 a 3 anos estimada}      & 7160 & 6716 & -6.2\% \\ \hline
		\textbf{Fila relativa à população}             & 7.4\% & 1.24\% & -83.3\% \\ \hline
	\end{tabular}
	\caption{Comparação da demanda no mês de dezembro, em 2006 e 2017.}
\end{table}
\section{Lajeado}
\begin{figure}[H]
	\centering
	\includegraphics[width=0.66\linewidth]{../Analises/graficos/tempo/tempo_LAJEADO}
	\includegraphics[width=0.33\linewidth]{../Analises/mapas/mapas/distrito_LAJ}
	\caption{Evolução das matrículas, da demanda, e da população de 0 a 3 anos no distrito Lajeado e a sua localização no município, respectivamente.}
\end{figure}
\begin{table}[H]
	\begin{tabular}{|l|l|l|l|}
		\hline
		\textbf{}                                      & \textbf{junho de 2006}       & \textbf{dezembro de 2017}    & \textbf{Variação} \\ \hline
		\textbf{Número de matrículas}                  & 947 & 9361 & 888.49\% \\ \hline
		\textbf{Crianças na fila}                      & 1611 & 101 & -93.73\% \\ \hline
		\textbf{População de 0 a 3 anos estimada}      & 11968 & 11308 & -5.51\% \\ \hline
		\textbf{Matrículas relativas à população}      & 7.91\% & 82.78\% & 946.18\% \\ \hline
		\textbf{Fila relativa à população}             & 13.46\% & 0.89\% & -93.36\% \\ \hline
		\textbf{Proporção das matrículas no município} & 1.53\% & 3.16\% & 105.96\% \\ \hline
		\textbf{Proporção da demanda no município}     & 1.91\% & 0.23\% & -88.0\% \\ \hline
		\textbf{Proporção da população de 0 a 3 anos}  & 1.91\% & 1.76\% & -7.94\% \\ \hline
		\textbf{Posição no \textit{ranking} das matrículas}     & 20 & 6 & 14 \\ \hline
		\textbf{Posição no \textit{ranking} da demanda}         & 14 & 82 & -68 \\ \hline
	\end{tabular}
	\caption{Comparação entre o primeiro e o último período da amostra}
\end{table}
\begin{table}[H]
	\begin{tabular}{|l|l|l|l|}
		\hline
		\textbf{}                                 & \textbf{dezembro de 2006} & \textbf{dezembro de 2017} & \textbf{Variação} \\ \hline
		\textbf{Crianças na fila}                      & 2479 & 101 & -95.93\% \\ \hline
		\textbf{População de 0 a 3 anos estimada}      & 11968 & 11308 & -5.51\% \\ \hline
		\textbf{Fila relativa à população}             & 9.26\% & 0.89\% & -90.35\% \\ \hline
	\end{tabular}
	\caption{Comparação da demanda no mês de dezembro, em 2006 e 2017.}
\end{table}
\section{Lapa}
\begin{figure}[H]
	\centering
	\includegraphics[width=0.66\linewidth]{../Analises/graficos/tempo/tempo_LAPA}
	\includegraphics[width=0.33\linewidth]{../Analises/mapas/mapas/distrito_LAP}
	\caption{Evolução das matrículas, da demanda, e da população de 0 a 3 anos no distrito Lapa e a sua localização no município, respectivamente.}
\end{figure}
\begin{table}[H]
	\begin{tabular}{|l|l|l|l|}
		\hline
		\textbf{}                                      & \textbf{junho de 2006}       & \textbf{dezembro de 2017}    & \textbf{Variação} \\ \hline
		\textbf{Número de matrículas}                  & 185 & 515 & 178.38\% \\ \hline
		\textbf{Crianças na fila}                      & 132 & 142 & 7.58\% \\ \hline
		\textbf{População de 0 a 3 anos estimada}      & 2330 & 2686 & 15.28\% \\ \hline
		\textbf{Matrículas relativas à população}      & 7.94\% & 19.17\% & 141.48\% \\ \hline
		\textbf{Fila relativa à população}             & 5.67\% & 5.29\% & -6.68\% \\ \hline
		\textbf{Proporção das matrículas no município} & 0.3\% & 0.17\% & -42.0\% \\ \hline
		\textbf{Proporção da demanda no município}     & 0.16\% & 0.32\% & 105.94\% \\ \hline
		\textbf{Proporção da população de 0 a 3 anos}  & 0.37\% & 0.42\% & 12.32\% \\ \hline
		\textbf{Posição no \textit{ranking} das matrículas}     & 82 & 82 & 0 \\ \hline
		\textbf{Posição no \textit{ranking} da demanda}         & 90 & 67 & 23 \\ \hline
	\end{tabular}
	\caption{Comparação entre o primeiro e o último período da amostra}
\end{table}
\begin{table}[H]
	\begin{tabular}{|l|l|l|l|}
		\hline
		\textbf{}                                 & \textbf{dezembro de 2006} & \textbf{dezembro de 2017} & \textbf{Variação} \\ \hline
		\textbf{Crianças na fila}                      & 386 & 142 & -63.21\% \\ \hline
		\textbf{População de 0 a 3 anos estimada}      & 2330 & 2686 & 15.28\% \\ \hline
		\textbf{Fila relativa à população}             & 8.11\% & 5.29\% & -34.81\% \\ \hline
	\end{tabular}
	\caption{Comparação da demanda no mês de dezembro, em 2006 e 2017.}
\end{table}
\section{Liberdade}
\begin{figure}[H]
	\centering
	\includegraphics[width=0.66\linewidth]{../Analises/graficos/tempo/tempo_LIBERDADE}
	\includegraphics[width=0.33\linewidth]{../Analises/mapas/mapas/distrito_LIB}
	\caption{Evolução das matrículas, da demanda, e da população de 0 a 3 anos no distrito Liberdade e a sua localização no município, respectivamente.}
\end{figure}
\begin{table}[H]
	\begin{tabular}{|l|l|l|l|}
		\hline
		\textbf{}                                      & \textbf{junho de 2006}       & \textbf{dezembro de 2017}    & \textbf{Variação} \\ \hline
		\textbf{Número de matrículas}                  & 391 & 646 & 65.22\% \\ \hline
		\textbf{Crianças na fila}                      & 330 & 150 & -54.55\% \\ \hline
		\textbf{População de 0 a 3 anos estimada}      & 2614 & 3292 & 25.94\% \\ \hline
		\textbf{Matrículas relativas à população}      & 14.96\% & 19.62\% & 31.19\% \\ \hline
		\textbf{Fila relativa à população}             & 12.62\% & 4.56\% & -63.91\% \\ \hline
		\textbf{Proporção das matrículas no município} & 0.63\% & 0.22\% & -65.58\% \\ \hline
		\textbf{Proporção da demanda no município}     & 0.39\% & 0.34\% & -12.98\% \\ \hline
		\textbf{Proporção da população de 0 a 3 anos}  & 0.42\% & 0.51\% & 22.7\% \\ \hline
		\textbf{Posição no \textit{ranking} das matrículas}     & 57 & 78 & -21 \\ \hline
		\textbf{Posição no \textit{ranking} da demanda}         & 70 & 64 & 6 \\ \hline
	\end{tabular}
	\caption{Comparação entre o primeiro e o último período da amostra}
\end{table}
\begin{table}[H]
	\begin{tabular}{|l|l|l|l|}
		\hline
		\textbf{}                                 & \textbf{dezembro de 2006} & \textbf{dezembro de 2017} & \textbf{Variação} \\ \hline
		\textbf{Crianças na fila}                      & 507 & 150 & -70.41\% \\ \hline
		\textbf{População de 0 a 3 anos estimada}      & 2614 & 3292 & 25.94\% \\ \hline
		\textbf{Fila relativa à população}             & 16.56\% & 4.56\% & -72.48\% \\ \hline
	\end{tabular}
	\caption{Comparação da demanda no mês de dezembro, em 2006 e 2017.}
\end{table}
\section{Limão}
\begin{figure}[H]
	\centering
	\includegraphics[width=0.66\linewidth]{../Analises/graficos/tempo/tempo_LIMAO}
	\includegraphics[width=0.33\linewidth]{../Analises/mapas/mapas/distrito_LIM}
	\caption{Evolução das matrículas, da demanda, e da população de 0 a 3 anos no distrito Limão e a sua localização no município, respectivamente.}
\end{figure}
\begin{table}[H]
	\begin{tabular}{|l|l|l|l|}
		\hline
		\textbf{}                                      & \textbf{junho de 2006}       & \textbf{dezembro de 2017}    & \textbf{Variação} \\ \hline
		\textbf{Número de matrículas}                  & 336 & 1573 & 368.15\% \\ \hline
		\textbf{Crianças na fila}                      & 305 & 145 & -52.46\% \\ \hline
		\textbf{População de 0 a 3 anos estimada}      & 4291 & 4652 & 8.41\% \\ \hline
		\textbf{Matrículas relativas à população}      & 7.83\% & 33.81\% & 331.83\% \\ \hline
		\textbf{Fila relativa à população}             & 7.11\% & 3.12\% & -56.15\% \\ \hline
		\textbf{Proporção das matrículas no município} & 0.54\% & 0.53\% & -2.45\% \\ \hline
		\textbf{Proporção da demanda no município}     & 0.36\% & 0.33\% & -8.99\% \\ \hline
		\textbf{Proporção da população de 0 a 3 anos}  & 0.68\% & 0.72\% & 5.63\% \\ \hline
		\textbf{Posição no \textit{ranking} das matrículas}     & 61 & 56 & 5 \\ \hline
		\textbf{Posição no \textit{ranking} da demanda}         & 73 & 66 & 7 \\ \hline
	\end{tabular}
	\caption{Comparação entre o primeiro e o último período da amostra}
\end{table}
\begin{table}[H]
	\begin{tabular}{|l|l|l|l|}
		\hline
		\textbf{}                                 & \textbf{dezembro de 2006} & \textbf{dezembro de 2017} & \textbf{Variação} \\ \hline
		\textbf{Crianças na fila}                      & 353 & 145 & -58.92\% \\ \hline
		\textbf{População de 0 a 3 anos estimada}      & 4291 & 4652 & 8.41\% \\ \hline
		\textbf{Fila relativa à população}             & 7.43\% & 3.12\% & -58.05\% \\ \hline
	\end{tabular}
	\caption{Comparação da demanda no mês de dezembro, em 2006 e 2017.}
\end{table}
\section{Mandaqui}
\begin{figure}[H]
	\centering
	\includegraphics[width=0.66\linewidth]{../Analises/graficos/tempo/tempo_MANDAQUI}
	\includegraphics[width=0.33\linewidth]{../Analises/mapas/mapas/distrito_MAN}
	\caption{Evolução das matrículas, da demanda, e da população de 0 a 3 anos no distrito Mandaqui e a sua localização no município, respectivamente.}
\end{figure}
\begin{table}[H]
	\begin{tabular}{|l|l|l|l|}
		\hline
		\textbf{}                                      & \textbf{junho de 2006}       & \textbf{dezembro de 2017}    & \textbf{Variação} \\ \hline
		\textbf{Número de matrículas}                  & 440 & 1785 & 305.68\% \\ \hline
		\textbf{Crianças na fila}                      & 755 & 294 & -61.06\% \\ \hline
		\textbf{População de 0 a 3 anos estimada}      & 5108 & 5057 & -1.0\% \\ \hline
		\textbf{Matrículas relativas à população}      & 8.61\% & 35.3\% & 309.77\% \\ \hline
		\textbf{Fila relativa à população}             & 14.78\% & 5.81\% & -60.67\% \\ \hline
		\textbf{Proporção das matrículas no município} & 0.71\% & 0.6\% & -15.47\% \\ \hline
		\textbf{Proporção da demanda no município}     & 0.89\% & 0.67\% & -25.45\% \\ \hline
		\textbf{Proporção da população de 0 a 3 anos}  & 0.81\% & 0.79\% & -3.54\% \\ \hline
		\textbf{Posição no \textit{ranking} das matrículas}     & 53 & 50 & 3 \\ \hline
		\textbf{Posição no \textit{ranking} da demanda}         & 37 & 40 & -3 \\ \hline
	\end{tabular}
	\caption{Comparação entre o primeiro e o último período da amostra}
\end{table}
\begin{table}[H]
	\begin{tabular}{|l|l|l|l|}
		\hline
		\textbf{}                                 & \textbf{dezembro de 2006} & \textbf{dezembro de 2017} & \textbf{Variação} \\ \hline
		\textbf{Crianças na fila}                      & 991 & 294 & -70.33\% \\ \hline
		\textbf{População de 0 a 3 anos estimada}      & 5108 & 5057 & -1.0\% \\ \hline
		\textbf{Fila relativa à população}             & 9.59\% & 5.81\% & -39.38\% \\ \hline
	\end{tabular}
	\caption{Comparação da demanda no mês de dezembro, em 2006 e 2017.}
\end{table}
\section{Marsilac}
\begin{figure}[H]
	\centering
	\includegraphics[width=0.66\linewidth]{../Analises/graficos/tempo/tempo_MARSILAC}
	\includegraphics[width=0.33\linewidth]{../Analises/mapas/mapas/distrito_MAR}
	\caption{Evolução das matrículas, da demanda, e da população de 0 a 3 anos no distrito Marsilac e a sua localização no município, respectivamente.}
\end{figure}
\begin{table}[H]
	\begin{tabular}{|l|l|l|l|}
		\hline
		\textbf{}                                      & \textbf{junho de 2006}       & \textbf{dezembro de 2017}    & \textbf{Variação} \\ \hline
		\textbf{Número de matrículas}                  & 63 & 421 & 568.25\% \\ \hline
		\textbf{Crianças na fila}                      & 1 & 112 & 11100.0\% \\ \hline
		\textbf{População de 0 a 3 anos estimada}      & 616 & 420 & -31.82\% \\ \hline
		\textbf{Matrículas relativas à população}      & 10.23\% & 100.24\% & 880.11\% \\ \hline
		\textbf{Fila relativa à população}             & 0.16\% & 26.67\% & 16326.67\% \\ \hline
		\textbf{Proporção das matrículas no município} & 0.1\% & 0.14\% & 39.24\% \\ \hline
		\textbf{Proporção da demanda no município}     & 0.0\% & 0.25\% & 21340.84\% \\ \hline
		\textbf{Proporção da população de 0 a 3 anos}  & 0.1\% & 0.07\% & -33.57\% \\ \hline
		\textbf{Posição no \textit{ranking} das matrículas}     & 93 & 86 & 7 \\ \hline
		\textbf{Posição no \textit{ranking} da demanda}         & 96 & 78 & 18 \\ \hline
	\end{tabular}
	\caption{Comparação entre o primeiro e o último período da amostra}
\end{table}
\begin{table}[H]
	\begin{tabular}{|l|l|l|l|}
		\hline
		\textbf{}                                 & \textbf{dezembro de 2006} & \textbf{dezembro de 2017} & \textbf{Variação} \\ \hline
		\textbf{Crianças na fila}                      & 1 & 112 & 11100.0\% \\ \hline
		\textbf{População de 0 a 3 anos estimada}      & 616 & 420 & -31.82\% \\ \hline
		\textbf{Fila relativa à população}             & 10.88\% & 26.67\% & 145.1\% \\ \hline
	\end{tabular}
	\caption{Comparação da demanda no mês de dezembro, em 2006 e 2017.}
\end{table}
\section{Moema}
\begin{figure}[H]
	\centering
	\includegraphics[width=0.66\linewidth]{../Analises/graficos/tempo/tempo_MOEMA}
	\includegraphics[width=0.33\linewidth]{../Analises/mapas/mapas/distrito_MOE}
	\caption{Evolução das matrículas, da demanda, e da população de 0 a 3 anos no distrito Moema e a sua localização no município, respectivamente.}
\end{figure}
\begin{table}[H]
	\begin{tabular}{|l|l|l|l|}
		\hline
		\textbf{}                                      & \textbf{junho de 2006}       & \textbf{dezembro de 2017}    & \textbf{Variação} \\ \hline
		\textbf{Número de matrículas}                  & 146 & 401 & 174.66\% \\ \hline
		\textbf{Crianças na fila}                      & 113 & 81 & -28.32\% \\ \hline
		\textbf{População de 0 a 3 anos estimada}      & 2901 & 3186 & 9.82\% \\ \hline
		\textbf{Matrículas relativas à população}      & 5.03\% & 12.59\% & 150.09\% \\ \hline
		\textbf{Fila relativa à população}             & 3.9\% & 2.54\% & -34.73\% \\ \hline
		\textbf{Proporção das matrículas no município} & 0.24\% & 0.14\% & -42.77\% \\ \hline
		\textbf{Proporção da demanda no município}     & 0.13\% & 0.18\% & 37.22\% \\ \hline
		\textbf{Proporção da população de 0 a 3 anos}  & 0.46\% & 0.49\% & 7.0\% \\ \hline
		\textbf{Posição no \textit{ranking} das matrículas}     & 85 & 88 & -3 \\ \hline
		\textbf{Posição no \textit{ranking} da demanda}         & 93 & 86 & 7 \\ \hline
	\end{tabular}
	\caption{Comparação entre o primeiro e o último período da amostra}
\end{table}
\begin{table}[H]
	\begin{tabular}{|l|l|l|l|}
		\hline
		\textbf{}                                 & \textbf{dezembro de 2006} & \textbf{dezembro de 2017} & \textbf{Variação} \\ \hline
		\textbf{Crianças na fila}                      & 178 & 81 & -54.49\% \\ \hline
		\textbf{População de 0 a 3 anos estimada}      & 2901 & 3186 & 9.82\% \\ \hline
		\textbf{Fila relativa à população}             & 5.17\% & 2.54\% & -50.82\% \\ \hline
	\end{tabular}
	\caption{Comparação da demanda no mês de dezembro, em 2006 e 2017.}
\end{table}
\section{Mooca}
\begin{figure}[H]
	\centering
	\includegraphics[width=0.66\linewidth]{../Analises/graficos/tempo/tempo_MOOCA}
	\includegraphics[width=0.33\linewidth]{../Analises/mapas/mapas/distrito_MOO}
	\caption{Evolução das matrículas, da demanda, e da população de 0 a 3 anos no distrito Mooca e a sua localização no município, respectivamente.}
\end{figure}
\begin{table}[H]
	\begin{tabular}{|l|l|l|l|}
		\hline
		\textbf{}                                      & \textbf{junho de 2006}       & \textbf{dezembro de 2017}    & \textbf{Variação} \\ \hline
		\textbf{Número de matrículas}                  & 551 & 778 & 41.2\% \\ \hline
		\textbf{Crianças na fila}                      & 444 & 125 & -71.85\% \\ \hline
		\textbf{População de 0 a 3 anos estimada}      & 2811 & 3579 & 27.32\% \\ \hline
		\textbf{Matrículas relativas à população}      & 19.6\% & 21.74\% & 10.9\% \\ \hline
		\textbf{Fila relativa à população}             & 15.8\% & 3.49\% & -77.89\% \\ \hline
		\textbf{Proporção das matrículas no município} & 0.89\% & 0.26\% & -70.58\% \\ \hline
		\textbf{Proporção da demanda no município}     & 0.53\% & 0.28\% & -46.1\% \\ \hline
		\textbf{Proporção da população de 0 a 3 anos}  & 0.45\% & 0.56\% & 24.05\% \\ \hline
		\textbf{Posição no \textit{ranking} das matrículas}     & 42 & 74 & -32 \\ \hline
		\textbf{Posição no \textit{ranking} da demanda}         & 61 & 72 & -11 \\ \hline
	\end{tabular}
	\caption{Comparação entre o primeiro e o último período da amostra}
\end{table}
\begin{table}[H]
	\begin{tabular}{|l|l|l|l|}
		\hline
		\textbf{}                                 & \textbf{dezembro de 2006} & \textbf{dezembro de 2017} & \textbf{Variação} \\ \hline
		\textbf{Crianças na fila}                      & 358 & 125 & -65.08\% \\ \hline
		\textbf{População de 0 a 3 anos estimada}      & 2811 & 3579 & 27.32\% \\ \hline
		\textbf{Fila relativa à população}             & 14.37\% & 3.49\% & -75.7\% \\ \hline
	\end{tabular}
	\caption{Comparação da demanda no mês de dezembro, em 2006 e 2017.}
\end{table}
\section{Morumbi}
\begin{figure}[H]
	\centering
	\includegraphics[width=0.66\linewidth]{../Analises/graficos/tempo/tempo_MORUMBI}
	\includegraphics[width=0.33\linewidth]{../Analises/mapas/mapas/distrito_MOR}
	\caption{Evolução das matrículas, da demanda, e da população de 0 a 3 anos no distrito Morumbi e a sua localização no município, respectivamente.}
\end{figure}
\begin{table}[H]
	\begin{tabular}{|l|l|l|l|}
		\hline
		\textbf{}                                      & \textbf{junho de 2006}       & \textbf{dezembro de 2017}    & \textbf{Variação} \\ \hline
		\textbf{Número de matrículas}                  & 209 & 944 & 351.67\% \\ \hline
		\textbf{Crianças na fila}                      & 402 & 517 & 28.61\% \\ \hline
		\textbf{População de 0 a 3 anos estimada}      & 2025 & 2868 & 41.63\% \\ \hline
		\textbf{Matrículas relativas à população}      & 10.32\% & 32.91\% & 218.91\% \\ \hline
		\textbf{Fila relativa à população}             & 19.85\% & 18.03\% & -9.19\% \\ \hline
		\textbf{Proporção das matrículas no município} & 0.34\% & 0.32\% & -5.89\% \\ \hline
		\textbf{Proporção da demanda no município}     & 0.48\% & 1.17\% & 146.2\% \\ \hline
		\textbf{Proporção da população de 0 a 3 anos}  & 0.32\% & 0.45\% & 37.99\% \\ \hline
		\textbf{Posição no \textit{ranking} das matrículas}     & 78 & 71 & 7 \\ \hline
		\textbf{Posição no \textit{ranking} da demanda}         & 66 & 22 & 44 \\ \hline
	\end{tabular}
	\caption{Comparação entre o primeiro e o último período da amostra}
\end{table}
\begin{table}[H]
	\begin{tabular}{|l|l|l|l|}
		\hline
		\textbf{}                                 & \textbf{dezembro de 2006} & \textbf{dezembro de 2017} & \textbf{Variação} \\ \hline
		\textbf{Crianças na fila}                      & 528 & 517 & -2.08\% \\ \hline
		\textbf{População de 0 a 3 anos estimada}      & 2025 & 2868 & 41.63\% \\ \hline
		\textbf{Fila relativa à população}             & 10.27\% & 18.03\% & 75.53\% \\ \hline
	\end{tabular}
	\caption{Comparação da demanda no mês de dezembro, em 2006 e 2017.}
\end{table}
\section{Parelheiros}
\begin{figure}[H]
	\centering
	\includegraphics[width=0.66\linewidth]{../Analises/graficos/tempo/tempo_PARELHEIROS}
	\includegraphics[width=0.33\linewidth]{../Analises/mapas/mapas/distrito_PLH}
	\caption{Evolução das matrículas, da demanda, e da população de 0 a 3 anos no distrito Parelheiros e a sua localização no município, respectivamente.}
\end{figure}
\begin{table}[H]
	\begin{tabular}{|l|l|l|l|}
		\hline
		\textbf{}                                      & \textbf{junho de 2006}       & \textbf{dezembro de 2017}    & \textbf{Variação} \\ \hline
		\textbf{Número de matrículas}                  & 980 & 5573 & 468.67\% \\ \hline
		\textbf{Crianças na fila}                      & 775 & 848 & 9.42\% \\ \hline
		\textbf{População de 0 a 3 anos estimada}      & 9073 & 10320 & 13.74\% \\ \hline
		\textbf{Matrículas relativas à população}      & 10.8\% & 54.0\% & 399.96\% \\ \hline
		\textbf{Fila relativa à população}             & 8.54\% & 8.22\% & -3.8\% \\ \hline
		\textbf{Proporção das matrículas no município} & 1.59\% & 1.88\% & 18.49\% \\ \hline
		\textbf{Proporção da demanda no município}     & 0.92\% & 1.92\% & 109.47\% \\ \hline
		\textbf{Proporção da população de 0 a 3 anos}  & 1.45\% & 1.6\% & 10.82\% \\ \hline
		\textbf{Posição no \textit{ranking} das matrículas}     & 19 & 16 & 3 \\ \hline
		\textbf{Posição no \textit{ranking} da demanda}         & 36 & 15 & 21 \\ \hline
	\end{tabular}
	\caption{Comparação entre o primeiro e o último período da amostra}
\end{table}
\begin{table}[H]
	\begin{tabular}{|l|l|l|l|}
		\hline
		\textbf{}                                 & \textbf{dezembro de 2006} & \textbf{dezembro de 2017} & \textbf{Variação} \\ \hline
		\textbf{Crianças na fila}                      & 1320 & 848 & -35.76\% \\ \hline
		\textbf{População de 0 a 3 anos estimada}      & 9073 & 10320 & 13.74\% \\ \hline
		\textbf{Fila relativa à população}             & 11.71\% & 8.22\% & -29.83\% \\ \hline
	\end{tabular}
	\caption{Comparação da demanda no mês de dezembro, em 2006 e 2017.}
\end{table}
\section{Pari}
\begin{figure}[H]
	\centering
	\includegraphics[width=0.66\linewidth]{../Analises/graficos/tempo/tempo_PARI}
	\includegraphics[width=0.33\linewidth]{../Analises/mapas/mapas/distrito_PRI}
	\caption{Evolução das matrículas, da demanda, e da população de 0 a 3 anos no distrito Pari e a sua localização no município, respectivamente.}
\end{figure}
\begin{table}[H]
	\begin{tabular}{|l|l|l|l|}
		\hline
		\textbf{}                                      & \textbf{junho de 2006}       & \textbf{dezembro de 2017}    & \textbf{Variação} \\ \hline
		\textbf{Número de matrículas}                  & 54 & 430 & 696.3\% \\ \hline
		\textbf{Crianças na fila}                      & 178 & 170 & -4.49\% \\ \hline
		\textbf{População de 0 a 3 anos estimada}      & 820 & 1236 & 50.73\% \\ \hline
		\textbf{Matrículas relativas à população}      & 6.59\% & 34.79\% & 428.29\% \\ \hline
		\textbf{Fila relativa à população}             & 21.71\% & 13.75\% & -36.64\% \\ \hline
		\textbf{Proporção das matrículas no município} & 0.09\% & 0.15\% & 65.92\% \\ \hline
		\textbf{Proporção da demanda no município}     & 0.21\% & 0.39\% & 82.83\% \\ \hline
		\textbf{Proporção da população de 0 a 3 anos}  & 0.13\% & 0.19\% & 46.86\% \\ \hline
		\textbf{Posição no \textit{ranking} das matrículas}     & 94 & 85 & 9 \\ \hline
		\textbf{Posição no \textit{ranking} da demanda}         & 83 & 58 & 25 \\ \hline
	\end{tabular}
	\caption{Comparação entre o primeiro e o último período da amostra}
\end{table}
\begin{table}[H]
	\begin{tabular}{|l|l|l|l|}
		\hline
		\textbf{}                                 & \textbf{dezembro de 2006} & \textbf{dezembro de 2017} & \textbf{Variação} \\ \hline
		\textbf{Crianças na fila}                      & 271 & 170 & -37.27\% \\ \hline
		\textbf{População de 0 a 3 anos estimada}      & 820 & 1236 & 50.73\% \\ \hline
		\textbf{Fila relativa à população}             & 10.24\% & 13.75\% & 34.32\% \\ \hline
	\end{tabular}
	\caption{Comparação da demanda no mês de dezembro, em 2006 e 2017.}
\end{table}
\section{Parque do Carmo}
\begin{figure}[H]
	\centering
	\includegraphics[width=0.66\linewidth]{../Analises/graficos/tempo/tempo_PARQUE_DO_CARMO}
	\includegraphics[width=0.33\linewidth]{../Analises/mapas/mapas/distrito_PQC}
	\caption{Evolução das matrículas, da demanda, e da população de 0 a 3 anos no distrito Parque do Carmo e a sua localização no município, respectivamente.}
\end{figure}
\begin{table}[H]
	\begin{tabular}{|l|l|l|l|}
		\hline
		\textbf{}                                      & \textbf{junho de 2006}       & \textbf{dezembro de 2017}    & \textbf{Variação} \\ \hline
		\textbf{Número de matrículas}                  & 590 & 2265 & 283.9\% \\ \hline
		\textbf{Crianças na fila}                      & 526 & 155 & -70.53\% \\ \hline
		\textbf{População de 0 a 3 anos estimada}      & 4293 & 4130 & -3.8\% \\ \hline
		\textbf{Matrículas relativas à população}      & 13.74\% & 54.84\% & 299.05\% \\ \hline
		\textbf{Fila relativa à população}             & 12.25\% & 3.75\% & -69.37\% \\ \hline
		\textbf{Proporção das matrículas no município} & 0.96\% & 0.76\% & -20.01\% \\ \hline
		\textbf{Proporção da demanda no município}     & 0.62\% & 0.35\% & -43.59\% \\ \hline
		\textbf{Proporção da população de 0 a 3 anos}  & 0.68\% & 0.64\% & -6.27\% \\ \hline
		\textbf{Posição no \textit{ranking} das matrículas}     & 40 & 47 & -7 \\ \hline
		\textbf{Posição no \textit{ranking} da demanda}         & 53 & 62 & -9 \\ \hline
	\end{tabular}
	\caption{Comparação entre o primeiro e o último período da amostra}
\end{table}
\begin{table}[H]
	\begin{tabular}{|l|l|l|l|}
		\hline
		\textbf{}                                 & \textbf{dezembro de 2006} & \textbf{dezembro de 2017} & \textbf{Variação} \\ \hline
		\textbf{Crianças na fila}                      & 1215 & 155 & -87.24\% \\ \hline
		\textbf{População de 0 a 3 anos estimada}      & 4293 & 4130 & -3.8\% \\ \hline
		\textbf{Fila relativa à população}             & 13.84\% & 3.75\% & -72.88\% \\ \hline
	\end{tabular}
	\caption{Comparação da demanda no mês de dezembro, em 2006 e 2017.}
\end{table}
\section{Pedreira}
\begin{figure}[H]
	\centering
	\includegraphics[width=0.66\linewidth]{../Analises/graficos/tempo/tempo_PEDREIRA}
	\includegraphics[width=0.33\linewidth]{../Analises/mapas/mapas/distrito_PDR}
	\caption{Evolução das matrículas, da demanda, e da população de 0 a 3 anos no distrito Pedreira e a sua localização no município, respectivamente.}
\end{figure}
\begin{table}[H]
	\begin{tabular}{|l|l|l|l|}
		\hline
		\textbf{}                                      & \textbf{junho de 2006}       & \textbf{dezembro de 2017}    & \textbf{Variação} \\ \hline
		\textbf{Número de matrículas}                  & 493 & 3135 & 535.9\% \\ \hline
		\textbf{Crianças na fila}                      & 1184 & 1774 & 49.83\% \\ \hline
		\textbf{População de 0 a 3 anos estimada}      & 9671 & 8928 & -7.68\% \\ \hline
		\textbf{Matrículas relativas à população}      & 5.1\% & 35.11\% & 588.82\% \\ \hline
		\textbf{Fila relativa à população}             & 12.24\% & 19.87\% & 62.3\% \\ \hline
		\textbf{Proporção das matrículas no município} & 0.8\% & 1.06\% & 32.5\% \\ \hline
		\textbf{Proporção da demanda no município}     & 1.4\% & 4.02\% & 186.83\% \\ \hline
		\textbf{Proporção da população de 0 a 3 anos}  & 1.54\% & 1.39\% & -10.05\% \\ \hline
		\textbf{Posição no \textit{ranking} das matrículas}     & 45 & 36 & 9 \\ \hline
		\textbf{Posição no \textit{ranking} da demanda}         & 21 & 7 & 14 \\ \hline
	\end{tabular}
	\caption{Comparação entre o primeiro e o último período da amostra}
\end{table}
\begin{table}[H]
	\begin{tabular}{|l|l|l|l|}
		\hline
		\textbf{}                                 & \textbf{dezembro de 2006} & \textbf{dezembro de 2017} & \textbf{Variação} \\ \hline
		\textbf{Crianças na fila}                      & 1738 & 1774 & 2.07\% \\ \hline
		\textbf{População de 0 a 3 anos estimada}      & 9671 & 8928 & -7.68\% \\ \hline
		\textbf{Fila relativa à população}             & 5.69\% & 19.87\% & 249.21\% \\ \hline
	\end{tabular}
	\caption{Comparação da demanda no mês de dezembro, em 2006 e 2017.}
\end{table}
\section{Penha}
\begin{figure}[H]
	\centering
	\includegraphics[width=0.66\linewidth]{../Analises/graficos/tempo/tempo_PENHA}
	\includegraphics[width=0.33\linewidth]{../Analises/mapas/mapas/distrito_PEN}
	\caption{Evolução das matrículas, da demanda, e da população de 0 a 3 anos no distrito Penha e a sua localização no município, respectivamente.}
\end{figure}
\begin{table}[H]
	\begin{tabular}{|l|l|l|l|}
		\hline
		\textbf{}                                      & \textbf{junho de 2006}       & \textbf{dezembro de 2017}    & \textbf{Variação} \\ \hline
		\textbf{Número de matrículas}                  & 447 & 3022 & 576.06\% \\ \hline
		\textbf{Crianças na fila}                      & 623 & 55 & -91.17\% \\ \hline
		\textbf{População de 0 a 3 anos estimada}      & 5852 & 6084 & 3.96\% \\ \hline
		\textbf{Matrículas relativas à população}      & 7.64\% & 49.67\% & 550.28\% \\ \hline
		\textbf{Fila relativa à população}             & 10.65\% & 0.9\% & -91.51\% \\ \hline
		\textbf{Proporção das matrículas no município} & 0.72\% & 1.02\% & 40.87\% \\ \hline
		\textbf{Proporção da demanda no município}     & 0.74\% & 0.12\% & -83.1\% \\ \hline
		\textbf{Proporção da população de 0 a 3 anos}  & 0.93\% & 0.94\% & 1.29\% \\ \hline
		\textbf{Posição no \textit{ranking} das matrículas}     & 52 & 38 & 14 \\ \hline
		\textbf{Posição no \textit{ranking} da demanda}         & 44 & 90 & -46 \\ \hline
	\end{tabular}
	\caption{Comparação entre o primeiro e o último período da amostra}
\end{table}
\begin{table}[H]
	\begin{tabular}{|l|l|l|l|}
		\hline
		\textbf{}                                 & \textbf{dezembro de 2006} & \textbf{dezembro de 2017} & \textbf{Variação} \\ \hline
		\textbf{Crianças na fila}                      & 995 & 55 & -94.47\% \\ \hline
		\textbf{População de 0 a 3 anos estimada}      & 5852 & 6084 & 3.96\% \\ \hline
		\textbf{Fila relativa à população}             & 8.0\% & 0.9\% & -88.7\% \\ \hline
	\end{tabular}
	\caption{Comparação da demanda no mês de dezembro, em 2006 e 2017.}
\end{table}
\section{Perdizes}
\begin{figure}[H]
	\centering
	\includegraphics[width=0.66\linewidth]{../Analises/graficos/tempo/tempo_PERDIZES}
	\includegraphics[width=0.33\linewidth]{../Analises/mapas/mapas/distrito_PRD}
	\caption{Evolução das matrículas, da demanda, e da população de 0 a 3 anos no distrito Perdizes e a sua localização no município, respectivamente.}
\end{figure}
\begin{table}[H]
	\begin{tabular}{|l|l|l|l|}
		\hline
		\textbf{}                                      & \textbf{junho de 2006}       & \textbf{dezembro de 2017}    & \textbf{Variação} \\ \hline
		\textbf{Número de matrículas}                  & 286 & 532 & 86.01\% \\ \hline
		\textbf{Crianças na fila}                      & 200 & 120 & -40.0\% \\ \hline
		\textbf{População de 0 a 3 anos estimada}      & 3778 & 4170 & 10.38\% \\ \hline
		\textbf{Matrículas relativas à população}      & 7.57\% & 12.76\% & 68.53\% \\ \hline
		\textbf{Fila relativa à população}             & 5.29\% & 2.88\% & -45.64\% \\ \hline
		\textbf{Proporção das matrículas no município} & 0.46\% & 0.18\% & -61.24\% \\ \hline
		\textbf{Proporção da demanda no município}     & 0.24\% & 0.27\% & 14.86\% \\ \hline
		\textbf{Proporção da população de 0 a 3 anos}  & 0.6\% & 0.65\% & 7.54\% \\ \hline
		\textbf{Posição no \textit{ranking} das matrículas}     & 66 & 81 & -15 \\ \hline
		\textbf{Posição no \textit{ranking} da demanda}         & 81 & 74 & 7 \\ \hline
	\end{tabular}
	\caption{Comparação entre o primeiro e o último período da amostra}
\end{table}
\begin{table}[H]
	\begin{tabular}{|l|l|l|l|}
		\hline
		\textbf{}                                 & \textbf{dezembro de 2006} & \textbf{dezembro de 2017} & \textbf{Variação} \\ \hline
		\textbf{Crianças na fila}                      & 349 & 120 & -65.62\% \\ \hline
		\textbf{População de 0 a 3 anos estimada}      & 3778 & 4170 & 10.38\% \\ \hline
		\textbf{Fila relativa à população}             & 8.02\% & 2.88\% & -64.12\% \\ \hline
	\end{tabular}
	\caption{Comparação da demanda no mês de dezembro, em 2006 e 2017.}
\end{table}
\section{Perus}
\begin{figure}[H]
	\centering
	\includegraphics[width=0.66\linewidth]{../Analises/graficos/tempo/tempo_PERUS}
	\includegraphics[width=0.33\linewidth]{../Analises/mapas/mapas/distrito_PRS}
	\caption{Evolução das matrículas, da demanda, e da população de 0 a 3 anos no distrito Perus e a sua localização no município, respectivamente.}
\end{figure}
\begin{table}[H]
	\begin{tabular}{|l|l|l|l|}
		\hline
		\textbf{}                                      & \textbf{junho de 2006}       & \textbf{dezembro de 2017}    & \textbf{Variação} \\ \hline
		\textbf{Número de matrículas}                  & 484 & 3484 & 619.83\% \\ \hline
		\textbf{Crianças na fila}                      & 911 & 176 & -80.68\% \\ \hline
		\textbf{População de 0 a 3 anos estimada}      & 5408 & 5885 & 8.82\% \\ \hline
		\textbf{Matrículas relativas à população}      & 8.95\% & 59.2\% & 561.49\% \\ \hline
		\textbf{Fila relativa à população}             & 16.85\% & 2.99\% & -82.25\% \\ \hline
		\textbf{Proporção das matrículas no município} & 0.78\% & 1.18\% & 49.99\% \\ \hline
		\textbf{Proporção da demanda no município}     & 1.08\% & 0.4\% & -63.02\% \\ \hline
		\textbf{Proporção da população de 0 a 3 anos}  & 0.86\% & 0.91\% & 6.03\% \\ \hline
		\textbf{Posição no \textit{ranking} das matrículas}     & 47 & 32 & 15 \\ \hline
		\textbf{Posição no \textit{ranking} da demanda}         & 30 & 55 & -25 \\ \hline
	\end{tabular}
	\caption{Comparação entre o primeiro e o último período da amostra}
\end{table}
\begin{table}[H]
	\begin{tabular}{|l|l|l|l|}
		\hline
		\textbf{}                                 & \textbf{dezembro de 2006} & \textbf{dezembro de 2017} & \textbf{Variação} \\ \hline
		\textbf{Crianças na fila}                      & 1448 & 176 & -87.85\% \\ \hline
		\textbf{População de 0 a 3 anos estimada}      & 5408 & 5885 & 8.82\% \\ \hline
		\textbf{Fila relativa à população}             & 9.01\% & 2.99\% & -66.81\% \\ \hline
	\end{tabular}
	\caption{Comparação da demanda no mês de dezembro, em 2006 e 2017.}
\end{table}
\section{Pinheiros}
\begin{figure}[H]
	\centering
	\includegraphics[width=0.66\linewidth]{../Analises/graficos/tempo/tempo_PINHEIROS}
	\includegraphics[width=0.33\linewidth]{../Analises/mapas/mapas/distrito_PIN}
	\caption{Evolução das matrículas, da demanda, e da população de 0 a 3 anos no distrito Pinheiros e a sua localização no município, respectivamente.}
\end{figure}
\begin{table}[H]
	\begin{tabular}{|l|l|l|l|}
		\hline
		\textbf{}                                      & \textbf{junho de 2006}       & \textbf{dezembro de 2017}    & \textbf{Variação} \\ \hline
		\textbf{Número de matrículas}                  & 251 & 381 & 51.79\% \\ \hline
		\textbf{Crianças na fila}                      & 230 & 120 & -47.83\% \\ \hline
		\textbf{População de 0 a 3 anos estimada}      & 2115 & 2445 & 15.6\% \\ \hline
		\textbf{Matrículas relativas à população}      & 11.87\% & 15.58\% & 31.31\% \\ \hline
		\textbf{Fila relativa à população}             & 10.87\% & 4.91\% & -54.87\% \\ \hline
		\textbf{Proporção das matrículas no município} & 0.41\% & 0.13\% & -68.37\% \\ \hline
		\textbf{Proporção da demanda no município}     & 0.27\% & 0.27\% & -0.12\% \\ \hline
		\textbf{Proporção da população de 0 a 3 anos}  & 0.34\% & 0.38\% & 12.63\% \\ \hline
		\textbf{Posição no \textit{ranking} das matrículas}     & 71 & 91 & -20 \\ \hline
		\textbf{Posição no \textit{ranking} da demanda}         & 79 & 73 & 6 \\ \hline
	\end{tabular}
	\caption{Comparação entre o primeiro e o último período da amostra}
\end{table}
\begin{table}[H]
	\begin{tabular}{|l|l|l|l|}
		\hline
		\textbf{}                                 & \textbf{dezembro de 2006} & \textbf{dezembro de 2017} & \textbf{Variação} \\ \hline
		\textbf{Crianças na fila}                      & 390 & 120 & -69.23\% \\ \hline
		\textbf{População de 0 a 3 anos estimada}      & 2115 & 2445 & 15.6\% \\ \hline
		\textbf{Fila relativa à população}             & 12.39\% & 4.91\% & -60.39\% \\ \hline
	\end{tabular}
	\caption{Comparação da demanda no mês de dezembro, em 2006 e 2017.}
\end{table}
\section{Pirituba}
\begin{figure}[H]
	\centering
	\includegraphics[width=0.66\linewidth]{../Analises/graficos/tempo/tempo_PIRITUBA}
	\includegraphics[width=0.33\linewidth]{../Analises/mapas/mapas/distrito_PIR}
	\caption{Evolução das matrículas, da demanda, e da população de 0 a 3 anos no distrito Pirituba e a sua localização no município, respectivamente.}
\end{figure}
\begin{table}[H]
	\begin{tabular}{|l|l|l|l|}
		\hline
		\textbf{}                                      & \textbf{junho de 2006}       & \textbf{dezembro de 2017}    & \textbf{Variação} \\ \hline
		\textbf{Número de matrículas}                  & 932 & 3712 & 298.28\% \\ \hline
		\textbf{Crianças na fila}                      & 955 & 370 & -61.26\% \\ \hline
		\textbf{População de 0 a 3 anos estimada}      & 9077 & 8880 & -2.17\% \\ \hline
		\textbf{Matrículas relativas à população}      & 10.27\% & 41.8\% & 307.12\% \\ \hline
		\textbf{Fila relativa à população}             & 10.52\% & 4.17\% & -60.4\% \\ \hline
		\textbf{Proporção das matrículas no município} & 1.51\% & 1.25\% & -17.01\% \\ \hline
		\textbf{Proporção da demanda no município}     & 1.13\% & 0.84\% & -25.83\% \\ \hline
		\textbf{Proporção da população de 0 a 3 anos}  & 1.45\% & 1.38\% & -4.68\% \\ \hline
		\textbf{Posição no \textit{ranking} das matrículas}     & 21 & 29 & -8 \\ \hline
		\textbf{Posição no \textit{ranking} da demanda}         & 29 & 30 & -1 \\ \hline
	\end{tabular}
	\caption{Comparação entre o primeiro e o último período da amostra}
\end{table}
\begin{table}[H]
	\begin{tabular}{|l|l|l|l|}
		\hline
		\textbf{}                                 & \textbf{dezembro de 2006} & \textbf{dezembro de 2017} & \textbf{Variação} \\ \hline
		\textbf{Crianças na fila}                      & 1669 & 370 & -77.83\% \\ \hline
		\textbf{População de 0 a 3 anos estimada}      & 9077 & 8880 & -2.17\% \\ \hline
		\textbf{Fila relativa à população}             & 9.74\% & 4.17\% & -57.22\% \\ \hline
	\end{tabular}
	\caption{Comparação da demanda no mês de dezembro, em 2006 e 2017.}
\end{table}
\section{Ponte Rasa}
\begin{figure}[H]
	\centering
	\includegraphics[width=0.66\linewidth]{../Analises/graficos/tempo/tempo_PONTE_RASA}
	\includegraphics[width=0.33\linewidth]{../Analises/mapas/mapas/distrito_PRA}
	\caption{Evolução das matrículas, da demanda, e da população de 0 a 3 anos no distrito Ponte Rasa e a sua localização no município, respectivamente.}
\end{figure}
\begin{table}[H]
	\begin{tabular}{|l|l|l|l|}
		\hline
		\textbf{}                                      & \textbf{junho de 2006}       & \textbf{dezembro de 2017}    & \textbf{Variação} \\ \hline
		\textbf{Número de matrículas}                  & 461 & 2063 & 347.51\% \\ \hline
		\textbf{Crianças na fila}                      & 605 & 32 & -94.71\% \\ \hline
		\textbf{População de 0 a 3 anos estimada}      & 5097 & 4443 & -12.83\% \\ \hline
		\textbf{Matrículas relativas à população}      & 9.04\% & 46.43\% & 413.38\% \\ \hline
		\textbf{Fila relativa à população}             & 11.87\% & 0.72\% & -93.93\% \\ \hline
		\textbf{Proporção das matrículas no município} & 0.75\% & 0.7\% & -6.76\% \\ \hline
		\textbf{Proporção da demanda no município}     & 0.72\% & 0.07\% & -89.87\% \\ \hline
		\textbf{Proporção da população de 0 a 3 anos}  & 0.81\% & 0.69\% & -15.07\% \\ \hline
		\textbf{Posição no \textit{ranking} das matrículas}     & 50 & 48 & 2 \\ \hline
		\textbf{Posição no \textit{ranking} da demanda}         & 46 & 92 & -46 \\ \hline
	\end{tabular}
	\caption{Comparação entre o primeiro e o último período da amostra}
\end{table}
\begin{table}[H]
	\begin{tabular}{|l|l|l|l|}
		\hline
		\textbf{}                                 & \textbf{dezembro de 2006} & \textbf{dezembro de 2017} & \textbf{Variação} \\ \hline
		\textbf{Crianças na fila}                      & 869 & 32 & -96.32\% \\ \hline
		\textbf{População de 0 a 3 anos estimada}      & 5097 & 4443 & -12.83\% \\ \hline
		\textbf{Fila relativa à população}             & 8.91\% & 0.72\% & -91.92\% \\ \hline
	\end{tabular}
	\caption{Comparação da demanda no mês de dezembro, em 2006 e 2017.}
\end{table}
\section{Raposo Tavares}
\begin{figure}[H]
	\centering
	\includegraphics[width=0.66\linewidth]{../Analises/graficos/tempo/tempo_RAPOSO_TAVARES}
	\includegraphics[width=0.33\linewidth]{../Analises/mapas/mapas/distrito_RTA}
	\caption{Evolução das matrículas, da demanda, e da população de 0 a 3 anos no distrito Raposo Tavares e a sua localização no município, respectivamente.}
\end{figure}
\begin{table}[H]
	\begin{tabular}{|l|l|l|l|}
		\hline
		\textbf{}                                      & \textbf{junho de 2006}       & \textbf{dezembro de 2017}    & \textbf{Variação} \\ \hline
		\textbf{Número de matrículas}                  & 927 & 3479 & 275.3\% \\ \hline
		\textbf{Crianças na fila}                      & 1089 & 548 & -49.68\% \\ \hline
		\textbf{População de 0 a 3 anos estimada}      & 6079 & 6713 & 10.43\% \\ \hline
		\textbf{Matrículas relativas à população}      & 15.25\% & 51.82\% & 239.85\% \\ \hline
		\textbf{Fila relativa à população}             & 17.91\% & 8.16\% & -54.43\% \\ \hline
		\textbf{Proporção das matrículas no município} & 1.5\% & 1.17\% & -21.8\% \\ \hline
		\textbf{Proporção da demanda no município}     & 1.29\% & 1.24\% & -3.67\% \\ \hline
		\textbf{Proporção da população de 0 a 3 anos}  & 0.97\% & 1.04\% & 7.59\% \\ \hline
		\textbf{Posição no \textit{ranking} das matrículas}     & 22 & 33 & -11 \\ \hline
		\textbf{Posição no \textit{ranking} da demanda}         & 27 & 21 & 6 \\ \hline
	\end{tabular}
	\caption{Comparação entre o primeiro e o último período da amostra}
\end{table}
\begin{table}[H]
	\begin{tabular}{|l|l|l|l|}
		\hline
		\textbf{}                                 & \textbf{dezembro de 2006} & \textbf{dezembro de 2017} & \textbf{Variação} \\ \hline
		\textbf{Crianças na fila}                      & 1869 & 548 & -70.68\% \\ \hline
		\textbf{População de 0 a 3 anos estimada}      & 6079 & 6713 & 10.43\% \\ \hline
		\textbf{Fila relativa à população}             & 15.41\% & 8.16\% & -47.03\% \\ \hline
	\end{tabular}
	\caption{Comparação da demanda no mês de dezembro, em 2006 e 2017.}
\end{table}
\section{República}
\begin{figure}[H]
	\centering
	\includegraphics[width=0.66\linewidth]{../Analises/graficos/tempo/tempo_REPUBLICA}
	\includegraphics[width=0.33\linewidth]{../Analises/mapas/mapas/distrito_REP}
	\caption{Evolução das matrículas, da demanda, e da população de 0 a 3 anos no distrito República e a sua localização no município, respectivamente.}
\end{figure}
\begin{table}[H]
	\begin{tabular}{|l|l|l|l|}
		\hline
		\textbf{}                                      & \textbf{junho de 2006}       & \textbf{dezembro de 2017}    & \textbf{Variação} \\ \hline
		\textbf{Número de matrículas}                  & 135 & 32 & -76.3\% \\ \hline
		\textbf{Crianças na fila}                      & 301 & 2 & -99.34\% \\ \hline
		\textbf{População de 0 a 3 anos estimada}      & 2015 & 2920 & 44.91\% \\ \hline
		\textbf{Matrículas relativas à população}      & 6.7\% & 1.1\% & -83.64\% \\ \hline
		\textbf{Fila relativa à população}             & 14.94\% & 0.07\% & -99.54\% \\ \hline
		\textbf{Proporção das matrículas no município} & 0.22\% & 0.01\% & -95.06\% \\ \hline
		\textbf{Proporção da demanda no município}     & 0.36\% & 0.0\% & -98.73\% \\ \hline
		\textbf{Proporção da população de 0 a 3 anos}  & 0.32\% & 0.45\% & 41.19\% \\ \hline
		\textbf{Posição no \textit{ranking} das matrículas}     & 87 & 96 & -9 \\ \hline
		\textbf{Posição no \textit{ranking} da demanda}         & 74 & 96 & -22 \\ \hline
	\end{tabular}
	\caption{Comparação entre o primeiro e o último período da amostra}
\end{table}
\begin{table}[H]
	\begin{tabular}{|l|l|l|l|}
		\hline
		\textbf{}                                 & \textbf{dezembro de 2006} & \textbf{dezembro de 2017} & \textbf{Variação} \\ \hline
		\textbf{Crianças na fila}                      & 4 & 2 & -50.0\% \\ \hline
		\textbf{População de 0 a 3 anos estimada}      & 2015 & 2920 & 44.91\% \\ \hline
		\textbf{Fila relativa à população}             & 0.45\% & 0.07\% & -84.78\% \\ \hline
	\end{tabular}
	\caption{Comparação da demanda no mês de dezembro, em 2006 e 2017.}
\end{table}
\section{Rio Pequeno}
\begin{figure}[H]
	\centering
	\includegraphics[width=0.66\linewidth]{../Analises/graficos/tempo/tempo_RIO_PEQUENO}
	\includegraphics[width=0.33\linewidth]{../Analises/mapas/mapas/distrito_RPE}
	\caption{Evolução das matrículas, da demanda, e da população de 0 a 3 anos no distrito Rio Pequeno e a sua localização no município, respectivamente.}
\end{figure}
\begin{table}[H]
	\begin{tabular}{|l|l|l|l|}
		\hline
		\textbf{}                                      & \textbf{junho de 2006}       & \textbf{dezembro de 2017}    & \textbf{Variação} \\ \hline
		\textbf{Número de matrículas}                  & 1226 & 3016 & 146.0\% \\ \hline
		\textbf{Crianças na fila}                      & 1140 & 221 & -80.61\% \\ \hline
		\textbf{População de 0 a 3 anos estimada}      & 6598 & 8021 & 21.57\% \\ \hline
		\textbf{Matrículas relativas à população}      & 18.58\% & 37.6\% & 102.36\% \\ \hline
		\textbf{Fila relativa à população}             & 17.28\% & 2.76\% & -84.05\% \\ \hline
		\textbf{Proporção das matrículas no município} & 1.99\% & 1.02\% & -48.74\% \\ \hline
		\textbf{Proporção da demanda no município}     & 1.35\% & 0.5\% & -62.89\% \\ \hline
		\textbf{Proporção da população de 0 a 3 anos}  & 1.05\% & 1.25\% & 18.44\% \\ \hline
		\textbf{Posição no \textit{ranking} das matrículas}     & 13 & 39 & -26 \\ \hline
		\textbf{Posição no \textit{ranking} da demanda}         & 24 & 46 & -22 \\ \hline
	\end{tabular}
	\caption{Comparação entre o primeiro e o último período da amostra}
\end{table}
\begin{table}[H]
	\begin{tabular}{|l|l|l|l|}
		\hline
		\textbf{}                                 & \textbf{dezembro de 2006} & \textbf{dezembro de 2017} & \textbf{Variação} \\ \hline
		\textbf{Crianças na fila}                      & 2012 & 221 & -89.02\% \\ \hline
		\textbf{População de 0 a 3 anos estimada}      & 6598 & 8021 & 21.57\% \\ \hline
		\textbf{Fila relativa à população}             & 18.9\% & 2.76\% & -85.42\% \\ \hline
	\end{tabular}
	\caption{Comparação da demanda no mês de dezembro, em 2006 e 2017.}
\end{table}
\section{Sacomã}
\begin{figure}[H]
	\centering
	\includegraphics[width=0.66\linewidth]{../Analises/graficos/tempo/tempo_SACOMA}
	\includegraphics[width=0.33\linewidth]{../Analises/mapas/mapas/distrito_SAC}
	\caption{Evolução das matrículas, da demanda, e da população de 0 a 3 anos no distrito Sacomã e a sua localização no município, respectivamente.}
\end{figure}
\begin{table}[H]
	\begin{tabular}{|l|l|l|l|}
		\hline
		\textbf{}                                      & \textbf{junho de 2006}       & \textbf{dezembro de 2017}    & \textbf{Variação} \\ \hline
		\textbf{Número de matrículas}                  & 1380 & 7199 & 421.67\% \\ \hline
		\textbf{Crianças na fila}                      & 2203 & 347 & -84.25\% \\ \hline
		\textbf{População de 0 a 3 anos estimada}      & 13625 & 13132 & -3.62\% \\ \hline
		\textbf{Matrículas relativas à população}      & 10.13\% & 54.82\% & 441.25\% \\ \hline
		\textbf{Fila relativa à população}             & 16.17\% & 2.64\% & -83.66\% \\ \hline
		\textbf{Proporção das matrículas no município} & 2.24\% & 2.43\% & 8.69\% \\ \hline
		\textbf{Proporção da demanda no município}     & 2.61\% & 0.79\% & -69.85\% \\ \hline
		\textbf{Proporção da população de 0 a 3 anos}  & 2.17\% & 2.04\% & -6.09\% \\ \hline
		\textbf{Posição no \textit{ranking} das matrículas}     & 10 & 11 & -1 \\ \hline
		\textbf{Posição no \textit{ranking} da demanda}         & 11 & 32 & -21 \\ \hline
	\end{tabular}
	\caption{Comparação entre o primeiro e o último período da amostra}
\end{table}
\begin{table}[H]
	\begin{tabular}{|l|l|l|l|}
		\hline
		\textbf{}                                 & \textbf{dezembro de 2006} & \textbf{dezembro de 2017} & \textbf{Variação} \\ \hline
		\textbf{Crianças na fila}                      & 2830 & 347 & -87.74\% \\ \hline
		\textbf{População de 0 a 3 anos estimada}      & 13625 & 13132 & -3.62\% \\ \hline
		\textbf{Fila relativa à população}             & 10.32\% & 2.64\% & -74.4\% \\ \hline
	\end{tabular}
	\caption{Comparação da demanda no mês de dezembro, em 2006 e 2017.}
\end{table}
\section{Santa Cecília}
\begin{figure}[H]
	\centering
	\includegraphics[width=0.66\linewidth]{../Analises/graficos/tempo/tempo_SANTA_CECILIA}
	\includegraphics[width=0.33\linewidth]{../Analises/mapas/mapas/distrito_SCE}
	\caption{Evolução das matrículas, da demanda, e da população de 0 a 3 anos no distrito Santa Cecília e a sua localização no município, respectivamente.}
\end{figure}
\begin{table}[H]
	\begin{tabular}{|l|l|l|l|}
		\hline
		\textbf{}                                      & \textbf{junho de 2006}       & \textbf{dezembro de 2017}    & \textbf{Variação} \\ \hline
		\textbf{Número de matrículas}                  & 357 & 1538 & 330.81\% \\ \hline
		\textbf{Crianças na fila}                      & 571 & 346 & -39.4\% \\ \hline
		\textbf{População de 0 a 3 anos estimada}      & 2960 & 3956 & 33.65\% \\ \hline
		\textbf{Matrículas relativas à população}      & 12.06\% & 38.88\% & 222.35\% \\ \hline
		\textbf{Fila relativa à população}             & 19.29\% & 8.75\% & -54.66\% \\ \hline
		\textbf{Proporção das matrículas no município} & 0.58\% & 0.52\% & -10.24\% \\ \hline
		\textbf{Proporção da demanda no município}     & 0.68\% & 0.78\% & 16.0\% \\ \hline
		\textbf{Proporção da população de 0 a 3 anos}  & 0.47\% & 0.61\% & 30.22\% \\ \hline
		\textbf{Posição no \textit{ranking} das matrículas}     & 59 & 58 & 1 \\ \hline
		\textbf{Posição no \textit{ranking} da demanda}         & 49 & 33 & 16 \\ \hline
	\end{tabular}
	\caption{Comparação entre o primeiro e o último período da amostra}
\end{table}
\begin{table}[H]
	\begin{tabular}{|l|l|l|l|}
		\hline
		\textbf{}                                 & \textbf{dezembro de 2006} & \textbf{dezembro de 2017} & \textbf{Variação} \\ \hline
		\textbf{Crianças na fila}                      & 757 & 346 & -54.29\% \\ \hline
		\textbf{População de 0 a 3 anos estimada}      & 2960 & 3956 & 33.65\% \\ \hline
		\textbf{Fila relativa à população}             & 12.03\% & 8.75\% & -27.3\% \\ \hline
	\end{tabular}
	\caption{Comparação da demanda no mês de dezembro, em 2006 e 2017.}
\end{table}
\section{Santana}
\begin{figure}[H]
	\centering
	\includegraphics[width=0.66\linewidth]{../Analises/graficos/tempo/tempo_SANTANA}
	\includegraphics[width=0.33\linewidth]{../Analises/mapas/mapas/distrito_STN}
	\caption{Evolução das matrículas, da demanda, e da população de 0 a 3 anos no distrito Santana e a sua localização no município, respectivamente.}
\end{figure}
\begin{table}[H]
	\begin{tabular}{|l|l|l|l|}
		\hline
		\textbf{}                                      & \textbf{junho de 2006}       & \textbf{dezembro de 2017}    & \textbf{Variação} \\ \hline
		\textbf{Número de matrículas}                  & 310 & 1263 & 307.42\% \\ \hline
		\textbf{Crianças na fila}                      & 425 & 165 & -61.18\% \\ \hline
		\textbf{População de 0 a 3 anos estimada}      & 4418 & 4932 & 11.63\% \\ \hline
		\textbf{Matrículas relativas à população}      & 7.02\% & 25.61\% & 264.96\% \\ \hline
		\textbf{Fila relativa à população}             & 9.62\% & 3.35\% & -65.22\% \\ \hline
		\textbf{Proporção das matrículas no município} & 0.5\% & 0.43\% & -15.11\% \\ \hline
		\textbf{Proporção da demanda no município}     & 0.5\% & 0.37\% & -25.68\% \\ \hline
		\textbf{Proporção da população de 0 a 3 anos}  & 0.7\% & 0.77\% & 8.77\% \\ \hline
		\textbf{Posição no \textit{ranking} das matrículas}     & 64 & 64 & 0 \\ \hline
		\textbf{Posição no \textit{ranking} da demanda}         & 64 & 61 & 3 \\ \hline
	\end{tabular}
	\caption{Comparação entre o primeiro e o último período da amostra}
\end{table}
\begin{table}[H]
	\begin{tabular}{|l|l|l|l|}
		\hline
		\textbf{}                                 & \textbf{dezembro de 2006} & \textbf{dezembro de 2017} & \textbf{Variação} \\ \hline
		\textbf{Crianças na fila}                      & 600 & 165 & -72.5\% \\ \hline
		\textbf{População de 0 a 3 anos estimada}      & 4418 & 4932 & 11.63\% \\ \hline
		\textbf{Fila relativa à população}             & 6.77\% & 3.35\% & -50.58\% \\ \hline
	\end{tabular}
	\caption{Comparação da demanda no mês de dezembro, em 2006 e 2017.}
\end{table}
\section{Santo Amaro}
\begin{figure}[H]
	\centering
	\includegraphics[width=0.66\linewidth]{../Analises/graficos/tempo/tempo_SANTO_AMARO}
	\includegraphics[width=0.33\linewidth]{../Analises/mapas/mapas/distrito_SAM}
	\caption{Evolução das matrículas, da demanda, e da população de 0 a 3 anos no distrito Santo Amaro e a sua localização no município, respectivamente.}
\end{figure}
\begin{table}[H]
	\begin{tabular}{|l|l|l|l|}
		\hline
		\textbf{}                                      & \textbf{junho de 2006}       & \textbf{dezembro de 2017}    & \textbf{Variação} \\ \hline
		\textbf{Número de matrículas}                  & 136 & 391 & 187.5\% \\ \hline
		\textbf{Crianças na fila}                      & 352 & 147 & -58.24\% \\ \hline
		\textbf{População de 0 a 3 anos estimada}      & 2412 & 2894 & 19.98\% \\ \hline
		\textbf{Matrículas relativas à população}      & 5.64\% & 13.51\% & 139.62\% \\ \hline
		\textbf{Fila relativa à população}             & 14.59\% & 5.08\% & -65.19\% \\ \hline
		\textbf{Proporção das matrículas no município} & 0.22\% & 0.13\% & -40.1\% \\ \hline
		\textbf{Proporção da demanda no município}     & 0.42\% & 0.33\% & -20.05\% \\ \hline
		\textbf{Proporção da população de 0 a 3 anos}  & 0.38\% & 0.45\% & 16.9\% \\ \hline
		\textbf{Posição no \textit{ranking} das matrículas}     & 86 & 90 & -4 \\ \hline
		\textbf{Posição no \textit{ranking} da demanda}         & 67 & 65 & 2 \\ \hline
	\end{tabular}
	\caption{Comparação entre o primeiro e o último período da amostra}
\end{table}
\begin{table}[H]
	\begin{tabular}{|l|l|l|l|}
		\hline
		\textbf{}                                 & \textbf{dezembro de 2006} & \textbf{dezembro de 2017} & \textbf{Variação} \\ \hline
		\textbf{Crianças na fila}                      & 420 & 147 & -65.0\% \\ \hline
		\textbf{População de 0 a 3 anos estimada}      & 2412 & 2894 & 19.98\% \\ \hline
		\textbf{Fila relativa à população}             & 5.68\% & 5.08\% & -10.57\% \\ \hline
	\end{tabular}
	\caption{Comparação da demanda no mês de dezembro, em 2006 e 2017.}
\end{table}
\section{São Domingos}
\begin{figure}[H]
	\centering
	\includegraphics[width=0.66\linewidth]{../Analises/graficos/tempo/tempo_SAO_DOMINGOS}
	\includegraphics[width=0.33\linewidth]{../Analises/mapas/mapas/distrito_SDO}
	\caption{Evolução das matrículas, da demanda, e da população de 0 a 3 anos no distrito São Domingos e a sua localização no município, respectivamente.}
\end{figure}
\begin{table}[H]
	\begin{tabular}{|l|l|l|l|}
		\hline
		\textbf{}                                      & \textbf{junho de 2006}       & \textbf{dezembro de 2017}    & \textbf{Variação} \\ \hline
		\textbf{Número de matrículas}                  & 479 & 1660 & 246.56\% \\ \hline
		\textbf{Crianças na fila}                      & 471 & 215 & -54.35\% \\ \hline
		\textbf{População de 0 a 3 anos estimada}      & 4673 & 4240 & -9.27\% \\ \hline
		\textbf{Matrículas relativas à população}      & 10.25\% & 39.15\% & 281.95\% \\ \hline
		\textbf{Fila relativa à população}             & 10.08\% & 5.07\% & -49.69\% \\ \hline
		\textbf{Proporção das matrículas no município} & 0.78\% & 0.56\% & -27.79\% \\ \hline
		\textbf{Proporção da demanda no município}     & 0.56\% & 0.49\% & -12.61\% \\ \hline
		\textbf{Proporção da população de 0 a 3 anos}  & 0.74\% & 0.66\% & -11.6\% \\ \hline
		\textbf{Posição no \textit{ranking} das matrículas}     & 48 & 54 & -6 \\ \hline
		\textbf{Posição no \textit{ranking} da demanda}         & 58 & 47 & 11 \\ \hline
	\end{tabular}
	\caption{Comparação entre o primeiro e o último período da amostra}
\end{table}
\begin{table}[H]
	\begin{tabular}{|l|l|l|l|}
		\hline
		\textbf{}                                 & \textbf{dezembro de 2006} & \textbf{dezembro de 2017} & \textbf{Variação} \\ \hline
		\textbf{Crianças na fila}                      & 766 & 215 & -71.93\% \\ \hline
		\textbf{População de 0 a 3 anos estimada}      & 4673 & 4240 & -9.27\% \\ \hline
		\textbf{Fila relativa à população}             & 9.95\% & 5.07\% & -49.04\% \\ \hline
	\end{tabular}
	\caption{Comparação da demanda no mês de dezembro, em 2006 e 2017.}
\end{table}
\section{São Lucas}
\begin{figure}[H]
	\centering
	\includegraphics[width=0.66\linewidth]{../Analises/graficos/tempo/tempo_SAO_LUCAS}
	\includegraphics[width=0.33\linewidth]{../Analises/mapas/mapas/distrito_SLU}
	\caption{Evolução das matrículas, da demanda, e da população de 0 a 3 anos no distrito São Lucas e a sua localização no município, respectivamente.}
\end{figure}
\begin{table}[H]
	\begin{tabular}{|l|l|l|l|}
		\hline
		\textbf{}                                      & \textbf{junho de 2006}       & \textbf{dezembro de 2017}    & \textbf{Variação} \\ \hline
		\textbf{Número de matrículas}                  & 366 & 1558 & 325.68\% \\ \hline
		\textbf{Crianças na fila}                      & 438 & 422 & -3.65\% \\ \hline
		\textbf{População de 0 a 3 anos estimada}      & 7104 & 6375 & -10.26\% \\ \hline
		\textbf{Matrículas relativas à população}      & 5.15\% & 24.44\% & 374.36\% \\ \hline
		\textbf{Fila relativa à população}             & 6.17\% & 6.62\% & 7.36\% \\ \hline
		\textbf{Proporção das matrículas no município} & 0.59\% & 0.53\% & -11.3\% \\ \hline
		\textbf{Proporção da demanda no município}     & 0.52\% & 0.96\% & 84.44\% \\ \hline
		\textbf{Proporção da população de 0 a 3 anos}  & 1.13\% & 0.99\% & -12.57\% \\ \hline
		\textbf{Posição no \textit{ranking} das matrículas}     & 58 & 57 & 1 \\ \hline
		\textbf{Posição no \textit{ranking} da demanda}         & 62 & 27 & 35 \\ \hline
	\end{tabular}
	\caption{Comparação entre o primeiro e o último período da amostra}
\end{table}
\begin{table}[H]
	\begin{tabular}{|l|l|l|l|}
		\hline
		\textbf{}                                 & \textbf{dezembro de 2006} & \textbf{dezembro de 2017} & \textbf{Variação} \\ \hline
		\textbf{Crianças na fila}                      & 496 & 422 & -14.92\% \\ \hline
		\textbf{População de 0 a 3 anos estimada}      & 7104 & 6375 & -10.26\% \\ \hline
		\textbf{Fila relativa à população}             & 5.1\% & 6.62\% & 29.8\% \\ \hline
	\end{tabular}
	\caption{Comparação da demanda no mês de dezembro, em 2006 e 2017.}
\end{table}
\section{São Mateus}
\begin{figure}[H]
	\centering
	\includegraphics[width=0.66\linewidth]{../Analises/graficos/tempo/tempo_SAO_MATEUS}
	\includegraphics[width=0.33\linewidth]{../Analises/mapas/mapas/distrito_SMT}
	\caption{Evolução das matrículas, da demanda, e da população de 0 a 3 anos no distrito São Mateus e a sua localização no município, respectivamente.}
\end{figure}
\begin{table}[H]
	\begin{tabular}{|l|l|l|l|}
		\hline
		\textbf{}                                      & \textbf{junho de 2006}       & \textbf{dezembro de 2017}    & \textbf{Variação} \\ \hline
		\textbf{Número de matrículas}                  & 989 & 4382 & 343.07\% \\ \hline
		\textbf{Crianças na fila}                      & 1562 & 574 & -63.25\% \\ \hline
		\textbf{População de 0 a 3 anos estimada}      & 9304 & 8849 & -4.89\% \\ \hline
		\textbf{Matrículas relativas à população}      & 10.63\% & 49.52\% & 365.86\% \\ \hline
		\textbf{Fila relativa à população}             & 16.79\% & 6.49\% & -61.36\% \\ \hline
		\textbf{Proporção das matrículas no município} & 1.6\% & 1.48\% & -7.68\% \\ \hline
		\textbf{Proporção da demanda no município}     & 1.85\% & 1.3\% & -29.65\% \\ \hline
		\textbf{Proporção da população de 0 a 3 anos}  & 1.48\% & 1.37\% & -7.33\% \\ \hline
		\textbf{Posição no \textit{ranking} das matrículas}     & 17 & 23 & -6 \\ \hline
		\textbf{Posição no \textit{ranking} da demanda}         & 16 & 20 & -4 \\ \hline
	\end{tabular}
	\caption{Comparação entre o primeiro e o último período da amostra}
\end{table}
\begin{table}[H]
	\begin{tabular}{|l|l|l|l|}
		\hline
		\textbf{}                                 & \textbf{dezembro de 2006} & \textbf{dezembro de 2017} & \textbf{Variação} \\ \hline
		\textbf{Crianças na fila}                      & 1638 & 574 & -64.96\% \\ \hline
		\textbf{População de 0 a 3 anos estimada}      & 9304 & 8849 & -4.89\% \\ \hline
		\textbf{Fila relativa à população}             & 10.9\% & 6.49\% & -40.49\% \\ \hline
	\end{tabular}
	\caption{Comparação da demanda no mês de dezembro, em 2006 e 2017.}
\end{table}
\section{São Miguel}
\begin{figure}[H]
	\centering
	\includegraphics[width=0.66\linewidth]{../Analises/graficos/tempo/tempo_SAO_MIGUEL}
	\includegraphics[width=0.33\linewidth]{../Analises/mapas/mapas/distrito_SMI}
	\caption{Evolução das matrículas, da demanda, e da população de 0 a 3 anos no distrito São Miguel e a sua localização no município, respectivamente.}
\end{figure}
\begin{table}[H]
	\begin{tabular}{|l|l|l|l|}
		\hline
		\textbf{}                                      & \textbf{junho de 2006}       & \textbf{dezembro de 2017}    & \textbf{Variação} \\ \hline
		\textbf{Número de matrículas}                  & 621 & 2662 & 328.66\% \\ \hline
		\textbf{Crianças na fila}                      & 560 & 173 & -69.11\% \\ \hline
		\textbf{População de 0 a 3 anos estimada}      & 5652 & 5949 & 5.25\% \\ \hline
		\textbf{Matrículas relativas à população}      & 10.99\% & 44.75\% & 307.26\% \\ \hline
		\textbf{Fila relativa à população}             & 9.91\% & 2.91\% & -70.65\% \\ \hline
		\textbf{Proporção das matrículas no município} & 1.01\% & 0.9\% & -10.68\% \\ \hline
		\textbf{Proporção da demanda no município}     & 0.66\% & 0.39\% & -40.86\% \\ \hline
		\textbf{Proporção da população de 0 a 3 anos}  & 0.9\% & 0.92\% & 2.55\% \\ \hline
		\textbf{Posição no \textit{ranking} das matrículas}     & 37 & 42 & -5 \\ \hline
		\textbf{Posição no \textit{ranking} da demanda}         & 51 & 56 & -5 \\ \hline
	\end{tabular}
	\caption{Comparação entre o primeiro e o último período da amostra}
\end{table}
\begin{table}[H]
	\begin{tabular}{|l|l|l|l|}
		\hline
		\textbf{}                                 & \textbf{dezembro de 2006} & \textbf{dezembro de 2017} & \textbf{Variação} \\ \hline
		\textbf{Crianças na fila}                      & 1309 & 173 & -86.78\% \\ \hline
		\textbf{População de 0 a 3 anos estimada}      & 5652 & 5949 & 5.25\% \\ \hline
		\textbf{Fila relativa à população}             & 13.13\% & 2.91\% & -77.85\% \\ \hline
	\end{tabular}
	\caption{Comparação da demanda no mês de dezembro, em 2006 e 2017.}
\end{table}
\section{São Rafael}
\begin{figure}[H]
	\centering
	\includegraphics[width=0.66\linewidth]{../Analises/graficos/tempo/tempo_SAO_RAFAEL}
	\includegraphics[width=0.33\linewidth]{../Analises/mapas/mapas/distrito_SRA}
	\caption{Evolução das matrículas, da demanda, e da população de 0 a 3 anos no distrito São Rafael e a sua localização no município, respectivamente.}
\end{figure}
\begin{table}[H]
	\begin{tabular}{|l|l|l|l|}
		\hline
		\textbf{}                                      & \textbf{junho de 2006}       & \textbf{dezembro de 2017}    & \textbf{Variação} \\ \hline
		\textbf{Número de matrículas}                  & 1414 & 4023 & 184.51\% \\ \hline
		\textbf{Crianças na fila}                      & 2420 & 1412 & -41.65\% \\ \hline
		\textbf{População de 0 a 3 anos estimada}      & 9704 & 8890 & -8.39\% \\ \hline
		\textbf{Matrículas relativas à população}      & 14.57\% & 45.25\% & 210.56\% \\ \hline
		\textbf{Fila relativa à população}             & 24.94\% & 15.88\% & -36.31\% \\ \hline
		\textbf{Proporção das matrículas no município} & 2.29\% & 1.36\% & -40.72\% \\ \hline
		\textbf{Proporção da demanda no município}     & 2.87\% & 3.2\% & 11.7\% \\ \hline
		\textbf{Proporção da população de 0 a 3 anos}  & 1.55\% & 1.38\% & -10.74\% \\ \hline
		\textbf{Posição no \textit{ranking} das matrículas}     & 9 & 27 & -18 \\ \hline
		\textbf{Posição no \textit{ranking} da demanda}         & 7 & 10 & -3 \\ \hline
	\end{tabular}
	\caption{Comparação entre o primeiro e o último período da amostra}
\end{table}
\begin{table}[H]
	\begin{tabular}{|l|l|l|l|}
		\hline
		\textbf{}                                 & \textbf{dezembro de 2006} & \textbf{dezembro de 2017} & \textbf{Variação} \\ \hline
		\textbf{Crianças na fila}                      & 2752 & 1412 & -48.69\% \\ \hline
		\textbf{População de 0 a 3 anos estimada}      & 9704 & 8890 & -8.39\% \\ \hline
		\textbf{Fila relativa à população}             & 13.36\% & 15.88\% & 18.88\% \\ \hline
	\end{tabular}
	\caption{Comparação da demanda no mês de dezembro, em 2006 e 2017.}
\end{table}
\section{Sapopemba}
\begin{figure}[H]
	\centering
	\includegraphics[width=0.66\linewidth]{../Analises/graficos/tempo/tempo_SAPOPEMBA}
	\includegraphics[width=0.33\linewidth]{../Analises/mapas/mapas/distrito_SAP}
	\caption{Evolução das matrículas, da demanda, e da população de 0 a 3 anos no distrito Sapopemba e a sua localização no município, respectivamente.}
\end{figure}
\begin{table}[H]
	\begin{tabular}{|l|l|l|l|}
		\hline
		\textbf{}                                      & \textbf{junho de 2006}       & \textbf{dezembro de 2017}    & \textbf{Variação} \\ \hline
		\textbf{Número de matrículas}                  & 2133 & 6939 & 225.32\% \\ \hline
		\textbf{Crianças na fila}                      & 2781 & 1910 & -31.32\% \\ \hline
		\textbf{População de 0 a 3 anos estimada}      & 17481 & 16380 & -6.3\% \\ \hline
		\textbf{Matrículas relativas à população}      & 12.2\% & 42.36\% & 247.18\% \\ \hline
		\textbf{Fila relativa à população}             & 15.91\% & 11.66\% & -26.7\% \\ \hline
		\textbf{Proporção das matrículas no município} & 3.46\% & 2.34\% & -32.22\% \\ \hline
		\textbf{Proporção da demanda no município}     & 3.29\% & 4.33\% & 31.48\% \\ \hline
		\textbf{Proporção da população de 0 a 3 anos}  & 2.79\% & 2.54\% & -8.71\% \\ \hline
		\textbf{Posição no \textit{ranking} das matrículas}     & 3 & 13 & -10 \\ \hline
		\textbf{Posição no \textit{ranking} da demanda}         & 6 & 6 & 0 \\ \hline
	\end{tabular}
	\caption{Comparação entre o primeiro e o último período da amostra}
\end{table}
\begin{table}[H]
	\begin{tabular}{|l|l|l|l|}
		\hline
		\textbf{}                                 & \textbf{dezembro de 2006} & \textbf{dezembro de 2017} & \textbf{Variação} \\ \hline
		\textbf{Crianças na fila}                      & 3322 & 1910 & -42.5\% \\ \hline
		\textbf{População de 0 a 3 anos estimada}      & 17481 & 16380 & -6.3\% \\ \hline
		\textbf{Fila relativa à população}             & 12.06\% & 11.66\% & -3.31\% \\ \hline
	\end{tabular}
	\caption{Comparação da demanda no mês de dezembro, em 2006 e 2017.}
\end{table}
\section{Saúde}
\begin{figure}[H]
	\centering
	\includegraphics[width=0.66\linewidth]{../Analises/graficos/tempo/tempo_SAUDE}
	\includegraphics[width=0.33\linewidth]{../Analises/mapas/mapas/distrito_SAU}
	\caption{Evolução das matrículas, da demanda, e da população de 0 a 3 anos no distrito Saúde e a sua localização no município, respectivamente.}
\end{figure}
\begin{table}[H]
	\begin{tabular}{|l|l|l|l|}
		\hline
		\textbf{}                                      & \textbf{junho de 2006}       & \textbf{dezembro de 2017}    & \textbf{Variação} \\ \hline
		\textbf{Número de matrículas}                  & 189 & 1154 & 510.58\% \\ \hline
		\textbf{Crianças na fila}                      & 238 & 200 & -15.97\% \\ \hline
		\textbf{População de 0 a 3 anos estimada}      & 4590 & 5242 & 14.2\% \\ \hline
		\textbf{Matrículas relativas à população}      & 4.12\% & 22.01\% & 434.64\% \\ \hline
		\textbf{Fila relativa à população}             & 5.19\% & 3.82\% & -26.42\% \\ \hline
		\textbf{Proporção das matrículas no município} & 0.31\% & 0.39\% & 27.22\% \\ \hline
		\textbf{Proporção da demanda no município}     & 0.28\% & 0.45\% & 60.87\% \\ \hline
		\textbf{Proporção da população de 0 a 3 anos}  & 0.73\% & 0.81\% & 11.27\% \\ \hline
		\textbf{Posição no \textit{ranking} das matrículas}     & 81 & 66 & 15 \\ \hline
		\textbf{Posição no \textit{ranking} da demanda}         & 77 & 50 & 27 \\ \hline
	\end{tabular}
	\caption{Comparação entre o primeiro e o último período da amostra}
\end{table}
\begin{table}[H]
	\begin{tabular}{|l|l|l|l|}
		\hline
		\textbf{}                                 & \textbf{dezembro de 2006} & \textbf{dezembro de 2017} & \textbf{Variação} \\ \hline
		\textbf{Crianças na fila}                      & 287 & 200 & -30.31\% \\ \hline
		\textbf{População de 0 a 3 anos estimada}      & 4590 & 5242 & 14.2\% \\ \hline
		\textbf{Fila relativa à população}             & 4.14\% & 3.82\% & -7.84\% \\ \hline
	\end{tabular}
	\caption{Comparação da demanda no mês de dezembro, em 2006 e 2017.}
\end{table}
\section{Sé}
\begin{figure}[H]
	\centering
	\includegraphics[width=0.66\linewidth]{../Analises/graficos/tempo/tempo_SE}
	\includegraphics[width=0.33\linewidth]{../Analises/mapas/mapas/distrito_SEE}
	\caption{Evolução das matrículas, da demanda, e da população de 0 a 3 anos no distrito Sé e a sua localização no município, respectivamente.}
\end{figure}
\begin{table}[H]
	\begin{tabular}{|l|l|l|l|}
		\hline
		\textbf{}                                      & \textbf{junho de 2006}       & \textbf{dezembro de 2017}    & \textbf{Variação} \\ \hline
		\textbf{Número de matrículas}                  & 49 & 92 & 87.76\% \\ \hline
		\textbf{Crianças na fila}                      & 141 & 117 & -17.02\% \\ \hline
		\textbf{População de 0 a 3 anos estimada}      & 1031 & 1892 & 83.51\% \\ \hline
		\textbf{Matrículas relativas à população}      & 4.75\% & 4.86\% & 2.31\% \\ \hline
		\textbf{Fila relativa à população}             & 13.68\% & 6.18\% & -54.78\% \\ \hline
		\textbf{Proporção das matrículas no município} & 0.08\% & 0.03\% & -60.88\% \\ \hline
		\textbf{Proporção da demanda no município}     & 0.17\% & 0.27\% & 58.85\% \\ \hline
		\textbf{Proporção da população de 0 a 3 anos}  & 0.16\% & 0.29\% & 78.8\% \\ \hline
		\textbf{Posição no \textit{ranking} das matrículas}     & 95 & 95 & 0 \\ \hline
		\textbf{Posição no \textit{ranking} da demanda}         & 88 & 77 & 11 \\ \hline
	\end{tabular}
	\caption{Comparação entre o primeiro e o último período da amostra}
\end{table}
\begin{table}[H]
	\begin{tabular}{|l|l|l|l|}
		\hline
		\textbf{}                                 & \textbf{dezembro de 2006} & \textbf{dezembro de 2017} & \textbf{Variação} \\ \hline
		\textbf{Crianças na fila}                      & 172 & 117 & -31.98\% \\ \hline
		\textbf{População de 0 a 3 anos estimada}      & 1031 & 1892 & 83.51\% \\ \hline
		\textbf{Fila relativa à população}             & 5.24\% & 6.18\% & 18.01\% \\ \hline
	\end{tabular}
	\caption{Comparação da demanda no mês de dezembro, em 2006 e 2017.}
\end{table}
\section{Socorro}
\begin{figure}[H]
	\centering
	\includegraphics[width=0.66\linewidth]{../Analises/graficos/tempo/tempo_SOCORRO}
	\includegraphics[width=0.33\linewidth]{../Analises/mapas/mapas/distrito_SOC}
	\caption{Evolução das matrículas, da demanda, e da população de 0 a 3 anos no distrito Socorro e a sua localização no município, respectivamente.}
\end{figure}
\begin{table}[H]
	\begin{tabular}{|l|l|l|l|}
		\hline
		\textbf{}                                      & \textbf{junho de 2006}       & \textbf{dezembro de 2017}    & \textbf{Variação} \\ \hline
		\textbf{Número de matrículas}                  & 327 & 546 & 66.97\% \\ \hline
		\textbf{Crianças na fila}                      & 160 & 88 & -45.0\% \\ \hline
		\textbf{População de 0 a 3 anos estimada}      & 1505 & 1655 & 9.97\% \\ \hline
		\textbf{Matrículas relativas à população}      & 21.73\% & 32.99\% & 51.84\% \\ \hline
		\textbf{Fila relativa à população}             & 10.63\% & 5.32\% & -49.98\% \\ \hline
		\textbf{Proporção das matrículas no município} & 0.53\% & 0.18\% & -65.21\% \\ \hline
		\textbf{Proporção da demanda no município}     & 0.19\% & 0.2\% & 5.29\% \\ \hline
		\textbf{Proporção da população de 0 a 3 anos}  & 0.24\% & 0.26\% & 7.14\% \\ \hline
		\textbf{Posição no \textit{ranking} das matrículas}     & 62 & 80 & -18 \\ \hline
		\textbf{Posição no \textit{ranking} da demanda}         & 86 & 84 & 2 \\ \hline
	\end{tabular}
	\caption{Comparação entre o primeiro e o último período da amostra}
\end{table}
\begin{table}[H]
	\begin{tabular}{|l|l|l|l|}
		\hline
		\textbf{}                                 & \textbf{dezembro de 2006} & \textbf{dezembro de 2017} & \textbf{Variação} \\ \hline
		\textbf{Crianças na fila}                      & 218 & 88 & -59.63\% \\ \hline
		\textbf{População de 0 a 3 anos estimada}      & 1505 & 1655 & 9.97\% \\ \hline
		\textbf{Fila relativa à população}             & 22.13\% & 5.32\% & -75.97\% \\ \hline
	\end{tabular}
	\caption{Comparação da demanda no mês de dezembro, em 2006 e 2017.}
\end{table}
\section{Tatuapé}
\begin{figure}[H]
	\centering
	\includegraphics[width=0.66\linewidth]{../Analises/graficos/tempo/tempo_TATUAPE}
	\includegraphics[width=0.33\linewidth]{../Analises/mapas/mapas/distrito_TAT}
	\caption{Evolução das matrículas, da demanda, e da população de 0 a 3 anos no distrito Tatuapé e a sua localização no município, respectivamente.}
\end{figure}
\begin{table}[H]
	\begin{tabular}{|l|l|l|l|}
		\hline
		\textbf{}                                      & \textbf{junho de 2006}       & \textbf{dezembro de 2017}    & \textbf{Variação} \\ \hline
		\textbf{Número de matrículas}                  & 73 & 675 & 824.66\% \\ \hline
		\textbf{Crianças na fila}                      & 162 & 126 & -22.22\% \\ \hline
		\textbf{População de 0 a 3 anos estimada}      & 3254 & 3814 & 17.21\% \\ \hline
		\textbf{Matrículas relativas à população}      & 2.24\% & 17.7\% & 688.89\% \\ \hline
		\textbf{Fila relativa à população}             & 4.98\% & 3.3\% & -33.64\% \\ \hline
		\textbf{Proporção das matrículas no município} & 0.12\% & 0.23\% & 92.66\% \\ \hline
		\textbf{Proporção da demanda no município}     & 0.19\% & 0.29\% & 48.89\% \\ \hline
		\textbf{Proporção da população de 0 a 3 anos}  & 0.52\% & 0.59\% & 14.2\% \\ \hline
		\textbf{Posição no \textit{ranking} das matrículas}     & 92 & 76 & 16 \\ \hline
		\textbf{Posição no \textit{ranking} da demanda}         & 85 & 70 & 15 \\ \hline
	\end{tabular}
	\caption{Comparação entre o primeiro e o último período da amostra}
\end{table}
\begin{table}[H]
	\begin{tabular}{|l|l|l|l|}
		\hline
		\textbf{}                                 & \textbf{dezembro de 2006} & \textbf{dezembro de 2017} & \textbf{Variação} \\ \hline
		\textbf{Crianças na fila}                      & 308 & 126 & -59.09\% \\ \hline
		\textbf{População de 0 a 3 anos estimada}      & 3254 & 3814 & 17.21\% \\ \hline
		\textbf{Fila relativa à população}             & 2.21\% & 3.3\% & 49.48\% \\ \hline
	\end{tabular}
	\caption{Comparação da demanda no mês de dezembro, em 2006 e 2017.}
\end{table}
\section{Tremembé}
\begin{figure}[H]
	\centering
	\includegraphics[width=0.66\linewidth]{../Analises/graficos/tempo/tempo_TREMEMBE}
	\includegraphics[width=0.33\linewidth]{../Analises/mapas/mapas/distrito_TRE}
	\caption{Evolução das matrículas, da demanda, e da população de 0 a 3 anos no distrito Tremembé e a sua localização no município, respectivamente.}
\end{figure}
\begin{table}[H]
	\begin{tabular}{|l|l|l|l|}
		\hline
		\textbf{}                                      & \textbf{junho de 2006}       & \textbf{dezembro de 2017}    & \textbf{Variação} \\ \hline
		\textbf{Número de matrículas}                  & 616 & 4078 & 562.01\% \\ \hline
		\textbf{Crianças na fila}                      & 1175 & 1503 & 27.91\% \\ \hline
		\textbf{População de 0 a 3 anos estimada}      & 11490 & 13139 & 14.35\% \\ \hline
		\textbf{Matrículas relativas à população}      & 5.36\% & 31.04\% & 478.93\% \\ \hline
		\textbf{Fila relativa à população}             & 10.23\% & 11.44\% & 11.86\% \\ \hline
		\textbf{Proporção das matrículas no município} & 1.0\% & 1.38\% & 37.94\% \\ \hline
		\textbf{Proporção da demanda no município}     & 1.39\% & 3.41\% & 144.88\% \\ \hline
		\textbf{Proporção da população de 0 a 3 anos}  & 1.83\% & 2.04\% & 11.41\% \\ \hline
		\textbf{Posição no \textit{ranking} das matrículas}     & 38 & 26 & 12 \\ \hline
		\textbf{Posição no \textit{ranking} da demanda}         & 22 & 9 & 13 \\ \hline
	\end{tabular}
	\caption{Comparação entre o primeiro e o último período da amostra}
\end{table}
\begin{table}[H]
	\begin{tabular}{|l|l|l|l|}
		\hline
		\textbf{}                                 & \textbf{dezembro de 2006} & \textbf{dezembro de 2017} & \textbf{Variação} \\ \hline
		\textbf{Crianças na fila}                      & 1806 & 1503 & -16.78\% \\ \hline
		\textbf{População de 0 a 3 anos estimada}      & 11490 & 13139 & 14.35\% \\ \hline
		\textbf{Fila relativa à população}             & 5.33\% & 11.44\% & 114.62\% \\ \hline
	\end{tabular}
	\caption{Comparação da demanda no mês de dezembro, em 2006 e 2017.}
\end{table}
\section{Tucuruvi}
\begin{figure}[H]
	\centering
	\includegraphics[width=0.66\linewidth]{../Analises/graficos/tempo/tempo_TUCURUVI}
	\includegraphics[width=0.33\linewidth]{../Analises/mapas/mapas/distrito_TUC}
	\caption{Evolução das matrículas, da demanda, e da população de 0 a 3 anos no distrito Tucuruvi e a sua localização no município, respectivamente.}
\end{figure}
\begin{table}[H]
	\begin{tabular}{|l|l|l|l|}
		\hline
		\textbf{}                                      & \textbf{junho de 2006}       & \textbf{dezembro de 2017}    & \textbf{Variação} \\ \hline
		\textbf{Número de matrículas}                  & 291 & 1154 & 296.56\% \\ \hline
		\textbf{Crianças na fila}                      & 212 & 106 & -50.0\% \\ \hline
		\textbf{População de 0 a 3 anos estimada}      & 4007 & 4020 & 0.32\% \\ \hline
		\textbf{Matrículas relativas à população}      & 7.26\% & 28.71\% & 295.28\% \\ \hline
		\textbf{Fila relativa à população}             & 5.29\% & 2.64\% & -50.16\% \\ \hline
		\textbf{Proporção das matrículas no município} & 0.47\% & 0.39\% & -17.37\% \\ \hline
		\textbf{Proporção da demanda no município}     & 0.25\% & 0.24\% & -4.28\% \\ \hline
		\textbf{Proporção da população de 0 a 3 anos}  & 0.64\% & 0.62\% & -2.25\% \\ \hline
		\textbf{Posição no \textit{ranking} das matrículas}     & 65 & 65 & 0 \\ \hline
		\textbf{Posição no \textit{ranking} da demanda}         & 80 & 80 & 0 \\ \hline
	\end{tabular}
	\caption{Comparação entre o primeiro e o último período da amostra}
\end{table}
\begin{table}[H]
	\begin{tabular}{|l|l|l|l|}
		\hline
		\textbf{}                                 & \textbf{dezembro de 2006} & \textbf{dezembro de 2017} & \textbf{Variação} \\ \hline
		\textbf{Crianças na fila}                      & 404 & 106 & -73.76\% \\ \hline
		\textbf{População de 0 a 3 anos estimada}      & 4007 & 4020 & 0.32\% \\ \hline
		\textbf{Fila relativa à população}             & 7.34\% & 2.64\% & -64.08\% \\ \hline
	\end{tabular}
	\caption{Comparação da demanda no mês de dezembro, em 2006 e 2017.}
\end{table}
\section{Vila Andrade}
\begin{figure}[H]
	\centering
	\includegraphics[width=0.66\linewidth]{../Analises/graficos/tempo/tempo_VILA_ANDRADE}
	\includegraphics[width=0.33\linewidth]{../Analises/mapas/mapas/distrito_VAN}
	\caption{Evolução das matrículas, da demanda, e da população de 0 a 3 anos no distrito Vila Andrade e a sua localização no município, respectivamente.}
\end{figure}
\begin{table}[H]
	\begin{tabular}{|l|l|l|l|}
		\hline
		\textbf{}                                      & \textbf{junho de 2006}       & \textbf{dezembro de 2017}    & \textbf{Variação} \\ \hline
		\textbf{Número de matrículas}                  & 119 & 1473 & 1137.82\% \\ \hline
		\textbf{Crianças na fila}                      & 348 & 1285 & 269.25\% \\ \hline
		\textbf{População de 0 a 3 anos estimada}      & 7740 & 8400 & 8.53\% \\ \hline
		\textbf{Matrículas relativas à população}      & 1.54\% & 17.54\% & 1040.56\% \\ \hline
		\textbf{Fila relativa à população}             & 4.5\% & 15.3\% & 240.24\% \\ \hline
		\textbf{Proporção das matrículas no município} & 0.19\% & 0.5\% & 157.91\% \\ \hline
		\textbf{Proporção da demanda no município}     & 0.41\% & 2.91\% & 606.88\% \\ \hline
		\textbf{Proporção da população de 0 a 3 anos}  & 1.23\% & 1.3\% & 5.74\% \\ \hline
		\textbf{Posição no \textit{ranking} das matrículas}     & 90 & 61 & 29 \\ \hline
		\textbf{Posição no \textit{ranking} da demanda}         & 68 & 11 & 57 \\ \hline
	\end{tabular}
	\caption{Comparação entre o primeiro e o último período da amostra}
\end{table}
\begin{table}[H]
	\begin{tabular}{|l|l|l|l|}
		\hline
		\textbf{}                                 & \textbf{dezembro de 2006} & \textbf{dezembro de 2017} & \textbf{Variação} \\ \hline
		\textbf{Crianças na fila}                      & 1133 & 1285 & 13.42\% \\ \hline
		\textbf{População de 0 a 3 anos estimada}      & 7740 & 8400 & 8.53\% \\ \hline
		\textbf{Fila relativa à população}             & 2.12\% & 15.3\% & 621.59\% \\ \hline
	\end{tabular}
	\caption{Comparação da demanda no mês de dezembro, em 2006 e 2017.}
\end{table}
\section{Vila Curuçá}
\begin{figure}[H]
	\centering
	\includegraphics[width=0.66\linewidth]{../Analises/graficos/tempo/tempo_VILA_CURUCA}
	\includegraphics[width=0.33\linewidth]{../Analises/mapas/mapas/distrito_VCR}
	\caption{Evolução das matrículas, da demanda, e da população de 0 a 3 anos no distrito Vila Curuçá e a sua localização no município, respectivamente.}
\end{figure}
\begin{table}[H]
	\begin{tabular}{|l|l|l|l|}
		\hline
		\textbf{}                                      & \textbf{junho de 2006}       & \textbf{dezembro de 2017}    & \textbf{Variação} \\ \hline
		\textbf{Número de matrículas}                  & 1136 & 5349 & 370.86\% \\ \hline
		\textbf{Crianças na fila}                      & 1134 & 204 & -82.01\% \\ \hline
		\textbf{População de 0 a 3 anos estimada}      & 9949 & 8772 & -11.83\% \\ \hline
		\textbf{Matrículas relativas à população}      & 11.42\% & 60.98\% & 434.04\% \\ \hline
		\textbf{Fila relativa à população}             & 11.4\% & 2.33\% & -79.6\% \\ \hline
		\textbf{Proporção das matrículas no município} & 1.84\% & 1.81\% & -1.89\% \\ \hline
		\textbf{Proporção da demanda no município}     & 1.34\% & 0.46\% & -65.56\% \\ \hline
		\textbf{Proporção da população de 0 a 3 anos}  & 1.59\% & 1.36\% & -14.1\% \\ \hline
		\textbf{Posição no \textit{ranking} das matrículas}     & 14 & 19 & -5 \\ \hline
		\textbf{Posição no \textit{ranking} da demanda}         & 25 & 49 & -24 \\ \hline
	\end{tabular}
	\caption{Comparação entre o primeiro e o último período da amostra}
\end{table}
\begin{table}[H]
	\begin{tabular}{|l|l|l|l|}
		\hline
		\textbf{}                                 & \textbf{dezembro de 2006} & \textbf{dezembro de 2017} & \textbf{Variação} \\ \hline
		\textbf{Crianças na fila}                      & 2305 & 204 & -91.15\% \\ \hline
		\textbf{População de 0 a 3 anos estimada}      & 9949 & 8772 & -11.83\% \\ \hline
		\textbf{Fila relativa à população}             & 11.6\% & 2.33\% & -79.95\% \\ \hline
	\end{tabular}
	\caption{Comparação da demanda no mês de dezembro, em 2006 e 2017.}
\end{table}
\section{Vila Formosa}
\begin{figure}[H]
	\centering
	\includegraphics[width=0.66\linewidth]{../Analises/graficos/tempo/tempo_VILA_FORMOSA}
	\includegraphics[width=0.33\linewidth]{../Analises/mapas/mapas/distrito_VFO}
	\caption{Evolução das matrículas, da demanda, e da população de 0 a 3 anos no distrito Vila Formosa e a sua localização no município, respectivamente.}
\end{figure}
\begin{table}[H]
	\begin{tabular}{|l|l|l|l|}
		\hline
		\textbf{}                                      & \textbf{junho de 2006}       & \textbf{dezembro de 2017}    & \textbf{Variação} \\ \hline
		\textbf{Número de matrículas}                  & 258 & 1691 & 555.43\% \\ \hline
		\textbf{Crianças na fila}                      & 305 & 211 & -30.82\% \\ \hline
		\textbf{População de 0 a 3 anos estimada}      & 4279 & 4098 & -4.23\% \\ \hline
		\textbf{Matrículas relativas à população}      & 6.03\% & 41.26\% & 584.38\% \\ \hline
		\textbf{Fila relativa à população}             & 7.13\% & 5.15\% & -27.76\% \\ \hline
		\textbf{Proporção das matrículas no município} & 0.42\% & 0.57\% & 36.57\% \\ \hline
		\textbf{Proporção da demanda no município}     & 0.36\% & 0.48\% & 32.44\% \\ \hline
		\textbf{Proporção da população de 0 a 3 anos}  & 0.68\% & 0.64\% & -6.69\% \\ \hline
		\textbf{Posição no \textit{ranking} das matrículas}     & 69 & 52 & 17 \\ \hline
		\textbf{Posição no \textit{ranking} da demanda}         & 72 & 48 & 24 \\ \hline
	\end{tabular}
	\caption{Comparação entre o primeiro e o último período da amostra}
\end{table}
\begin{table}[H]
	\begin{tabular}{|l|l|l|l|}
		\hline
		\textbf{}                                 & \textbf{dezembro de 2006} & \textbf{dezembro de 2017} & \textbf{Variação} \\ \hline
		\textbf{Crianças na fila}                      & 733 & 211 & -71.21\% \\ \hline
		\textbf{População de 0 a 3 anos estimada}      & 4279 & 4098 & -4.23\% \\ \hline
		\textbf{Fila relativa à população}             & 6.01\% & 5.15\% & -14.33\% \\ \hline
	\end{tabular}
	\caption{Comparação da demanda no mês de dezembro, em 2006 e 2017.}
\end{table}
\section{Vila Guilherme}
\begin{figure}[H]
	\centering
	\includegraphics[width=0.66\linewidth]{../Analises/graficos/tempo/tempo_VILA_GUILHERME}
	\includegraphics[width=0.33\linewidth]{../Analises/mapas/mapas/distrito_VGL}
	\caption{Evolução das matrículas, da demanda, e da população de 0 a 3 anos no distrito Vila Guilherme e a sua localização no município, respectivamente.}
\end{figure}
\begin{table}[H]
	\begin{tabular}{|l|l|l|l|}
		\hline
		\textbf{}                                      & \textbf{junho de 2006}       & \textbf{dezembro de 2017}    & \textbf{Variação} \\ \hline
		\textbf{Número de matrículas}                  & 176 & 933 & 430.11\% \\ \hline
		\textbf{Crianças na fila}                      & 177 & 141 & -20.34\% \\ \hline
		\textbf{População de 0 a 3 anos estimada}      & 2335 & 3099 & 32.72\% \\ \hline
		\textbf{Matrículas relativas à população}      & 7.54\% & 30.11\% & 299.42\% \\ \hline
		\textbf{Fila relativa à população}             & 7.58\% & 4.55\% & -39.98\% \\ \hline
		\textbf{Proporção das matrículas no município} & 0.29\% & 0.31\% & 10.45\% \\ \hline
		\textbf{Proporção da demanda no município}     & 0.21\% & 0.32\% & 52.5\% \\ \hline
		\textbf{Proporção da população de 0 a 3 anos}  & 0.37\% & 0.48\% & 29.31\% \\ \hline
		\textbf{Posição no \textit{ranking} das matrículas}     & 83 & 72 & 11 \\ \hline
		\textbf{Posição no \textit{ranking} da demanda}         & 84 & 68 & 16 \\ \hline
	\end{tabular}
	\caption{Comparação entre o primeiro e o último período da amostra}
\end{table}
\begin{table}[H]
	\begin{tabular}{|l|l|l|l|}
		\hline
		\textbf{}                                 & \textbf{dezembro de 2006} & \textbf{dezembro de 2017} & \textbf{Variação} \\ \hline
		\textbf{Crianças na fila}                      & 474 & 141 & -70.25\% \\ \hline
		\textbf{População de 0 a 3 anos estimada}      & 2335 & 3099 & 32.72\% \\ \hline
		\textbf{Fila relativa à população}             & 10.11\% & 4.55\% & -55.0\% \\ \hline
	\end{tabular}
	\caption{Comparação da demanda no mês de dezembro, em 2006 e 2017.}
\end{table}
\section{Vila Jacuí}
\begin{figure}[H]
	\centering
	\includegraphics[width=0.66\linewidth]{../Analises/graficos/tempo/tempo_VILA_JACUI}
	\includegraphics[width=0.33\linewidth]{../Analises/mapas/mapas/distrito_VJA}
	\caption{Evolução das matrículas, da demanda, e da população de 0 a 3 anos no distrito Vila Jacuí e a sua localização no município, respectivamente.}
\end{figure}
\begin{table}[H]
	\begin{tabular}{|l|l|l|l|}
		\hline
		\textbf{}                                      & \textbf{junho de 2006}       & \textbf{dezembro de 2017}    & \textbf{Variação} \\ \hline
		\textbf{Número de matrículas}                  & 1109 & 4368 & 293.87\% \\ \hline
		\textbf{Crianças na fila}                      & 585 & 313 & -46.5\% \\ \hline
		\textbf{População de 0 a 3 anos estimada}      & 9765 & 7412 & -24.1\% \\ \hline
		\textbf{Matrículas relativas à população}      & 11.36\% & 58.93\% & 418.91\% \\ \hline
		\textbf{Fila relativa à população}             & 5.99\% & 4.22\% & -29.51\% \\ \hline
		\textbf{Proporção das matrículas no município} & 1.8\% & 1.47\% & -17.93\% \\ \hline
		\textbf{Proporção da demanda no município}     & 0.69\% & 0.71\% & 2.43\% \\ \hline
		\textbf{Proporção da população de 0 a 3 anos}  & 1.56\% & 1.15\% & -26.05\% \\ \hline
		\textbf{Posição no \textit{ranking} das matrículas}     & 15 & 24 & -9 \\ \hline
		\textbf{Posição no \textit{ranking} da demanda}         & 48 & 38 & 10 \\ \hline
	\end{tabular}
	\caption{Comparação entre o primeiro e o último período da amostra}
\end{table}
\begin{table}[H]
	\begin{tabular}{|l|l|l|l|}
		\hline
		\textbf{}                                 & \textbf{dezembro de 2006} & \textbf{dezembro de 2017} & \textbf{Variação} \\ \hline
		\textbf{Crianças na fila}                      & 1913 & 313 & -83.64\% \\ \hline
		\textbf{População de 0 a 3 anos estimada}      & 9765 & 7412 & -24.1\% \\ \hline
		\textbf{Fila relativa à população}             & 11.73\% & 4.22\% & -64.0\% \\ \hline
	\end{tabular}
	\caption{Comparação da demanda no mês de dezembro, em 2006 e 2017.}
\end{table}
\section{Vila Leopoldina}
\begin{figure}[H]
	\centering
	\includegraphics[width=0.66\linewidth]{../Analises/graficos/tempo/tempo_VILA_LEOPOLDINA}
	\includegraphics[width=0.33\linewidth]{../Analises/mapas/mapas/distrito_VLE}
	\caption{Evolução das matrículas, da demanda, e da população de 0 a 3 anos no distrito Vila Leopoldina e a sua localização no município, respectivamente.}
\end{figure}
\begin{table}[H]
	\begin{tabular}{|l|l|l|l|}
		\hline
		\textbf{}                                      & \textbf{junho de 2006}       & \textbf{dezembro de 2017}    & \textbf{Variação} \\ \hline
		\textbf{Número de matrículas}                  & 214 & 635 & 196.73\% \\ \hline
		\textbf{Crianças na fila}                      & 145 & 107 & -26.21\% \\ \hline
		\textbf{População de 0 a 3 anos estimada}      & 1675 & 1891 & 12.9\% \\ \hline
		\textbf{Matrículas relativas à população}      & 12.78\% & 33.58\% & 162.84\% \\ \hline
		\textbf{Fila relativa à população}             & 8.66\% & 5.66\% & -34.64\% \\ \hline
		\textbf{Proporção das matrículas no município} & 0.35\% & 0.21\% & -38.17\% \\ \hline
		\textbf{Proporção da demanda no município}     & 0.17\% & 0.24\% & 41.27\% \\ \hline
		\textbf{Proporção da população de 0 a 3 anos}  & 0.27\% & 0.29\% & 10.0\% \\ \hline
		\textbf{Posição no \textit{ranking} das matrículas}     & 77 & 79 & -2 \\ \hline
		\textbf{Posição no \textit{ranking} da demanda}         & 87 & 79 & 8 \\ \hline
	\end{tabular}
	\caption{Comparação entre o primeiro e o último período da amostra}
\end{table}
\begin{table}[H]
	\begin{tabular}{|l|l|l|l|}
		\hline
		\textbf{}                                 & \textbf{dezembro de 2006} & \textbf{dezembro de 2017} & \textbf{Variação} \\ \hline
		\textbf{Crianças na fila}                      & 243 & 107 & -55.97\% \\ \hline
		\textbf{População de 0 a 3 anos estimada}      & 1675 & 1891 & 12.9\% \\ \hline
		\textbf{Fila relativa à população}             & 12.84\% & 5.66\% & -55.93\% \\ \hline
	\end{tabular}
	\caption{Comparação da demanda no mês de dezembro, em 2006 e 2017.}
\end{table}
\section{Vila Maria}
\begin{figure}[H]
	\centering
	\includegraphics[width=0.66\linewidth]{../Analises/graficos/tempo/tempo_VILA_MARIA}
	\includegraphics[width=0.33\linewidth]{../Analises/mapas/mapas/distrito_VMR}
	\caption{Evolução das matrículas, da demanda, e da população de 0 a 3 anos no distrito Vila Maria e a sua localização no município, respectivamente.}
\end{figure}
\begin{table}[H]
	\begin{tabular}{|l|l|l|l|}
		\hline
		\textbf{}                                      & \textbf{junho de 2006}       & \textbf{dezembro de 2017}    & \textbf{Variação} \\ \hline
		\textbf{Número de matrículas}                  & 881 & 2559 & 190.47\% \\ \hline
		\textbf{Crianças na fila}                      & 959 & 450 & -53.08\% \\ \hline
		\textbf{População de 0 a 3 anos estimada}      & 6557 & 6834 & 4.22\% \\ \hline
		\textbf{Matrículas relativas à população}      & 13.44\% & 37.45\% & 178.69\% \\ \hline
		\textbf{Fila relativa à população}             & 14.63\% & 6.58\% & -54.98\% \\ \hline
		\textbf{Proporção das matrículas no município} & 1.43\% & 0.86\% & -39.48\% \\ \hline
		\textbf{Proporção da demanda no município}     & 1.14\% & 1.02\% & -10.17\% \\ \hline
		\textbf{Proporção da população de 0 a 3 anos}  & 1.04\% & 1.06\% & 1.55\% \\ \hline
		\textbf{Posição no \textit{ranking} das matrículas}     & 26 & 44 & -18 \\ \hline
		\textbf{Posição no \textit{ranking} da demanda}         & 28 & 26 & 2 \\ \hline
	\end{tabular}
	\caption{Comparação entre o primeiro e o último período da amostra}
\end{table}
\begin{table}[H]
	\begin{tabular}{|l|l|l|l|}
		\hline
		\textbf{}                                 & \textbf{dezembro de 2006} & \textbf{dezembro de 2017} & \textbf{Variação} \\ \hline
		\textbf{Crianças na fila}                      & 1720 & 450 & -73.84\% \\ \hline
		\textbf{População de 0 a 3 anos estimada}      & 6557 & 6834 & 4.22\% \\ \hline
		\textbf{Fila relativa à população}             & 13.91\% & 6.58\% & -52.66\% \\ \hline
	\end{tabular}
	\caption{Comparação da demanda no mês de dezembro, em 2006 e 2017.}
\end{table}
\section{Vila Mariana}
\begin{figure}[H]
	\centering
	\includegraphics[width=0.66\linewidth]{../Analises/graficos/tempo/tempo_VILA_MARIANA}
	\includegraphics[width=0.33\linewidth]{../Analises/mapas/mapas/distrito_VMN}
	\caption{Evolução das matrículas, da demanda, e da população de 0 a 3 anos no distrito Vila Mariana e a sua localização no município, respectivamente.}
\end{figure}
\begin{table}[H]
	\begin{tabular}{|l|l|l|l|}
		\hline
		\textbf{}                                      & \textbf{junho de 2006}       & \textbf{dezembro de 2017}    & \textbf{Variação} \\ \hline
		\textbf{Número de matrículas}                  & 259 & 676 & 161.0\% \\ \hline
		\textbf{Crianças na fila}                      & 335 & 225 & -32.84\% \\ \hline
		\textbf{População de 0 a 3 anos estimada}      & 4124 & 4987 & 20.93\% \\ \hline
		\textbf{Matrículas relativas à população}      & 6.28\% & 13.56\% & 115.84\% \\ \hline
		\textbf{Fila relativa à população}             & 8.12\% & 4.51\% & -44.46\% \\ \hline
		\textbf{Proporção das matrículas no município} & 0.42\% & 0.23\% & -45.62\% \\ \hline
		\textbf{Proporção da demanda no município}     & 0.4\% & 0.51\% & 28.58\% \\ \hline
		\textbf{Proporção da população de 0 a 3 anos}  & 0.66\% & 0.77\% & 17.82\% \\ \hline
		\textbf{Posição no \textit{ranking} das matrículas}     & 68 & 75 & -7 \\ \hline
		\textbf{Posição no \textit{ranking} da demanda}         & 69 & 45 & 24 \\ \hline
	\end{tabular}
	\caption{Comparação entre o primeiro e o último período da amostra}
\end{table}
\begin{table}[H]
	\begin{tabular}{|l|l|l|l|}
		\hline
		\textbf{}                                 & \textbf{dezembro de 2006} & \textbf{dezembro de 2017} & \textbf{Variação} \\ \hline
		\textbf{Crianças na fila}                      & 362 & 225 & -37.85\% \\ \hline
		\textbf{População de 0 a 3 anos estimada}      & 4124 & 4987 & 20.93\% \\ \hline
		\textbf{Fila relativa à população}             & 7.18\% & 4.51\% & -37.16\% \\ \hline
	\end{tabular}
	\caption{Comparação da demanda no mês de dezembro, em 2006 e 2017.}
\end{table}
\section{Vila Matilde}
\begin{figure}[H]
	\centering
	\includegraphics[width=0.66\linewidth]{../Analises/graficos/tempo/tempo_VILA_MATILDE}
	\includegraphics[width=0.33\linewidth]{../Analises/mapas/mapas/distrito_VMT}
	\caption{Evolução das matrículas, da demanda, e da população de 0 a 3 anos no distrito Vila Matilde e a sua localização no município, respectivamente.}
\end{figure}
\begin{table}[H]
	\begin{tabular}{|l|l|l|l|}
		\hline
		\textbf{}                                      & \textbf{junho de 2006}       & \textbf{dezembro de 2017}    & \textbf{Variação} \\ \hline
		\textbf{Número de matrículas}                  & 277 & 1825 & 558.84\% \\ \hline
		\textbf{Crianças na fila}                      & 404 & 135 & -66.58\% \\ \hline
		\textbf{População de 0 a 3 anos estimada}      & 5092 & 4855 & -4.65\% \\ \hline
		\textbf{Matrículas relativas à população}      & 5.44\% & 37.59\% & 591.01\% \\ \hline
		\textbf{Fila relativa à população}             & 7.93\% & 2.78\% & -64.95\% \\ \hline
		\textbf{Proporção das matrículas no município} & 0.45\% & 0.62\% & 37.28\% \\ \hline
		\textbf{Proporção da demanda no município}     & 0.48\% & 0.31\% & -36.03\% \\ \hline
		\textbf{Proporção da população de 0 a 3 anos}  & 0.81\% & 0.75\% & -7.1\% \\ \hline
		\textbf{Posição no \textit{ranking} das matrículas}     & 67 & 49 & 18 \\ \hline
		\textbf{Posição no \textit{ranking} da demanda}         & 65 & 69 & -4 \\ \hline
	\end{tabular}
	\caption{Comparação entre o primeiro e o último período da amostra}
\end{table}
\begin{table}[H]
	\begin{tabular}{|l|l|l|l|}
		\hline
		\textbf{}                                 & \textbf{dezembro de 2006} & \textbf{dezembro de 2017} & \textbf{Variação} \\ \hline
		\textbf{Crianças na fila}                      & 679 & 135 & -80.12\% \\ \hline
		\textbf{População de 0 a 3 anos estimada}      & 5092 & 4855 & -4.65\% \\ \hline
		\textbf{Fila relativa à população}             & 6.99\% & 2.78\% & -60.22\% \\ \hline
	\end{tabular}
	\caption{Comparação da demanda no mês de dezembro, em 2006 e 2017.}
\end{table}
\section{Vila Medeiros}
\begin{figure}[H]
	\centering
	\includegraphics[width=0.66\linewidth]{../Analises/graficos/tempo/tempo_VILA_MEDEIROS}
	\includegraphics[width=0.33\linewidth]{../Analises/mapas/mapas/distrito_VMD}
	\caption{Evolução das matrículas, da demanda, e da população de 0 a 3 anos no distrito Vila Medeiros e a sua localização no município, respectivamente.}
\end{figure}
\begin{table}[H]
	\begin{tabular}{|l|l|l|l|}
		\hline
		\textbf{}                                      & \textbf{junho de 2006}       & \textbf{dezembro de 2017}    & \textbf{Variação} \\ \hline
		\textbf{Número de matrículas}                  & 727 & 3213 & 341.95\% \\ \hline
		\textbf{Crianças na fila}                      & 802 & 247 & -69.2\% \\ \hline
		\textbf{População de 0 a 3 anos estimada}      & 7116 & 6775 & -4.79\% \\ \hline
		\textbf{Matrículas relativas à população}      & 10.22\% & 47.42\% & 364.2\% \\ \hline
		\textbf{Fila relativa à população}             & 11.27\% & 3.65\% & -67.65\% \\ \hline
		\textbf{Proporção das matrículas no município} & 1.18\% & 1.08\% & -7.91\% \\ \hline
		\textbf{Proporção da demanda no município}     & 0.95\% & 0.56\% & -41.04\% \\ \hline
		\textbf{Proporção da população de 0 a 3 anos}  & 1.13\% & 1.05\% & -7.24\% \\ \hline
		\textbf{Posição no \textit{ranking} das matrículas}     & 34 & 34 & 0 \\ \hline
		\textbf{Posição no \textit{ranking} da demanda}         & 34 & 42 & -8 \\ \hline
	\end{tabular}
	\caption{Comparação entre o primeiro e o último período da amostra}
\end{table}
\begin{table}[H]
	\begin{tabular}{|l|l|l|l|}
		\hline
		\textbf{}                                 & \textbf{dezembro de 2006} & \textbf{dezembro de 2017} & \textbf{Variação} \\ \hline
		\textbf{Crianças na fila}                      & 1435 & 247 & -82.79\% \\ \hline
		\textbf{População de 0 a 3 anos estimada}      & 7116 & 6775 & -4.79\% \\ \hline
		\textbf{Fila relativa à população}             & 10.92\% & 3.65\% & -66.61\% \\ \hline
	\end{tabular}
	\caption{Comparação da demanda no mês de dezembro, em 2006 e 2017.}
\end{table}
\section{Vila Prudente}
\begin{figure}[H]
	\centering
	\includegraphics[width=0.66\linewidth]{../Analises/graficos/tempo/tempo_VILA_PRUDENTE}
	\includegraphics[width=0.33\linewidth]{../Analises/mapas/mapas/distrito_VPR}
	\caption{Evolução das matrículas, da demanda, e da população de 0 a 3 anos no distrito Vila Prudente e a sua localização no município, respectivamente.}
\end{figure}
\begin{table}[H]
	\begin{tabular}{|l|l|l|l|}
		\hline
		\textbf{}                                      & \textbf{junho de 2006}       & \textbf{dezembro de 2017}    & \textbf{Variação} \\ \hline
		\textbf{Número de matrículas}                  & 774 & 2340 & 202.33\% \\ \hline
		\textbf{Crianças na fila}                      & 686 & 117 & -82.94\% \\ \hline
		\textbf{População de 0 a 3 anos estimada}      & 4775 & 4635 & -2.93\% \\ \hline
		\textbf{Matrículas relativas à população}      & 16.21\% & 50.49\% & 211.46\% \\ \hline
		\textbf{Fila relativa à população}             & 14.37\% & 2.52\% & -82.43\% \\ \hline
		\textbf{Proporção das matrículas no município} & 1.25\% & 0.79\% & -37.01\% \\ \hline
		\textbf{Proporção da demanda no município}     & 0.81\% & 0.27\% & -67.35\% \\ \hline
		\textbf{Proporção da população de 0 a 3 anos}  & 0.76\% & 0.72\% & -5.43\% \\ \hline
		\textbf{Posição no \textit{ranking} das matrículas}     & 30 & 46 & -16 \\ \hline
		\textbf{Posição no \textit{ranking} da demanda}         & 40 & 76 & -36 \\ \hline
	\end{tabular}
	\caption{Comparação entre o primeiro e o último período da amostra}
\end{table}
\begin{table}[H]
	\begin{tabular}{|l|l|l|l|}
		\hline
		\textbf{}                                 & \textbf{dezembro de 2006} & \textbf{dezembro de 2017} & \textbf{Variação} \\ \hline
		\textbf{Crianças na fila}                      & 676 & 117 & -82.69\% \\ \hline
		\textbf{População de 0 a 3 anos estimada}      & 4775 & 4635 & -2.93\% \\ \hline
		\textbf{Fila relativa à população}             & 17.42\% & 2.52\% & -85.51\% \\ \hline
	\end{tabular}
	\caption{Comparação da demanda no mês de dezembro, em 2006 e 2017.}
\end{table}
\section{Vila Sônia}
\begin{figure}[H]
	\centering
	\includegraphics[width=0.66\linewidth]{../Analises/graficos/tempo/tempo_VILA_SONIA}
	\includegraphics[width=0.33\linewidth]{../Analises/mapas/mapas/distrito_VSO}
	\caption{Evolução das matrículas, da demanda, e da população de 0 a 3 anos no distrito Vila Sônia e a sua localização no município, respectivamente.}
\end{figure}
\begin{table}[H]
	\begin{tabular}{|l|l|l|l|}
		\hline
		\textbf{}                                      & \textbf{junho de 2006}       & \textbf{dezembro de 2017}    & \textbf{Variação} \\ \hline
		\textbf{Número de matrículas}                  & 546 & 3097 & 467.22\% \\ \hline
		\textbf{Crianças na fila}                      & 706 & 580 & -17.85\% \\ \hline
		\textbf{População de 0 a 3 anos estimada}      & 5412 & 6540 & 20.84\% \\ \hline
		\textbf{Matrículas relativas à população}      & 10.09\% & 47.35\% & 369.38\% \\ \hline
		\textbf{Fila relativa à população}             & 13.05\% & 8.87\% & -32.02\% \\ \hline
		\textbf{Proporção das matrículas no município} & 0.88\% & 1.05\% & 18.19\% \\ \hline
		\textbf{Proporção da demanda no município}     & 0.84\% & 1.32\% & 57.27\% \\ \hline
		\textbf{Proporção da população de 0 a 3 anos}  & 0.86\% & 1.02\% & 17.74\% \\ \hline
		\textbf{Posição no \textit{ranking} das matrículas}     & 43 & 37 & 6 \\ \hline
		\textbf{Posição no \textit{ranking} da demanda}         & 39 & 19 & 20 \\ \hline
	\end{tabular}
	\caption{Comparação entre o primeiro e o último período da amostra}
\end{table}
\begin{table}[H]
	\begin{tabular}{|l|l|l|l|}
		\hline
		\textbf{}                                 & \textbf{dezembro de 2006} & \textbf{dezembro de 2017} & \textbf{Variação} \\ \hline
		\textbf{Crianças na fila}                      & 1358 & 580 & -57.29\% \\ \hline
		\textbf{População de 0 a 3 anos estimada}      & 5412 & 6540 & 20.84\% \\ \hline
		\textbf{Fila relativa à população}             & 9.7\% & 8.87\% & -8.57\% \\ \hline
	\end{tabular}
	\caption{Comparação da demanda no mês de dezembro, em 2006 e 2017.}
\end{table}