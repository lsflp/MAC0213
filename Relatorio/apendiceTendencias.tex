\chapter{Listas sobre as Tendências}
\label{apend:tendencias}

\lettrine{N}{este} apêndice encontram-se três listas, referentes ao \autoref{cap:tendencias}. Elas estão em um formato parecido com os do \autoref{apend:rankings}.

Todas as listas estão em formato de tabela. Elas possuem quatro colunas:
 
\begin{itemize}
	\item \textbf{Posição}: a posição que o distrito fica em relação aos outros em relação a atingir determinada condição.
	\item \textbf{Distrito}: o nome do distrito analisado.
	\item \textbf{Zona}: a zona em que o distrito analisado se localiza.
	\item \textbf{Quando}: momento, no formato MM/AAAA, em que o distrito atinge (ou atingirá) determinada condição.
\end{itemize}

"Determinada condição", escrita acima, se refere à inversão, ou o atingimento da meta da demanda ou das matrículas.

Nas segunda e terceira tabelas, a última coluna apresenta o valor ''2031-03'' nas últimas linhas. Isso não quer dizer que o distrito vai atingir a meta nesse período, e sim \textbf{a partir} desse período, uma vez que a previsão foi feita até dezembro de 2030.